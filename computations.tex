\section{Computational Model}

In this section we describe our computational model for protocols, which is loosely based on the data-flow paradigm~\cite{abadi16} and a simplified UC model~\cite{canetti-cohen-lindell15}. Concretely, protocols are expressed as graphs where nodes represent operations to be performed by a specific party, and edges present values ``flowing'' between operations.

The main motivation for using the data-flow paradigm is that it leads to a very natural concurrent execution model, where in the extreme we can see each node as being executed by a separate task (e.g. green thread or actor). We take full advantage of this in the runtime which is based on the asynchronous execution paradigm. The reason for basing our computational model on the UC model is that it is a well-known paradigm for ensuring security under concurrent composition.


\subsection{Sessions}

Every execution of a graph is performed under a unique session id $\sid$ used to identify values and ensure isolation when running protocols concurrently. As we shall see in more detail later, session ids are for instance used to non-interactively derive nonces and sample correlated randomness by the secret sharing schemes.

Session ids must be unique and of fixed length for security reasons, but can otherwise safely be chosen by an untrusted coordinator, for instance by sampling a random string. To satisfy the security requirements, each party maintains a list of previous session ids in which it has engaged, and refuses to re-run any computation using those ids; this prevents for instance replay and selective failure attacks. It additionally checks that session ids have the correct length.


\subsection{Sub-protocols}

We allow graphs to call sub-graphs similar to calling sub-routines. When doing so, the sub-graph is executed under a sub-session id $\sid'$ derived from $\sid$ and an activation key $\mathsf{ac\_key}$ statically related to the call site:
$$
\sid' = \mathsf{h}( \sid \| \mathsf{ac\_key} )
$$
with $\mathsf{h}$ being a secure hash function and $\|$ denoting string concatenation. For security we require $\sid$ to have fixed length $\ell$. Calling a sub-graph is done through $\mathsf{Enter}$ and $\mathsf{Exit}$ operations that are implicitly linked to $\mathsf{Input}$ and $\mathsf{Output}$ operations.


\subsection{Communication}

Transmission of values between parties is done using $\mathsf{Send}$ and $\mathsf{Receive}$ nodes where each pair is linked together by a static $\mathsf{rdv\_key}$ attribute. Together with the session id, this allows us to uniquely identify all values by tagging them with $(\sid, \mathsf{rdv\_key})$ during transmission.

In some configurations we also use this pair to derive unique nonces for encrypting messages (see Section~\ref{sec:infrastructure}). This is done using a secure hash function as $\mathsf{nonce} = \mathsf{h}(\sid \| \mathsf{rdv\_key})$ and security again depends on $\sid$ having fixed length $\ell$.


\subsection{Inlining}

In the presentation given in this paper we make heavy use of calling sub-protocols, yet for performance reasons it may be interesting to inline sub-graphs. To do so securely, special attention must be paid to certain node attributes that control uniqueness.

As an example, the $\mathsf{rdv\_key}$ attribute of $\mathsf{Send}$ and $\mathsf{Receive}$ nodes in the graph being inlined must be updated to $\mathsf{h}(\mathsf{ac\_key} \| \mathsf{rdv\_key})$ where $\mathsf{ac\_key}$ is the activation key of the $\mathsf{Enter}$ and $\mathsf{Exit}$ nodes being replaced, and $\mathsf{rdv\_key}$ in the graph being inlined into must be updated to $\mathsf{h}(\mathsf{rdv\_key})$. Other examples are the $\mathsf{sync\_key}$ attribute of $\mathsf{DeriveSeed}$ operations and the $\mathsf{ac\_key}$ attribute of $\mathsf{Enter}$ and $\mathsf{Exit}$ operations. For this to be secure we require all $\mathsf{ac\_key}$, $\mathsf{rdv\_key}$, and $\mathsf{sync\_key}$ to be of fixed length $\ell$.
