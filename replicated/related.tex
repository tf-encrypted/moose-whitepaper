
One reason for choosing~\cite{CCS:AFLNO16} as our foundation is that it is currently the fastest protocol for achieving semi-honest three-party multiplications in the preprocessing model using ring arithmetic.
%An year after it was shown by \cite{SP:ABFLLN17} that the boolean version of~\cite{CCS:AFLNO16} can be converted into a protocol with active security that achieves $1$ billion AND triples per second with a cost of $7$ bits per AND gate.
%
Moreover, its ring variant can be efficiently upgraded to malicious security by (roughly) just repeating all procedures twice and use a "zero check" as described in~\cite{cryptoeprint:2020:1330}.

Finally, we also chose this line of work due to its efficiency at computing dot products by communicating a number of ring elements independent in the size of input tensors, and other optimizations made possible by being in the three-party setting.
%Note that the work of efficient dot products in MPC was inspired mainly from~\cite{CCS:LinNof17,C:CGHIKL18,EPRINT:ADEN19,cryptoeprint:2020:1330}.

% Later it was shown by Keller et al. \cite{SCN:KRSW18} how to extend the
% work of Araki et al. \cite{CCS:AFLNO16} to multiple parties assuming there is
% honest majority using some well-known techniques by Maurer
% \cite{SCN:Maurer02}. The caveat of Keller et al. \cite{SCN:KRSW18} is that
% there can be an exponential blowup in terms of $n \choose t$ where $t<n/2$ is
% the corruption threshold and $n$ the number of parties.
% To get a better picture of the available protocols with their underlying
% arithmetic we list the current state of the art protocols for obtaining
% honest-majority MPC. For readability purposes we abbreviate honest majority
% with HM:

% \begin{enumerate}
%   \item HM over large fields $\Fp$ \cite{CCS:LinNof17,C:CGHIKL18}. Although
%   the two constructions listed above are specialized for the malicious
%   setting (at most 1 party arbitrarily deviating from the protocol) these are
%   built assuming $\Frand, \Fmult$ hybrid model where $\Fmult$ needs to be a
%   multiplication protocol secure up to additive attacks. To get such an
%   $\Fmult$ we can simply take
%   a semi-honest protocol for HM for doing multiplications such as \cite{CCS:AFLNO16}.
%   \item HM for binary circuits $\F_2$. Note that Araki et al. \cite{CCS:AFLNO16}
%   works for binary circuits as well for the semi-honest case.
%   To get malicious security Chida et al. \cite{C:CGHIKL18} present a protocol
%   for small field multiplications using cut-and-choose techniques The cut and
%   choose framework works by generating large batches of triples
%   optimistically and then open a fraction to detect whether some cheating
%   occurred.
%   \item HM over small fields (eg. $\F_{2^8}$ which is suitable for AES). To
%   get malicious security, depending on the extension field size, we can
%   either repeat the computation $\sec/|\F|$ times or generate triples using cut and choose
%   from \cite{C:CGHIKL18}.
%   \item HM over rings $\mathbb{Z}_{2^k}$. Abspoel et al. \cite{EPRINT:ADEN19}
%   are the first ones who introduce vector dot-products of secret shares at a cost of
%   communicating a constant number of ring elements independently of the vector sizes
%   for the ring case - however they avoid giving any hint on how to realise
%   reactive computations, this is solved later in \cite{cryptoeprint:2020:1330}.
%   \item HM by mixing circuits: \cite{CCS:MohRin18,cryptoeprint:2020:1330}.
%   Escudero et al. \cite{C:EGKRS20} work in the more generic setting but they
%   have
%   improvements for the honest majority as well. They improve the mixed framework
%   specifically for $\Z_{2^k}$ and $\F_2$ very recently in \cite{cryptoeprint:2020:1330}.

% \end{enumerate}

% Note that almost every protocol can be built in the malicious setting using
% building blocks from the semi-honest protocols with the exception of binary
% triples $c = a \cdot b$ where $a,b \in \F_2$. For the binary triple case we
% need more complex techniques such as cut and choose. In our implementation we
% stick with Araki et al. protocol \cite{CCS:AFLNO16} while borrowing some
% improvements from \cite{cryptoeprint:2020:1330} such as probabilistic
% truncation.

% We
% should keep in mind that although triple generation for large fields is
% usually faster than rings there are benefits when working with ring sharings
% - such as faster methods for secret truncation which in turn makes RELU's
% faster.
