\subsection{Sharing data}

In Figure~\ref{fig:replicated-share} we show how a party $P_i$ can translate a private tensor $\vx$ (known only to $P_i$) into a replicated tensor. In the first part of the protocol parties use the same mechanism as in the $\mathsf{ZeroShare}$ to have a set of replicated shared seeds. Then $P_i$ sends to $P_{i-1}$ the shape $\vx$ in order for $P_{i-1}$ to derive a correct sized random tensor using the shared seed. In the last part the inputting party $P_i$ masks the private tensor with a random value $\vx - \vx_i^i$ and sends it to $P_{i+1}$.

Correctness can be seen from the fact that at the end of the protocol all $\vx_1^j + \vx_2^j + \vx_3^{j+1} = \vx$ for all $j \in [1,3]$. For example, when $j = 1$ we have the following:
$$\vx_1^1 + \vx_2^1 + \vx_3^2 = \mathbf{0} + \vx_2^1 + (\vx - \vx_2^2) = \vx $$ due to $\vx_2^1 = \vx_2^2$ since they were sampled from identical seeds. Security follows from~\cite{CCS:ABFLNO16} or~\cite{CCS:MohRin18}.

\subsection{Sharing an input}


\begin{Boxfig}{Replicated share}{fig:replicated-share}
  {Protocol $\mathsf{RepShare}[P_1, P_2, P_3](R; P_i, \vx)$}
  
  \begin{enumerate}
  \item Suppose each party $P_j$ has a shared seeds $(\seed_j^j, \seed_{j+1}^j)$.
  \item Divide the set of parties into two sets: $I = \{P_i\}, O = \{P_1, P_2, P_3\} - \{P_i\}$.
  \item On Host($P_{i-1}$):
  \begin{enumerate}
    \item $\vx_{i-1}^{i-1} = \mathsf{zero}(\shape(\vx))$.
    \item Compute $\vx_i^{i-1} = \RingSample(R; \shape(\vx), \seed_i^{i-1})$.
  \end{enumerate}
  \item On Host($P_i$):
  \begin{enumerate}
    \item $\vx_i^i = \RingSample(R; \shape(\vx), \seed_{i}^i)$.
    \item $\vx_{i+1}^i = \vx - \vx_i^i$.
    \item Send $\vx_{i+1}^i$ to $P_{i+1}$.
  \end{enumerate}
  \item On Host($P_{i+1}$):
  \begin{enumerate}
    \item Receive $\vx_{i+1}^i$ from $P_i$. Set $\vx_{i+1}^{i+1} = \vx_{i+1}^i$.
    \item $\vx_{i+2}^{i+1} = \mathsf{zero}(\shape(\vx))$.
  \end{enumerate}
   \item Return $\share{\vx}_R = \left( (\vx_1^1, \vx_2^1), (\vx_2^2, \vx_3^2), (\vx_3^3, \vx_1^3) \right)$.
  \end{enumerate}
\end{Boxfig}


