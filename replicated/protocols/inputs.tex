\subsection{Sharing data}

In Figure~\ref{fig:replicated-share} we show how a party $P_i$ can translate a private tensor $\vx$ (known only to $P_i$) into a replicated tensor. In the first part of the protocol parties use the same mechanism as in the $\mathsf{ZeroShare}$ to have a set of replicated shared seeds. Then $P_i$ sends to $P_{i-1}$ the shape $\vx$ in order for $P_{i-1}$ to derive a correct sized random tensor using the shared seed. In the last part the inputting party $P_i$ masks the private tensor with a random value $\vx - \vx_i^i$ and sends it to $P_{i+1}$.

Correctness can be seen from the fact that at the end of the protocol all $\vx_1^j + \vx_2^j + \vx_3^{j+1} = \vx$ for all $j \in [1,3]$. For example, when $j = 1$ we have the following:
$$\vx_1^1 + \vx_2^1 + \vx_3^2 = \mathbf{0} + \vx_2^1 + (\vx - \vx_2^2) = \vx $$ due to $\vx_2^1 = \vx_2^2$ since they were sampled from identical seeds. Security follows from~\cite{CCS:ABFLNO16} or~\cite{CCS:MohRin18}.


\begin{Boxfig}{Replicated sharing protocol.}{fig:replicated-share}
  {Protocol $\mathsf{ReplicatedShare}_{[P_1, P_2, P_3]}(R; \prfkeys, \vx)$}
  
  Let $P_i$ be the player holding $\vx$.
  
  \begin{enumerate}
  \item Let $\left( (k_1^1, k_2^1), (k_2^2, k_3^2), (k_2^3, k_1^3) \right) = \prfkeys$ be PRF keys generated during setup.
  
  \item On $P_i$:
  \begin{enumerate}
    \item $\shape^i \asn \mathsf{Shape}(\vx)$.
    \item ${\seed}^i \asn \mathsf{DeriveSeed}(0; k^i_i)$.
    \item $\vx_i^i \asn \RingSample(R; \shape^i, \seed^i)$.
    \item $\vx_{i+1}^i \asn \vx - \vx_i^i$.
    \item Send $\shape^i$ to $P_{i-1}$ and $P_{i+1}$.
    \item Send $\vx_{i+1}^i$ to $P_{i+1}$.
  \end{enumerate}
  
  \item On $P_{i-1}$:
  \begin{enumerate}
    \item Receive $\shape^{i-1}$ from $P_i$.
    \item ${\seed}^{i-1} \asn \mathsf{DeriveSeed}(0; k^{i-1}_i)$.
    \item $\vx_{i-1}^{i-1} \asn \mathsf{Zeros}(R; \shape^{i-1})$.
    \item $\vx_i^{i-1} \asn \RingSample(R; \shape^{i-1}, \seed^{i-1})$.
  \end{enumerate}
  
  \item On $P_{i+1}$:
  \begin{enumerate}
    \item Receive $\shape^{i+1}$ from $P_i$.
    \item Receive $\vx_{i+1}^i$ from $P_i$.
    \item $\vx_{i+1}^{i+1} \asn \vx_{i+1}^i$.
    \item $\vx_{i+2}^{i+1} \asn \mathsf{Zeros}(R; \shape^{i+1})$.
  \end{enumerate}
  
  \item Return $\share{\vx}_R = \left( (\vx_1^1, \vx_2^1), (\vx_2^2, \vx_3^2), (\vx_3^3, \vx_1^3) \right)$.
  \end{enumerate}
\end{Boxfig}


