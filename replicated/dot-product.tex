\subsection{Dot products}

Due to the bilinearity of the dot-product operations
one can simply replace the calls of ring tensor multiplication with
calls to ring tensor dot product and the protocol will be valid as long as
$\valpha_1, \valpha_2, \valpha_3$ have matching dimensions.
% randomize $\vx \cdot \vy$.
For more details, the reader can check the
\verb|DotOp| implementation in \verb|moose|.

% \subsubsection{3-party malicious (HM)}
% Note that the first progress
% for this problem was done for the field case by Chida et al. \cite{C:CGHIKL18}.
% For the ring case see \cite{EPRINT:ADEN19,cryptoeprint:2020:1330}.

% \subsubsection{Dishonest majority}

% Here parties fetch a triple for each $\share{x_k} \cdot \share{y_k}$ multiplication
% ending with a cost of $m$ preprocessed triples. This is valid for secret sharing (SPDZ-type
% \cite{EC:KelPasRot18,C:DPSZ12,C:CDESX18})
% but also for garbled circuits (\cite{AC:HazSchSor17,CCS:WanRanKat17b}) type protocols.

