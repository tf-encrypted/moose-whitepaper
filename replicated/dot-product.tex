\subsection{Secret dot-products using MPC}

The goal is to compute $\sum_{k=1}^m \share{x_k} \cdot \share{y_k}$.

\subsubsection{3-party semi-honest (HM)}

Although this was written with $x$ and $y$ being a single ring element,
i.e. $x,y \in \Z_{2^k}$ the same protocols works when $x, y \in \Z^m_{2^k}$
and the ring multiplication operator `$\cdot$` is replaced with the dot
product. In this manner $x \cdot y = \sum_{i=1}^m x_i y_i \in \Z_{2^k}$ while
$x, y \in \Z^m_{2^k}$. This can be extended for arbitrary length tensors as long as
$\alpha, \beta, \gamma$ have the correct dimensions to mask randomize $x \cdot y$.

\subsubsection{3-party malicious (HM)}
Note that the first progress
for this problem was done for the field case by Chida et al. \cite{C:CGHIKL18}.
For the ring case see \cite{EPRINT:ADEN19,cryptoeprint:2020:1330}.

\subsubsection{Dishonest majority}

Here parties fetch a triple for each $\share{x_k} \cdot \share{y_k}$ multiplication
ending with a cost of $m$ preprocessed triples. This is valid for secret sharing (SPDZ-type
\cite{EC:KelPasRot18,C:DPSZ12,C:CDESX18})
but also for garbled circuits (\cite{AC:HazSchSor17,CCS:WanRanKat17b}) type protocols.

