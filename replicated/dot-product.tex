\subsection{Secret dot-products using MPC}

The goal is to compute $\sum_{k=1}^m \share{x_k} \cdot \share{y_k}$.

\subsubsection{3-party semi-honest (HM)}
There are different ways to achieve this, depending on the secret sharing
type. Throughout this document we will stick with replicated secret sharing
although this works for Shamir Secret sharing. Both replicated and Shamir
work for the field and ring case, however for Shamir secret sharing it's a
bit more complicated to do MPC over rings - it is doable but with some
overhead \cite{EPRINT:CraRamXin19, TCC:ACDEY19}. To see how this works for
replicated secret sharing each party $P_i$ holds a sharing of $x$ as $(x_i,
x_{(i+1) \bmod 3})$ and a sharing of $y$ as $(y_i, y_{ (i+1) \bmod 3})$.
Since players offsets start with $1$:

\begin{itemize}
    \item $P_1$ has $(x_1, x_2)$ and $(y_1, y_2)$.
    \item $P_2$ has $(x_2, x_3)$ and $(y_2, y_3)$.
    \item $P_3$ has $(x_3, x_1)$ and $(y_3, y_1)$.
\end{itemize}

\noindent
To compute a sharing $\share{x \cdot y}$ using the shares of $x$ and $y$ parties do the following:

\begin{itemize}
    \item $P_1$ sets $(x \cdot y)_{1} \asn x_1 \cdot y_1 + x_1 \cdot y_2 + x_2 \cdot
    y_1 + \alpha$.
    \item $P_2$ sets $(x \cdot y)_2 \asn x_2 \cdot y_2 + x_2 \cdot y_3 + x_3
    \cdot y_2 + \beta$.
    \item $P_3$ sets $(x \cdot y)_3 \asn x_3 \cdot y_3 + x_3 \cdot y_1 + x_1 \cdot y_3 +
    \gamma$.
\end{itemize}

\noindent In the next phase parties send around the shares
\begin{itemize}
    \item $P_1$ sends privately $(x \cdot y)_1$ to $P_2$.
    \item $P_2$ sends privately $(x \cdot y)_2$ to $P_3$.
    \item $P_3$ sends privately $(x \cdot y)_3$ to $P_1$.
\end{itemize}

\noindent In the final step parties set their shares of $x \cdot y$ in the following way:
\begin{itemize}
    \item $P_1$ sets $\share{x \cdot y}$ as $(x\cdot y)_3, (x \cdot y)_1$.
    \item $P_2$ sets $\share{x \cdot y}$ as $(x\cdot y)_1, (x \cdot y)_2$.
    \item $P_3$ sets $\share{x \cdot y}$ as $(x\cdot y)_2, (x \cdot y)_3$.
\end{itemize}

Although this was written with $x$ and $y$ being a single ring element,
i.e. $x,y \in \Z_{2^k}$ the same protocols works when $x, y \in \Z^m_{2^k}$
and the ring multiplication operator `$\cdot$` is replaced with the dot
product. In this manner $x \cdot y = \sum_{i=1}^m x_i y_i \in \Z_{2^k}$ while
$x, y \in \Z^m_{2^k}$. This can be extended for arbitrary length tensors as long as
$\alpha, \beta, \gamma$ have the correct dimensions to mask randomize $x \cdot y$.

\subsubsection{3-party malicious (HM)}
Note that the first progress
for this problem was done for the field case by Chida et al. \cite{C:CGHIKL18}.
For the ring case see \cite{EPRINT:ADEN19,cryptoeprint:2020:1330}.

\subsubsection{Dishonest majority}

Here parties fetch a triple for each $\share{x_k} \cdot \share{y_k}$ multiplication
ending with a cost of $m$ preprocessed triples. This is valid for secret sharing (SPDZ-type
\cite{EC:KelPasRot18,C:DPSZ12,C:CDESX18})
but also for garbled circuits (\cite{AC:HazSchSor17,CCS:WanRanKat17b}) type protocols.

