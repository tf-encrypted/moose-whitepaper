\subsection{Comparison}

%In order to compute the MAPE (mean absolute percentage error) metric which boils down to compute the absolute value of a secret $\share{|\vx|}$. Since our primitives work with ring arithmetic one of its most efficient evaluation of the absolute value is $\share{|\vx|} = \share{\vx} \cdot \share{\vx > 0}$.

The problem of computing secure comparisons translates directly into computing a sharing of the most significant bit $\share{\mathsf{msb}(\vx)}$.
In the 3PC semi-honest model there are few approaches to this:
\begin{enumerate}
   \item ABY3 \cite{CCS:MohRin18} The main idea is for
   each party to locally bit-decompose their shares over $\Z_{2^k}$ and then
   reconstruct the secret modulo $\Z_{2^k}$ using shares from $\Z^k_2$ and a
   binary adder. Once a boolean sharing of the MSB is computed using the binary adder
   this is converted to a ring sharing in $\Z_{2^k}$.
   In ABY3 this was done using a three-party OT protocol. In MP-SPDZ \cite{CCS:Keller20}
   the boolean to ring sharing conversion was achieved using a daBit.
   \item SecureNN \cite{PoPETS:WagGupCha19}
   similar to ABY3 with the downside that it uses arithmetic modulo
   some small fields but avoids ring-to-boolean conversion.
  \item Comparisons using edaBits \cite{C:EGKRS20}. The 3PC case has roughly the same cost as ABY3.
\end{enumerate}

We use the MSB extraction protocol from ABY3 with a Kogge-Stone binary adder and minor optimizations for tensor operations. We avoid the special three-party OT that ABY3 had along with edaBits preprocessing by introducing our custom protocol \textsf{B2A} for binary to ring share conversion in Figure~\ref{fig:b2a-protocol}.
The MSB protocol can be found in Figure~\ref{fig:msb-protocol} where $\overline{\cdot}$ is used to denote a vector with $k$ elements indexed using $(i)$ for $i \in [k]$.




\begin{Boxfig}{Ring bit decomposition to binary shares}{fig:bitdecraw}
  {$\mathsf{BitDecRaw}_{[P_1, P_2, P_3]} \left( \share{\vx}^\mathsf{rep}_{2^k} \right) \rightarrow [\share{\cdot}^\mathsf{rep}_2; k]$}
  \begin{enumerate}
  \item Let $\left( (\vx_1^1, \vx_2^1), (\vx_2^2, \vx_3^2), (\vx_3^3, \vx_1^3) \right) = \share{\vx}^{\mathsf{rep}}_{2^k}$.

  \item On $P_1$:
  \begin{enumerate}
    \item $\overline{\va} \asn \textsf{LocalBitDec}(\vx^1_1 + \vx^1_2)$.
    \item $\overline{\share{\va}_2}(i) \asn \mathsf{ReplicatedShare}(\Z_2; \overline{\va}(i))$ for $i \in [k]$.
  \end{enumerate}

  \item On $P_2$:
  \begin{enumerate}
      \item $\overline{\vb^2} \asn \textsf{LocalBitDec}(\vx_3^2)$.
  \end{enumerate}

  \item On $P_3$:
  \begin{enumerate}
      \item $\overline{\vb^3} \asn \textsf{LocalBitDec}(\vx_3^3)$.
  \end{enumerate}
  \item Let $\overline{\share{\vb}_2}(i) = \left( (0, 0), (0, \overline{\vb^2}(i)), (\overline{\vb^3}(i), 0) \right)$ for $i \in [k]$.
  \item Return $\textsf{BinaryAdder}(\overline{\share{\va}_2}, \overline{\share{\vb}_2})$.
\end{enumerate}

\end{Boxfig}


\begin{Boxfig}{MSB computation from a replicated ring share}{fig:msb-protocol}
  {$\mathsf{MSB}_{[P_1, P_2, P_3]} \left( \share{\vx}^\mathsf{rep}_{2^k} \right)$}
  \begin{enumerate}
  \item $\overline{\share{\vb}} \asn \mathsf{BitDecRaw}(\share{\vx})$.
  \item Return $\textsf{B2A}(\overline{\share{\vb}_2}(k-1))$.
  \end{enumerate}

\end{Boxfig}



\subsubsection{Improved ABY3 boolean to ring sharing protocol}
When the inputs are tensors we can perform the share conversion $\share{\cdot}_2
\mapsto \share{\cdot}_{2^k}$ ($\mathsf{B2A}$ function)
more efficient by making use of the fact that all parties
follow the protocol specifications. The starting point is using a similar idea
from daBit/edaBit \cite{INDOCRYPT:RotWoo19,C:EGKRS20}
line of work with the twist that $P_3$ generates the preprocessing
material. We fully describe this share conversion protocol in Figure~\ref{fig:b2a-protocol}.

Acting as a trusted third party, $P_3$ generates a random daBit $(\share{\vb}_2,
\share{\vb}_{2^k})$ locally and shares it to $P_1$ and $P_2$. Then $P_1$ and
$P_2$ run a two-party protocol to convert $\share{\vx}_2$ to $\share{\vx}_{2^k}$
by computing $\vc \asn \vx_2 \oplus \vb_2$ and then locally XOR-ing in the
arithmetic domain $\vx_{2^k} = \vc + \vb_{2^k} - 2 \cdot \vc \cdot \vb_{2^k}$.
Finally they convert back to a replicated sharing using
$\mathsf{AdtToReplicated}$ from Figure~\ref{fig:two-to-three}. Note that in our
implementation of the additive to replicated share conversions we consider a
more general variant where $\share{\vx}^\mathsf{adt}$ is converted to
$\share{\vx}^\mathsf{rep}$ where the host placements on the additive placement
$\mathsf{adt}$ are not necessarily a subset of host placements composing $\mathsf{rep}$.

\begin{Boxfig}{Binary to arithmetic share conversion computation.}{fig:b2a-protocol}
  {$\mathsf{B2A}_{[P_1, P_2, P_3]} \left( \shareB{\vx}^\mathsf{rep} \right)$}
  \begin{enumerate}
  \item Let $\left( (\vx_1^1, \vx_2^1), (\vx_2^2, \vx_3^2), (\vx_3^3, \vx_1^3) \right) = \shareB{\vx}^\mathsf{rep}$.
  \item On $P_1$:
  \begin{enumerate}
  \item $\vx^1_{12} \asn \vx_1^1 \oplus \vx_2^1$.
  \item $\shape \asn \mathsf{Shape}(\vx^1_1)$.
  \end{enumerate}
  \item Let $\shareB{\vx}^\mathsf{add} = (\vx^1_{12}, \vx_3^2)$.
  \item $\shareB{\vb}^\mathsf{add}, \shareR{\vb}^\mathsf{add} \asn \textsf{DaBit}_{[D=P_3, P_1, P_2]}(\shape)$.
  \item $\shareB{\vc}^{\mathsf{add}} \asn \shareB{\vb}^\mathsf{add} \oplus \shareB{\vx}^{\mathsf{add}}$.
  \item $\vc \asn \Open_{[P_1, P_2]}(\shareB{\vc}^{\mathsf{add}})$.
  \item $\shareR{\vx}^\mathsf{add} \asn \vc + \shareR{\vb}^{\mathsf{add}} - 2 \cdot \vc \cdot \shareR{\vb}^{\mathsf{add}}$.
  \item Return $\textsf{AdditiveToReplicated}_{[P_1,P_2,P_3]}(\shareR{\vx}^\mathsf{add})$.
\end{enumerate}

\end{Boxfig}


\begin{Boxfig}{DaBit protocol generation}{fig:dabit-protocol}
  {Protocol $\Proto{DaBit[D, P_1, P_2]}(\shape)$}
  Roles: $P_1$, $P_2, P_3$ where $P_3$ acts as a dealer to distribute
  an additive share of a bit of shape $\shape$ in the boolean and arithmetic domain.
  \begin{enumerate}
  \item With Host(D):
  \begin{enumerate}
    \item $\seed \asn \derives(\dkey_D, \GenNonce)$.
    \item $\vb \asn \RingSample(\seed, \shape, \{0, 1\})$.
    \item $\seed(\textsf{bin}) \asn \derives(\dkey_D, \GenNonce), \seed(\textsf{ring}) \asn \derives(\dkey_D, \GenNonce)$.
    \item Send $\seed(\textsf{bin})$ and $\seed(\textsf{ring})$ to $P_1$.

  \end{enumerate}
  \item  With Host($P_1$):
 \begin{enumerate}
     \item Receive $\seed(\textsf{bin})$ and $\seed(\textsf{ring})$ from Dealer.
     \item Locally compute $\vb^1_{2^k} \asn \RingSample(\seed(\textsf{ring}), \shape, \Z_{2^k})$.
     \item Locally compute $\vb^1_{2} \asn \RingSample(\seed(\textsf{bin}), \shape, \{0, 1\})$.
 \end{enumerate}
\item With Host(D):
\begin{enumerate}
    \item Sample $\vb^1_{2^k} \asn \RingSample(\seed(\textsf{ring}), \shape, \Z_{2^k})$. Set $\vb^2_{2^k} = \vb - \vb^1_{2^k}$.
    \item Sample $\vb^1_{2} \asn \RingSample(\seed(\textsf{bin}), \shape, \{0, 1\})$. Set $\vb^2_{2} = \vb - \vb^1_{2}$. 
    \item Send $\vb^2_{2^k}$ and $\vb^2_2$ to $P_2$.
\end{enumerate}
\item With Host($P_2$):
\begin{enumerate}
    \item Receive $\vb^2_{2^k}$ and $\vb^2_2$ from Dealer.
\end{enumerate}
\item Let $\shareR{\vb} = \text{Additive}(\vb^1_{2^k}, \vb^2_{2^k})$ and $\shareB{\vb} = \text{Additive}(\vb^1_2, \vb^2_2)$. Output $(\shareR{\vb}, \shareB{\vb})$.
\end{enumerate}
\end{Boxfig}


% 
\begin{Boxfig}{$(2,2)$ to $(2,3)$ share conversion protocol for semi-honest
RSS.}{fig:two-to-three}{Protocol $\Proto{(2,2) \rightarrow (2,3)}$}
Input is $\share{x}$ which is additively shared amongst $P_1$ and $P_2$. Party $P_1$ holds $x_1$ and $P_2$ holds $x_2$ 
such that $x_1 + x_2 = x$. \\
$P_3$ does the following:
  \begin{enumerate}
    \item Sample seeds $s_1$ and $s_2$ and seeds $s_i$ to $P_i$.
    \item Compute $y_1 \sample \PRG(s_1)$ and $y_3 \sample \PRG(s_2)$.
    \item Set the output share as $(y_3, y_1)$.
 \end{enumerate}

$P_1$ and $P_2$ do the following:
\begin{enumerate}
   \item Parties extend the seeds received by $P_1$ computing $\hat{y}_1 \sample \PRG(s_1)$ and $P_2$ computing
   $\hat{y}_3 \sample \PRG(s_2)$.
   \item Compute $x_i - y_i$ and send it to $P_{3-i}$. 
   The received values are denoted by $\tilde{y}_i$.
   \item $P_1$ sets the output share as $(\hat{y}_1, x_1 - \hat{y}_1  + \tilde{y}_1)$
   \item $P_2$ sets the output share as $(x_2 - \hat{y}_2 + \tilde{y}_2, \hat{y}_3)$.
 \end{enumerate}
\end{Boxfig}


\begin{Boxfig}{Absolute value computation from a replicated ring share.}{fig:abs-protocol}
  {Protocol $\Proto{abs[P_1, P_2, P_3]}(\share{\vx})$}
  Roles: Replicated, $P_1$, $P_2, P_3$ where $\vx \in \Z_{2^k}^\shape$ and
  $\vx.\mathsf{placement}$ is on Replicated($P_1, P_2, P_3$). \newline
  With Replicated($P_1, P_2, P_3$):
  \begin{enumerate}
    \item Compute $\share{\vb} = \Proto{\mathsf{msb}}(\vx)$.
    \item Let $\share{\vs} \asn 1 - 2 \cdot \share{\vb}$.
    \item Output $\share{\vs} \cdot \share{\vb}$.
  \end{enumerate}

\end{Boxfig}

