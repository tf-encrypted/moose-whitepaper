\subsection{Comparisons}
Given $\share{x}$ where $x \in \Z_{2^k}$ our goal is to compute a ring sharing of $\share{x > 0}$.
This translates into how to compute a sharing of the most significant bit $\share{\mathsf{msb}(x)}$.
Since we are in the 3PC semi-honest model there are few approaches to this:
\begin{enumerate}

   \item SecureNN: there's some extra preprocessing which needs to be done
   although the round complexity seems to be a bit lower than ABY3 - need to
   check this though. The downside is that it uses arithmetic modulo weird
   fields. IMO we should stick with ABY3 because it either operates on rings
   or boolean type (so a bit more standard than SecureNN).
   \item ABY3:
   There's an implementation of this in MP-SPDZ as well, using
   the \textit{split} option when compiling a program. The main idea is for
   each party to locally bit-decompose their shares over $\Z_{2^k}$ and then
   reconstruct the secret modulo $\Z_{2^k}$ using shares from $\Z^k_2$ and a
   binary adder. Once we get a sharing of the MSB then in order to use it in
   other operations we need to convert the sharing back to one in $\Z_{2^k}$.
   In ABY3 this was done using a three-party OT protocol. In MP-SPDZ this was
   achieved using a daBit.
  \item The newest one is using edaBits \cite{C:EGKRS20}. The downside of
   this method is that we need to implement edaBit generation and a boolean
   circuit doing binary addition. The upside of this is that for 3PC
   semi-honest the edaBit generation doesn't seem to be that hard ie
   we can avoid the cut and choose since everyone will execute the protocol
   semi-honestly.
\end{enumerate}

\noindent In Moose we use ABY3 protocol with a Kogge-Stone binary adder with
some minor optimizations for tensor operations. We avoid the three party OT
done by ABY3 by using a slightly more efficient share conversion described
below.

\subsubsection{Improved ABY3 boolean to ring sharing protocol}
When using tensors we can perform the share conversion $\share{\cdot}_2
\mapsto \share{\cdot}_{\Z_{2^k}}$ much easier by making use that all parties
follow the protocol specifications.
Acting as a trusted third party, $P_3$ generates a
daBit $(\share{b}_2, \share{b}_{2^k})$ locally and then shares it to $P_1$ and $P_2$.
Since the shared tuple contains tensors which sum up to the bit $b$ then $P_3$ can
send to $P_2$ only two seeds $s_2, s_{2^k}$ which are going to be later expanded
locally by $P_2$. Now $P_3$ sends to $P_1$ the difference $b_2 \oplus
\mathsf{Expand}(s_2, b_2.\mathsf{shape})$ and $b_{2^k} -
\mathsf{Expand}(s_{2^k}, b_{2^k}.\mathsf{shape})$.

Now $P_1$ and $P_2$ run a two-party protocol to convert $\share{x}_2$ to
$\share{x}_{2^k}$ by computing $c \asn \mathsf{Open}(\share{x}_2 \oplus
\share{b}_2)$ and then locally XOR-ing in the arithmetic domain
$\share{x}_{2^k} = c + \share{b}_{2^k} - 2 \cdot c \cdot \share{b}_{2^k}$.
Finally they convert the $(2,2)$ sharing of $\share{x}_{2^k}$ to a $(2,3)$
sharing using $\Proto{(2,2) \rightarrow (2,3)}$ from
Figure~\ref{fig:two-to-three}.


\begin{Boxfig}{$(2,2)$ to $(2,3)$ share conversion protocol for semi-honest
RSS.}{fig:two-to-three}{Protocol $\Proto{(2,2) \rightarrow (2,3)}$}
Input is $\share{x}$ which is additively shared amongst $P_1$ and $P_2$. Party $P_1$ holds $x_1$ and $P_2$ holds $x_2$ 
such that $x_1 + x_2 = x$. \\
$P_3$ does the following:
  \begin{enumerate}
    \item Sample seeds $s_1$ and $s_2$ and seeds $s_i$ to $P_i$.
    \item Compute $y_1 \sample \PRG(s_1)$ and $y_3 \sample \PRG(s_2)$.
    \item Set the output share as $(y_3, y_1)$.
 \end{enumerate}

$P_1$ and $P_2$ do the following:
\begin{enumerate}
   \item Parties extend the seeds received by $P_1$ computing $\hat{y}_1 \sample \PRG(s_1)$ and $P_2$ computing
   $\hat{y}_3 \sample \PRG(s_2)$.
   \item Compute $x_i - y_i$ and send it to $P_{3-i}$. 
   The received values are denoted by $\tilde{y}_i$.
   \item $P_1$ sets the output share as $(\hat{y}_1, x_1 - \hat{y}_1  + \tilde{y}_1)$
   \item $P_2$ sets the output share as $(x_2 - \hat{y}_2 + \tilde{y}_2, \hat{y}_3)$.
 \end{enumerate}
\end{Boxfig}

