\section{Replicated Placements}


\subsection{Security Model}
In this section we expect the reader to have some familiarity with some basic
knowledge of how multiparty computation works, linear algebra and finite
field arithmetic. We strongly recommend reading the first sections
\cite{evans2017pragmatic} and more important Chapter 7 about honest majority
MPC.

In the last couple of years there was a boom in MPC for honest majority,
especially in 3PC one. This can be tracked down from the work of Araki et al.
\cite{CCS:AFLNO16} which at the moment provides the best performance for
semi-honest security in the preprocessing model. One year after this was
converted into a protocol with active security which achieves $1$ billion AND
triples per second \cite{SP:ABFLLN17} which had a cost of $7$ bits per AND
gate. Later it was shown by Keller et al. \cite{SCN:KRSW18} how to extend the
work of Araki et al. \cite{CCS:AFLNO16} to multiple parties assuming there is
honest majority using some well-known techniques by Maurer
\cite{SCN:Maurer02}. The caveat of Keller et al. \cite{SCN:KRSW18} is that
there can be an exponential blowup in terms of $n \choose t$ where $t<n/2$ is
the corruption threshold and $n$ the number of parties.


\subsection{Rationale}

For the linear regression use-case we believe that $128$ bit field size is
enough using a fixed point precision by $16$ bits while keeping the numbers
magnitude to $40$ bits (thus making the integral part of size $24$ bits). We
should keep in mind that although triple generation for large fields is
usually faster than rings there are benefits when working with ring sharings
- such as faster methods for secret truncation which in turn makes RELU's
faster.

Concretely, for the LR use-case we need to support a limited number of
operations on secret data but most important is a large secret matrix-matrix
multiplication. On cleartext data these multiplications are dealt using
floating point arithmetic, however when computing on encrypted data one must
unroll complex circuits into finite field operations. For this we stick to
fixed point computations described later in Section~\ref{subsec:fixed-point}.
Note that the underlying arithmetic can be either field ($\Fp$) or ring based
($\Z_{2^k}$). Finally, we might want to add some Gaussian noise over the
output to avoid one party getting the input matrix of the other party.

To get a better picture of the available protocols with their underlying
arithmetic we list the current state of the art protocols for obtaining
honest-majority MPC. For readability purposes we abbreviate honest majority
with HM:

\begin{enumerate}
  \item HM over large fields $\Fp$ \cite{CCS:LinNof17,C:CGHIKL18}. Although
  the two constructions listed above are specialized for the malicious
  setting (at most 1 party arbitrarily deviating from the protocol) these are
  built assuming $\Frand, \Fmult$ hybrid model where $\Fmult$ needs to be a
  multiplication protocol secure up to additive attacks. To get such an
  $\Fmult$ we can simply take
  a semi-honest protocol for HM for doing multiplications such as \cite{CCS:AFLNO16}.
  \item HM for binary circuits $\F_2$. Note that Araki et al. \cite{CCS:AFLNO16}
  works for binary circuits as well for the semi-honest case.
  To get malicious security Chida et al. \cite{C:CGHIKL18} present a protocol
  for small field multiplications using cut-and-choose techniques The cut and
  choose framework works by generating large batches of triples
  optimistically and then open a fraction to detect whether some cheating
  occurred.
  \item HM over small fields (eg. $\F_{2^8}$ which is suitable for AES). To
  get malicious security, depending on the extension field size, we can
  either repeat the computation $\sec/|\F|$ times or generate triples using cut and choose
  from \cite{C:CGHIKL18}.
  \item HM over rings $\mathbb{Z}_{2^k}$. Abspoel et al. \cite{EPRINT:ADEN19}
  are the first ones who introduce vector dot-products of secret shares at a cost of
  communicating a constant number of ring elements independently of the vector sizes
  for the ring case - however they avoid giving any hint on how to realise
  reactive computations, this is solved later in \cite{cryptoeprint:2020:1330}.
  \item HM by mixing circuits: \cite{CCS:MohRin18,cryptoeprint:2020:1330}.
  Escudero et al. \cite{C:EGKRS20} work in the more generic setting but they
  have
  improvements for the honest majority as well. They improve the mixed framework
  specifically for $\Z_{2^k}$ and $\F_2$ very recently in \cite{cryptoeprint:2020:1330}.

\end{enumerate}

Note that almost every protocol can be built in the malicious setting using
building blocks from the semi-honest protocols with the exception of binary
triples $c = a \cdot b$ where $a,b \in \F_2$. For the binary triple case we
need more complex techniques such as cut and choose. In our implementation we
stick with Araki et al. protocol \cite{CCS:AFLNO16} while borrowing some
improvements from \cite{cryptoeprint:2020:1330} such as probabilistic
truncation. One additional feature of Araki et al. \cite{CCS:AFLNO16} is that
it can be upgraded to malicious security by (roughly) just repeating all
procedures twice and use a "zero check" described in Dalskov, Escudero and
Keller \cite{cryptoeprint:2020:1330}.

\noindent{\textbf{Notation}}. We denote $\inc{i}$ as $i \mapsto (i+1) \bmod 3 + 1$.
Next, we denote $\share{x}$ as a value $x$ being secret shared
across the parties. In the cases below each party has a `replicated share` of
$x$, meaning $P_i$ has $(x_i, x_{ \inc{i+1} })$ where $\sum_{i=1}^3 = x
\in \Z_{2^k}$.

\subsection{Basic protocols}
\noindent{\textbf{Addition}}. To add two secret shares $\share{x}$ and $\share{y}$
each party locally adds their shares and computes $\share{x+y} =
(x_i + y_i, x_{\inc{i}} + y_{\inc{i}})$ for all $i \in [3]$.

\begin{Boxfig}{Local conversion to $(2,2)$ additive shares}{fig:replicated-add}
  {Protocol $\Proto{\mathsf{RepAdd}}(\vx: RepT_R, \vy: RepT_R)$}
  Roles: $P_1$, $P_2$, $P_3$ where $\vx.\mathsf{placement}$ and $\vy.\mathsf{placement}$ is Rep($P_1$, $P_2$, $P_3$) \newline
  Match $RepT((x_{11}, x_{12}), (x_{22}, x_{23}), (x_{33}, x_{31})) = x$ \newline
  Match $RepT((y_{11}, y_{12}), (y_{22}, y_{23}), (y_{33}, y_{31})) = y$ \newline
  With Host($P_i$):
  \begin{enumerate}
    \item $k \asn \mathsf{SampleKey}()$.
    \item $\seed \asn \derives(k, 0)$.
    \item $\vx^1 \asn \mathsf{SampleUniform}(\vx.\mathsf{shape}, \seed)$.
    \item $\vx^2 \asn \vx - \vx^1$.
    \item Send $\vx^i$ to $P_i$.
  \end{enumerate}
  With Host($P_1$):
  \begin{enumerate}
      \item Receive $\vx^1$ from Dealer.
  \end{enumerate}
  With Host($P_2$):
  \begin{enumerate}
      \item Receive $\vx^2$ from Dealer.
  \end{enumerate}
  Let $\vx = Additive(\vx^1, \vx^2)$.
\end{Boxfig}

\subsection{Multiplication}

Protocol $\mathsf{RepMul}$ in Figure~\ref{fig:replicated-mul} is used to compute the product of two replicated tensors $\share{\vz} = \share{\vx \cdot \vy}$. It follows the multiplication protocol of~\cite{CCS:AFLNO16} with some small computational optimizations in the underlying $\mathsf{ZeroShare}$ protocol in Figure~\ref{fig:replicated-sample-zero}, and inherit their correctness and security proofs.
%The main benefit of our $\mathsf{SampleZero}$ is that there are less $\RingSample$ calls per ring element than in Araki et al. which lends to a greater benefit when using AES with AVX instructions as described in Section~\ref{sec:aes}.

As for the $\mathsf{ZeroShare}$ protocol, note that the $\mathsf{DeriveSeed}$ operation is parameterized by an explicit $\mathsf{sync\_key}$ attribute allowing $P_i$ and $P_{i+1}$ to generate the same seeds non-interactively, i.e. $\seed^1_1 = \seed^3_1 \not = \seed^1_2 = \seed^2_2 \not = \seed^2_3 = \seed^3_3$. The operation is implemented as $\mathsf{PRF}(k, \sid \| \mathsf{sync\_key})$, which in turn is implemented using the keyed hash function from~\cite{libsodium} (BLAKE2b) with an output truncated to 128 bits; this ensures that all seeds are unique across sessions when $\sid$ and $\mathsf{sync\_key}$ are of fixed length $\ell$. Finally, the $\mathsf{SampleUniformSeeded}$ operation is implemented by running AES as a PRNG using the seed as the key and encrypting plaintexts $0, 1, ...$ in ECB mode.

\begin{Boxfig}{Replicated multiplication}{fig:replicated-mul}
  {Protocol $\mathsf{RepMul}_{[P_1, P_2, P_3]}\left( R; \prfkeys, \share{\vx}_R, \share{\vy}_R \right)$}
  \begin{enumerate}
  \item Let $\left( (\vx_1^1, \vx_2^1), (\vx_2^2, \vx_3^2), (\vx_3^3, \vx_1^3) \right) = \share{\vx}_R$.
  \item Let $\left( (\vy_1^1, \vy_2^1), (\vy_2^2, \vy_3^2), (\vy_3^3, \vy_1^3) \right) = \share{\vy}_R$.

  \item On $P_i$ for $i \in [3]$:
  \begin{enumerate}
    \item $\vv_i^i \asn \vx_i^i \cdot \vy_i^i + \vx_i^i \cdot \vy_{i+1}^i + \vx_{i+1}^i \cdot \vy_i^i$.
    \item $\shape_i \asn \mathsf{Shape}(\vv_i^i)$.
  \end{enumerate}

  \item $\valpha_1, \valpha_2, \valpha_3 \asn \mathsf{ZeroShare}_{[P_1, P_2, P_3]}\big( R; \prfkeys, \shape_1, \shape_2, \shape_3 \big)$

  \item On $P_i$ for $i \in [3]$:
  \begin{enumerate}
    \item $\vz_i^i \asn \vv_i^i + \valpha_i$.
    \item Send $\vz_i^i$ to $P_{i-1}$.
    \item Receive $\vz_{i+1}^i$ from $P_{i+1}$.
  \end{enumerate}

  \item Return $\share{\vz}_R = \left( (\vz_1^1, \vz_2^1), (\vz_2^2, \vz_3^2), (\vz_3^3, \vz_1^3) \right)$.
  \end{enumerate}
\end{Boxfig}


\begin{Boxfig}{Zero share protocol.}{fig:replicated-sample-zero}
  {Protocol $\mathsf{ZeroShare}_{[P_1, P_2, P_3]}(R; \prfkeys, \shape_1, \shape_2, \shape_3)$}

  \begin{enumerate}
  \item Let $\left( (k_1^1, k_2^1), (k_2^2, k_3^2), (k_3^3, k_1^3) \right) = \prfkeys$ be PRF keys generated during setup.

  \item Let $\mathsf{sync\_key}_1, \mathsf{sync\_key}_2, \mathsf{sync\_key}_3$ be distinct values.

  \item On $P_i$ for $i \in [3]$:
  \begin{enumerate}
    \item ${\seed}^i_i \asn \mathsf{DeriveSeed}(\mathsf{sync\_key}_i; k^i_i)$.
    \item ${\seed}^i_{i+1} \asn \mathsf{DeriveSeed}(\mathsf{sync\_key}_{i + 1}; k^i_{i+1})$.
    \item $\valpha_i \asn \mathsf{SampleUniformSeeded}(R; \seed^i_i, \shape_i) - \mathsf{SampleUniformSeeded}(R; \seed^i_{i+1}, \shape_i)$.
  \end{enumerate}

  \item Return $\valpha = \left( \valpha_1, \valpha_2, \valpha_3 \right)$.
  \end{enumerate}
\end{Boxfig}

\subsection{Sharing data}

In Figure~\ref{fig:replicated-share} we show how a party $D$ can translate a
private tensor $\vx$ (known only to $D$) into a replicated tensor.
Here we distinguish two cases: a) when $D$ is amongst the parties on the replicated placement
i.e. $D \in \{P_1, P_2, P_3\}$ and b) when $D$ is an external party, not in the replicated set.

In the first case the parties use a similar mechanism as in the
$\mathsf{ZeroShare}$ to produce a set of replicated shared seeds. Wlog., consider $D=P_i$,
then $P_i$ sends to $P_{i-1}$ the shape of $\vx$ in order for $P_{i-1}$ to derive a correct
sized random tensor using the shared seed. Then the inputting party
$P_i$ masks the private tensor with a random value $\vx - \vx_i^i$ and sends it
to $P_{i+1}$.

Correctness can be seen from the fact that at the end of the protocol all
$\vx_1^j + \vx_2^j + \vx_3^{j+1} = \vx$ for all $j \in [1,3]$. For example, when
$j = 1$ we have the following: $$\vx_1^1 + \vx_2^1 + \vx_3^2 = \mathbf{0} +
\vx_2^1 + (\vx - \vx_2^2) = \vx $$ due to $\vx_2^1 = \vx_2^2$ since they were
sampled from identical seeds. Security follows from~\cite{CCS:ABFLNO16}
or~\cite{CCS:MohRin18}.


In the second case $D$ is an external party sharing their input. Two of the
parties $(P_1, P_2)$ produce the randomness used to mask $\vx$ in $\vx_3 \asn
\vx - \vx_1 - \vx_2$ such that any two parties can reconstruct the secret
$\share{\vx}$ afterwards. Correctness can be seen as in the first case since all
$\vx_1^j + \vx_2^j + \vx_3^{j+1} = \vx$ for all $j \in [1,3]$. For example, when
$j = 1$ we have: $$ \vx_1^1 + \vx_2^1 + \vx_3^2 = \vx_1 + \vx_2 + (\vx - \vx_1 -
\vx_2) = \vx.$$ The security argument differs slightly then the first case since
$D$ sends $\vx - \vx_1 - \vx_2$ to $P_2$ and $P_3$. Since $\vx_1$ and $\vx_2$ are joint
randomness produced by $P_1$ and $P_2$ then any two parties (no less) can reconstruct the
secret.


\begin{Boxfig}{Replicated sharing protocol.}{fig:replicated-share}
  {Protocol $\mathsf{ReplicatedShare}_{[P_1, P_2, P_3]}(R; \prfkeys, \vx)$}
  
  Let $P_i$ be the player holding $\vx$.
  
  \begin{enumerate}
  \item Let $\left( (k_1^1, k_2^1), (k_2^2, k_3^2), (k_2^3, k_1^3) \right) = \prfkeys$ be PRF keys generated during setup.
  
  \item On $P_i$:
  \begin{enumerate}
    \item $\shape^i \asn \mathsf{Shape}(\vx)$.
    \item ${\seed}^i \asn \mathsf{DeriveSeed}(0; k^i_i)$.
    \item $\vx_i^i \asn \RingSample(R; \shape^i, \seed^i)$.
    \item $\vx_{i+1}^i \asn \vx - \vx_i^i$.
    \item Send $\shape^i$ to $P_{i-1}$ and $P_{i+1}$.
    \item Send $\vx_{i+1}^i$ to $P_{i+1}$.
  \end{enumerate}
  
  \item On $P_{i-1}$:
  \begin{enumerate}
    \item Receive $\shape^{i-1}$ from $P_i$.
    \item ${\seed}^{i-1} \asn \mathsf{DeriveSeed}(0; k^{i-1}_i)$.
    \item $\vx_{i-1}^{i-1} \asn \mathsf{Zeros}(R; \shape^{i-1})$.
    \item $\vx_i^{i-1} \asn \RingSample(R; \shape^{i-1}, \seed^{i-1})$.
  \end{enumerate}
  
  \item On $P_{i+1}$:
  \begin{enumerate}
    \item Receive $\shape^{i+1}$ from $P_i$.
    \item Receive $\vx_{i+1}^i$ from $P_i$.
    \item $\vx_{i+1}^{i+1} \asn \vx_{i+1}^i$.
    \item $\vx_{i+2}^{i+1} \asn \mathsf{Zeros}(R; \shape^{i+1})$.
  \end{enumerate}
  
  \item Return $\share{\vx}_R = \left( (\vx_1^1, \vx_2^1), (\vx_2^2, \vx_3^2), (\vx_3^3, \vx_1^3) \right)$.
  \end{enumerate}
\end{Boxfig}






\subsection{Advanced protocols}
\subsection{Secret dot-products using MPC}

The goal is to compute $\sum_{k=1}^m \share{x_k} \cdot \share{y_k}$.

\subsubsection{3-party semi-honest (HM)}
There are different ways to achieve this, depending on the secret sharing
type. Throughout this document we will stick with replicated secret sharing
although this works for Shamir Secret sharing. Both replicated and Shamir
work for the field and ring case, however for Shamir secret sharing it's a
bit more complicated to do MPC over rings - it is doable but with some
overhead \cite{EPRINT:CraRamXin19, TCC:ACDEY19}. To see how this works for
replicated secret sharing each party $P_i$ holds a sharing of $x$ as $(x_i,
x_{(i+1) \bmod 3})$ and a sharing of $y$ as $(y_i, y_{ (i+1) \bmod 3})$.
Since players offsets start with $1$:

\begin{itemize}
    \item $P_1$ has $(x_1, x_2)$ and $(y_1, y_2)$.
    \item $P_2$ has $(x_2, x_3)$ and $(y_2, y_3)$.
    \item $P_3$ has $(x_3, x_1)$ and $(y_3, y_1)$.
\end{itemize}

\noindent
To compute a sharing $\share{x \cdot y}$ using the shares of $x$ and $y$ parties do the following:

\begin{itemize}
    \item $P_1$ sets $(x \cdot y)_{1} \asn x_1 \cdot y_1 + x_1 \cdot y_2 + x_2 \cdot
    y_1 + \alpha$.
    \item $P_2$ sets $(x \cdot y)_2 \asn x_2 \cdot y_2 + x_2 \cdot y_3 + x_3
    \cdot y_2 + \beta$.
    \item $P_3$ sets $(x \cdot y)_3 \asn x_3 \cdot y_3 + x_3 \cdot y_1 + x_1 \cdot y_3 +
    \gamma$.
\end{itemize}

\noindent In the next phase parties send around the shares
\begin{itemize}
    \item $P_1$ sends privately $(x \cdot y)_1$ to $P_2$.
    \item $P_2$ sends privately $(x \cdot y)_2$ to $P_3$.
    \item $P_3$ sends privately $(x \cdot y)_3$ to $P_1$.
\end{itemize}

\noindent In the final step parties set their shares of $x \cdot y$ in the following way:
\begin{itemize}
    \item $P_1$ sets $\share{x \cdot y}$ as $(x\cdot y)_3, (x \cdot y)_1$.
    \item $P_2$ sets $\share{x \cdot y}$ as $(x\cdot y)_1, (x \cdot y)_2$.
    \item $P_3$ sets $\share{x \cdot y}$ as $(x\cdot y)_2, (x \cdot y)_3$.
\end{itemize}

Although this was written with $x$ and $y$ being a single ring element,
i.e. $x,y \in \Z_{2^k}$ the same protocols works when $x, y \in \Z^m_{2^k}$
and the ring multiplication operator `$\cdot$` is replaced with the dot
product. In this manner $x \cdot y = \sum_{i=1}^m x_i y_i \in \Z_{2^k}$ while
$x, y \in \Z^m_{2^k}$. This can be extended for arbitrary length tensors as long as
$\alpha, \beta, \gamma$ have the correct dimensions to mask randomize $x \cdot y$.

\subsubsection{3-party malicious (HM)}
Note that the first progress
for this problem was done for the field case by Chida et al. \cite{C:CGHIKL18}.
For the ring case see \cite{EPRINT:ADEN19,cryptoeprint:2020:1330}.

\subsubsection{Dishonest majority}

Here parties fetch a triple for each $\share{x_k} \cdot \share{y_k}$ multiplication
ending with a cost of $m$ preprocessed triples. This is valid for secret sharing (SPDZ-type
\cite{EC:KelPasRot18,C:DPSZ12,C:CDESX18})
but also for garbled circuits (\cite{AC:HazSchSor17,CCS:WanRanKat17b}) type protocols.



\section{Fixed point arithmetic}
\label{app:fixed-point}

\subsection{Encoding} To encode a number $x \in \mathbb{R}$ into its fixed
point representation define $\mathsf{Encode}(x, f) = \floor{x \cdot 2^f}$.
The rounding here is arbitrary, one can also use \verb|ceil| function as long
as the same rounding method is used in the decoding.

\subsection{Decoding} To decode a number from its fixed point representation
$x \in \verb|fixed|(k, f)$ output $\mathsf{Decode}(x, f) = x \cdot 2^{-f} \in \mathbb{R}$.

\subsection{Addition} To add two secret shared fixed point numbers
$\share{x}, \share{y}$ where $x,y \in \verb|fixed(k, f)|$ output
$\share{x}+\share{y}$. Note that $\mathsf{Decode}(x+y)$ will be successful
iff $(x+y)\cdot 2^f$ does not overflow $\Z_{2^k}$.

\subsection{Multiplication}

To multiply two fixed point numbers $x, y \in \verb|fixed|(k, f)$ we can
compute $z = x \cdot y \in \verb|fixed|(2 \cdot k, 2\cdot f)$. On the ring
level the operation works as $\bar{z} = x \cdot y \cdot 2 ^ {2f}$, hence the
correct decoding function becomes $\Decode(z, 2f) = x \cdot y$. Although
multiplications are pretty fast this way we can see that the numbers grow by
$2^f$ after each multiplication. To circumvent this we introduce the
\verb|Trunc| operation which takes $\bar{z} = x \cdot y \cdot 2^{2f}$ and outputs $\bar{z} / 2^{f}$
which makes $\verb|trunc|(\bar{z}, f) \in \verb|fixed|(2 \cdot k, f)$.
In the main body the \verb|trunc| operation is replaced by the probabilistic truncation $\mathsf{TruncPR}$ function.
Next we describe how these operations are done using secret shared inputs.

\noindent{\textbf{Fixed point multiplication between two secrets}.} Note that
multiplication between two fixed point numbers $\share{x}, \share{y}$ where
$x,y \in \verb|fixed(k, f)|$ gets us $z = x \cdot y \cdot 2^{2f}$. One can
now simply output $\Decode(z, 2f)$ if this is the last multiplication.
To avoid $z$ overflow in $\Z_{2^k}$ and allow
subsequent computations on $z$ call $z \asn \mathsf{trunc}(2 \cdot k, f)(z)$.

\noindent{\textbf{Multiplication by (clear) integer scalars}}.
To multiply a public scalar
$c$ with $\share{x}$ where $x \in \verb|fixed(k, f)|$ we output
$c \cdot \share{x}$. In this way when we decode the product
this becomes (over the ring)
$\Decode(c \cdot x \cdot 2^f) = c \cdot x$. Again, this operation is going to
be successful iff. the product $c \cdot x$ fits in $\Z_{2^k}$.

\noindent{\textbf{Multiplication by (clear) floating scalars}}. To multiply a
public floating point scalar $c$ with $\share{x}$ we compute $\hat{c} \asn
\Encode(c, f)$ and then set the output $\share{c \cdot x} =
\mathsf{trunc}(\hat{c} \cdot \share{x}, f)$. Note that $\hat{c} \cdot \share{x}$
can be done with local computations.

% \subsubsection{Division}
% TBD, SecureNN has some leakage, check MP-SPDZ - there was no description in
% ABY3 on fixed point division.

\subsection{Secret fixed point division by a public value}

Suppose here that $x = \bar{x} \cdot 2^f$ is a fixed point encoding of $\bar{x}$. 
Given two inputs $\share{x}$ and a public value $c$ which is known to all parties the goal is to output $\share{x /c}$.
This can be done naively by first encoding the public constant into its fixed point representation and compute the division between two fixed point numbers accordingly \cite{FC:CatSax10,SCN:CatDeH10}.

A slightly more efficient way is for all parties to compute an encoding of $1/c$
and then do a local multiplication i.e. $\share{t} \asn \mathsf{ceil}(1/c) \cdot 2^f \cdot \share{x}$.
Since $t$ evaluates to $1/c \cdot 2^f \cdot \bar{x} \cdot 2^f$ then truncating \verb|trunc|($\share{t}$) computes a fixed point encoding of $\share{x/c}$ which is want we wanted in the first place.



\subsection{Comparisons}
Given $\share{x}$ where $x \in \Z_{2^k}$ our goal is to compute a ring sharing of $\share{x > 0}$.
This translates into how to compute a sharing of the most significant bit $\share{\mathsf{msb}(x)}$.
Since we are in the 3PC semi-honest model there are few approaches to this:
\begin{enumerate}

   \item SecureNN: there's some extra preprocessing which needs to be done
   although the round complexity seems to be a bit lower than ABY3 - need to
   check this though. The downside is that it uses arithmetic modulo weird
   fields. IMO we should stick with ABY3 because it either operates on rings
   or boolean type (so a bit more standard than SecureNN).
   \item ABY3:
   There's an implementation of this in MP-SPDZ as well, using
   the \textit{split} option when compiling a program. The main idea is for
   each party to locally bit-decompose their shares over $\Z_{2^k}$ and then
   reconstruct the secret modulo $\Z_{2^k}$ using shares from $\Z^k_2$ and a
   binary adder. Once we get a sharing of the MSB then in order to use it in
   other operations we need to convert the sharing back to one in $\Z_{2^k}$.
   In ABY3 this was done using a three-party OT protocol. In MP-SPDZ this was
   achieved using a daBit.
  \item The newest one is using edaBits \cite{C:EGKRS20}. The downside of
   this method is that we need to implement edaBit generation and a boolean
   circuit doing binary addition. The upside of this is that for 3PC
   semi-honest the edaBit generation doesn't seem to be that hard ie
   we can avoid the cut and choose since everyone will execute the protocol
   semi-honestly.
\end{enumerate}

\noindent In Moose we use ABY3 protocol with a Kogge-Stone binary adder with
some minor optimizations for tensor operations. We avoid the three party OT
done by ABY3 by using a slightly more efficient share conversion described
below.

\subsubsection{Improved ABY3 boolean to ring sharing protocol}
When using tensors we can perform the share conversion $\share{\cdot}_2
\mapsto \share{\cdot}_{\Z_{2^k}}$ much easier by making use that all parties
follow the protocol specifications.
Acting as a trusted third party, $P_3$ generates a
daBit $(\share{b}_2, \share{b}_{2^k})$ locally and then shares it to $P_1$ and $P_2$.
Since the shared tuple contains tensors which sum up to the bit $b$ then $P_3$ can
send to $P_2$ only two seeds $s_2, s_{2^k}$ which are going to be later expanded
locally by $P_2$. Now $P_3$ sends to $P_1$ the difference $b_2 \oplus
\mathsf{Expand}(s_2, b_2.\mathsf{shape})$ and $b_{2^k} -
\mathsf{Expand}(s_{2^k}, b_{2^k}.\mathsf{shape})$.

Now $P_1$ and $P_2$ run a two-party protocol to convert $\share{x}_2$ to
$\share{x}_{2^k}$ by computing $c \asn \mathsf{Open}(\share{x}_2 \oplus
\share{b}_2)$ and then locally XOR-ing in the arithmetic domain
$\share{x}_{2^k} = c + \share{b}_{2^k} - 2 \cdot c \cdot \share{b}_{2^k}$.
Finally they convert the $(2,2)$ sharing of $\share{x}_{2^k}$ to a $(2,3)$
sharing using $\Proto{(2,2) \rightarrow (2,3)}$ from
Figure~\ref{fig:two-to-three}.


\begin{Boxfig}{$(2,2)$ to $(2,3)$ share conversion protocol for semi-honest
RSS.}{fig:two-to-three}{Protocol $\Proto{(2,2) \rightarrow (2,3)}$}
Input is $\share{x}$ which is additively shared amongst $P_1$ and $P_2$. Party $P_1$ holds $x_1$ and $P_2$ holds $x_2$ 
such that $x_1 + x_2 = x$. \\
$P_3$ does the following:
  \begin{enumerate}
    \item Sample seeds $s_1$ and $s_2$ and seeds $s_i$ to $P_i$.
    \item Compute $y_1 \sample \PRG(s_1)$ and $y_3 \sample \PRG(s_2)$.
    \item Set the output share as $(y_3, y_1)$.
 \end{enumerate}

$P_1$ and $P_2$ do the following:
\begin{enumerate}
   \item Parties extend the seeds received by $P_1$ computing $\hat{y}_1 \sample \PRG(s_1)$ and $P_2$ computing
   $\hat{y}_3 \sample \PRG(s_2)$.
   \item Compute $x_i - y_i$ and send it to $P_{3-i}$. 
   The received values are denoted by $\tilde{y}_i$.
   \item $P_1$ sets the output share as $(\hat{y}_1, x_1 - \hat{y}_1  + \tilde{y}_1)$
   \item $P_2$ sets the output share as $(x_2 - \hat{y}_2 + \tilde{y}_2, \hat{y}_3)$.
 \end{enumerate}
\end{Boxfig}



