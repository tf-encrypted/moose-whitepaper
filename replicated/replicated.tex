\section{Replicated Protocols}

In this section we describe our protocols for performing encrypted computations using the three-party replicated secret sharing scheme. Much of this work follows the lines of~\cite{CCS:AFLNO16} with some optimizations derived from the fact that we focus on tensorized computations. All protocols presented here operate in the honest-but-curious security model, meaning players are assumed to follow the protocols but may try to learn additional information from the messages they receive.


One reason for choosing~\cite{CCS:AFLNO16} as our foundation is that it is currently the fastest protocol for achieving semi-honest three-party multiplications in the preprocessing model using ring arithmetic.
%An year after it was shown by \cite{SP:ABFLLN17} that the boolean version of~\cite{CCS:AFLNO16} can be converted into a protocol with active security that achieves $1$ billion AND triples per second with a cost of $7$ bits per AND gate.
%
Moreover, its large ring variant (when performing arithmetic over $\Z_{2^k}$ for $k \geq 64$) can be upgraded to malicious security by (roughly) just repeating all procedures twice and use a "zero check" as described in~\cite{cryptoeprint:2020:1330}.

Finally, we also chose this line of work due to its efficiency at computing dot products by communicating a number of ring elements independently of the input tensor sizes, and other optimizations made possible by being in the three-party setting.
%Note that the work of efficient dot products in MPC was inspired mainly from~\cite{CCS:LinNof17,C:CGHIKL18,EPRINT:ADEN19,cryptoeprint:2020:1330}.

% Later it was shown by Keller et al. \cite{SCN:KRSW18} how to extend the
% work of Araki et al. \cite{CCS:AFLNO16} to multiple parties assuming there is
% honest majority using some well-known techniques by Maurer
% \cite{SCN:Maurer02}. The caveat of Keller et al. \cite{SCN:KRSW18} is that
% there can be an exponential blowup in terms of $n \choose t$ where $t<n/2$ is
% the corruption threshold and $n$ the number of parties.
% To get a better picture of the available protocols with their underlying
% arithmetic we list the current state of the art protocols for obtaining
% honest-majority MPC. For readability purposes we abbreviate honest majority
% with HM:

% \begin{enumerate}
%   \item HM over large fields $\Fp$ \cite{CCS:LinNof17,C:CGHIKL18}. Although
%   the two constructions listed above are specialized for the malicious
%   setting (at most 1 party arbitrarily deviating from the protocol) these are
%   built assuming $\Frand, \Fmult$ hybrid model where $\Fmult$ needs to be a
%   multiplication protocol secure up to additive attacks. To get such an
%   $\Fmult$ we can simply take
%   a semi-honest protocol for HM for doing multiplications such as \cite{CCS:AFLNO16}.
%   \item HM for binary circuits $\F_2$. Note that Araki et al. \cite{CCS:AFLNO16}
%   works for binary circuits as well for the semi-honest case.
%   To get malicious security Chida et al. \cite{C:CGHIKL18} present a protocol
%   for small field multiplications using cut-and-choose techniques The cut and
%   choose framework works by generating large batches of triples
%   optimistically and then open a fraction to detect whether some cheating
%   occurred.
%   \item HM over small fields (eg. $\F_{2^8}$ which is suitable for AES). To
%   get malicious security, depending on the extension field size, we can
%   either repeat the computation $\sec/|\F|$ times or generate triples using cut and choose
%   from \cite{C:CGHIKL18}.
%   \item HM over rings $\mathbb{Z}_{2^k}$. Abspoel et al. \cite{EPRINT:ADEN19}
%   are the first ones who introduce vector dot-products of secret shares at a cost of
%   communicating a constant number of ring elements independently of the vector sizes
%   for the ring case - however they avoid giving any hint on how to realise
%   reactive computations, this is solved later in \cite{cryptoeprint:2020:1330}.
%   \item HM by mixing circuits: \cite{CCS:MohRin18,cryptoeprint:2020:1330}.
%   Escudero et al. \cite{C:EGKRS20} work in the more generic setting but they
%   have
%   improvements for the honest majority as well. They improve the mixed framework
%   specifically for $\Z_{2^k}$ and $\F_2$ very recently in \cite{cryptoeprint:2020:1330}.

% \end{enumerate}

% Note that almost every protocol can be built in the malicious setting using
% building blocks from the semi-honest protocols with the exception of binary
% triples $c = a \cdot b$ where $a,b \in \F_2$. For the binary triple case we
% need more complex techniques such as cut and choose. In our implementation we
% stick with Araki et al. protocol \cite{CCS:AFLNO16} while borrowing some
% improvements from \cite{cryptoeprint:2020:1330} such as probabilistic
% truncation.

% We
% should keep in mind that although triple generation for large fields is
% usually faster than rings there are benefits when working with ring sharings
% - such as faster methods for secret truncation which in turn makes RELU's
% faster.


\subsection{Notation}
The protocols perform computations in rings $\ring$ of form $\mathbb{Z}_{2^{k}}$; concretely, $\mathbb{Z}_{2^{64}}$ and $\mathbb{Z}_{2^{128}}$. As outlined below, this allows us to emulate fixedpoint arithmetic as needed for the linear regression use case. We currently only present protocols for the operations needed to support linear regressions and associated metrics (see e.g. Section~\ref{??}).

Throughout this section we use 
$$
\share{\vx}_\ring = \left( (\vx_1^1, \vx_2^1), (\vx_2^2, \vx_3^2), (\vx_3^3, \vx_1^3) \right)
$$
to denote a replicated value consisting of shares $\vx_1 = \vx_1^1 = \vx_1^3$, $\vx_2 = \vx_2^2 = \vx_2^1$, and $\vx_3 = \vx_3^3 = \vx_3^2$ over ring $\ring$ such that $\vx = \sum_{i=1}^3 \vx_i \in \Z_{2^k}$. It will typically be the case that $P_i$ holds $\vx_i^i$ and $\vx_{i+1}^i$. To ease the notation we occasionally leave the $\ring$ implicit and simply write $\share{\vx}$, and we let indices wrapping around such that for instance $P_{3+1} = P_1$. We also often leave out $\ring$ when performing operations on shares, although these are always done using the corresponding ring arithmetic.
%Note that $\share{x}$ hence also states that the $x_{i+1}$ held by $P_i$ matches the $x_{i+1}$ held by $P_{i+1}$. We furthermore use this to express statements about results of computing on secret shared values such as for instance $\share{\vz} = \share{\vx + \vy} = \share{\vx} + \share{\vy}$.

\subsection{Setup}

Some of the replicated protocols rely on PRF keys produced during an initial
setup phase as described by the $\mathsf{Setup}$ protocol in
Figure~\ref{fig:replicated-setup}. Note that keys are distributed the same way
as replicated shares so that $P_i$ ends up knowing both $k_i$ and $k_{i+1}$.
Generation of individual keys by $\mathsf{GenPrfKey}$ is implemented by sampling
128 random bits using~\cite{libsodium}.

\msubsubsection {$\mathsf{Setup}_{[P_1, P_2, P_3]}$}
\label{fig:replicated-setup}
Replicated setup protocol.

  \begin{enumerate}
  \item On $P_i$ for $i \in [3]$:
  \begin{enumerate}
    \item $k^{i}_{i} \asn \mathsf{GenPrfKey}()$.
    \item Send $k^{i}_{i}$ to $P_{i-1}$.
    \item Receive $k^{i}_{i+1}$ from $P_{i+1}$.
  \end{enumerate}

  \item Return $\prfkeys = \left( (k^1_1, k^1_2), (k^2_2, k^2_3), (k^3_3, k^3_1) \right)$.
  \end{enumerate}


\subsection{Addition and Subtraction}

To compute the addition of two replicated tensors $\share{\vz} = \share{\vx + \vy}$
the parties simply add their local shares of $\share{\vx}$ and $\share{\vy}$
as described in protocol~$\mathsf{RepAdd}$ in Figure~\ref{fig:replicated-add}. Subtraction as shown in Figure~\ref{fig:replicated-sub} is almost identical, with the parties simply subtracting their shares locally instead of adding. Correctness and security follow from~\cite{CCS:ABFLNO16}.


\msubsubsection
{$\mathsf{RepAdd}_{[P_1, P_2, P_3]}\left(\share{\vx}, \share{\vy} \right)$}
\label{fig:replicated-add}
  Replicated addition protocol.

  \begin{enumerate}
  \item Let $\left( (\vx_1^1, \vx_2^1), (\vx_2^2, \vx_3^2), (\vx_3^3, \vx_1^3) \right) = \share{\vx}$.

  \item Let $\left( (\vy_1^1, \vy_2^1), (\vy_2^2, \vy_3^2), (\vy_3^3, \vy_1^3) \right) = \share{\vy}$.

  \item On $P_i$ for $i \in [3]$:
  \begin{enumerate}
    \item $\vz_i^i \asn \vx_i^i + \vy_i^i$
    \item $\vz_{i+1}^i \asn \vx_{i+1}^i + \vy_{i+1}^i$
  \end{enumerate}

  \item Return $\share{\vz} = \left( (\vz_1^1, \vz_2^1), (\vz_2^2, \vz_3^2), (\vz_3^3, \vz_1^3) \right)$.
  \end{enumerate}

\subsubsection
{$\mathsf{RepSub}_{[P_1, P_2, P_3]}\left(\share{\vx}, \share{\vy} \right)$}
\label{fig:replicated-sub}
  Replicated subtraction protocol.

  \begin{enumerate}
  \item Let $\left( (\vx_1^1, \vx_2^1), (\vx_2^2, \vx_3^2), (\vx_3^3, \vx_1^3) \right) = \share{\vx}$.

  \item Let $\left( (\vy_1^1, \vy_2^1), (\vy_2^2, \vy_3^2), (\vy_3^3, \vy_1^3) \right) = \share{\vy}$.

  \item On $P_i$ for $i \in [3]$:
  \begin{enumerate}
    \item $\vz_i^i \asn \vx_i^i - \vy_i^i$
    \item $\vz_{i+1}^i \asn \vx_{i+1}^i - \vy_{i+1}^i$
  \end{enumerate}

  \item Return $\share{\vz} = \left( (\vz_1^1, \vz_2^1), (\vz_2^2, \vz_3^2), (\vz_3^3, \vz_1^3) \right)$.
  \end{enumerate}

\subsection{Multiplication}

Protocol $\mathsf{RepMul}$ in Figure~\ref{fig:replicated-mul} is used to compute the product of two replicated tensors $\share{\vz} = \share{\vx \cdot \vy}$. It follows the multiplication protocol of~\cite{CCS:AFLNO16} with some small computational optimizations in the underlying $\mathsf{ZeroShare}$ protocol in Figure~\ref{fig:replicated-sample-zero}, and inherit their correctness and security proofs.
%The main benefit of our $\mathsf{SampleZero}$ is that there are less $\RingSample$ calls per ring element than in Araki et al. which lends to a greater benefit when using AES with AVX instructions as described in Section~\ref{sec:aes}.

As for the $\mathsf{ZeroShare}$ protocol, note that the $\mathsf{DeriveSeed}$ operation is parameterized by an explicit $\mathsf{sync\_key}$ attribute allowing $P_i$ and $P_{i+1}$ to generate the same seeds non-interactively, i.e. $\seed^1_1 = \seed^3_1 \not = \seed^1_2 = \seed^2_2 \not = \seed^2_3 = \seed^3_3$. The operation is implemented as $\mathsf{PRF}(k, \sid \| \mathsf{sync\_key})$, which in turn is implemented using the keyed hash function from~\cite{libsodium} (BLAKE2b) with an output truncated to 128 bits; this ensures that all seeds are unique across sessions when $\sid$ and $\mathsf{sync\_key}$ are of fixed length $\ell$. Finally, the $\mathsf{SampleUniformSeeded}$ operation is implemented by running AES as a PRNG using the seed as the key and encrypting plaintexts $0, 1, ...$ in ECB mode.

\begin{Boxfig}{Replicated multiplication}{fig:replicated-mul}
  {Protocol $\mathsf{RepMul}_{[P_1, P_2, P_3]}\left( R; \prfkeys, \share{\vx}_R, \share{\vy}_R \right)$}
  \begin{enumerate}
  \item Let $\left( (\vx_1^1, \vx_2^1), (\vx_2^2, \vx_3^2), (\vx_3^3, \vx_1^3) \right) = \share{\vx}_R$.
  \item Let $\left( (\vy_1^1, \vy_2^1), (\vy_2^2, \vy_3^2), (\vy_3^3, \vy_1^3) \right) = \share{\vy}_R$.

  \item On $P_i$ for $i \in [3]$:
  \begin{enumerate}
    \item $\vv_i^i \asn \vx_i^i \cdot \vy_i^i + \vx_i^i \cdot \vy_{i+1}^i + \vx_{i+1}^i \cdot \vy_i^i$.
    \item $\shape_i \asn \mathsf{Shape}(\vv_i^i)$.
  \end{enumerate}

  \item $\valpha_1, \valpha_2, \valpha_3 \asn \mathsf{ZeroShare}_{[P_1, P_2, P_3]}\big( R; \prfkeys, \shape_1, \shape_2, \shape_3 \big)$

  \item On $P_i$ for $i \in [3]$:
  \begin{enumerate}
    \item $\vz_i^i \asn \vv_i^i + \valpha_i$.
    \item Send $\vz_i^i$ to $P_{i-1}$.
    \item Receive $\vz_{i+1}^i$ from $P_{i+1}$.
  \end{enumerate}

  \item Return $\share{\vz}_R = \left( (\vz_1^1, \vz_2^1), (\vz_2^2, \vz_3^2), (\vz_3^3, \vz_1^3) \right)$.
  \end{enumerate}
\end{Boxfig}


\begin{Boxfig}{Zero share protocol.}{fig:replicated-sample-zero}
  {Protocol $\mathsf{ZeroShare}_{[P_1, P_2, P_3]}(R; \prfkeys, \shape_1, \shape_2, \shape_3)$}

  \begin{enumerate}
  \item Let $\left( (k_1^1, k_2^1), (k_2^2, k_3^2), (k_3^3, k_1^3) \right) = \prfkeys$ be PRF keys generated during setup.

  \item Let $\mathsf{sync\_key}_1, \mathsf{sync\_key}_2, \mathsf{sync\_key}_3$ be distinct values.

  \item On $P_i$ for $i \in [3]$:
  \begin{enumerate}
    \item ${\seed}^i_i \asn \mathsf{DeriveSeed}(\mathsf{sync\_key}_i; k^i_i)$.
    \item ${\seed}^i_{i+1} \asn \mathsf{DeriveSeed}(\mathsf{sync\_key}_{i + 1}; k^i_{i+1})$.
    \item $\valpha_i \asn \mathsf{SampleUniformSeeded}(R; \seed^i_i, \shape_i) - \mathsf{SampleUniformSeeded}(R; \seed^i_{i+1}, \shape_i)$.
  \end{enumerate}

  \item Return $\valpha = \left( \valpha_1, \valpha_2, \valpha_3 \right)$.
  \end{enumerate}
\end{Boxfig}

\subsection{Sharing data}

In Figure~\ref{fig:replicated-share} we show how a party $D$ can translate a
private tensor $\vx$ (known only to $D$) into a replicated tensor.
Here we distinguish two cases: a) when $D$ is amongst the parties on the replicated placement
i.e. $D \in \{P_1, P_2, P_3\}$ and b) when $D$ is an external party, not in the replicated set.

In the first case the parties use a similar mechanism as in the
$\mathsf{ZeroShare}$ to produce a set of replicated shared seeds. Wlog., consider $D=P_i$,
then $P_i$ sends to $P_{i-1}$ the shape of $\vx$ in order for $P_{i-1}$ to derive a correct
sized random tensor using the shared seed. Then the inputting party
$P_i$ masks the private tensor with a random value $\vx - \vx_i^i$ and sends it
to $P_{i+1}$.

Correctness can be seen from the fact that at the end of the protocol all
$\vx_1^j + \vx_2^j + \vx_3^{j+1} = \vx$ for all $j \in [1,3]$. For example, when
$j = 1$ we have the following: $$\vx_1^1 + \vx_2^1 + \vx_3^2 = \mathbf{0} +
\vx_2^1 + (\vx - \vx_2^2) = \vx $$ due to $\vx_2^1 = \vx_2^2$ since they were
sampled from identical seeds. Security follows from~\cite{CCS:ABFLNO16}
or~\cite{CCS:MohRin18}.


In the second case $D$ is an external party sharing their input. Two of the
parties $(P_1, P_2)$ produce the randomness used to mask $\vx$ in $\vx_3 \asn
\vx - \vx_1 - \vx_2$ such that any two parties can reconstruct the secret
$\share{\vx}$ afterwards. Correctness can be seen as in the first case since all
$\vx_1^j + \vx_2^j + \vx_3^{j+1} = \vx$ for all $j \in [1,3]$. For example, when
$j = 1$ we have: $$ \vx_1^1 + \vx_2^1 + \vx_3^2 = \vx_1 + \vx_2 + (\vx - \vx_1 -
\vx_2) = \vx.$$ The security argument differs slightly then the first case since
$D$ sends $\vx - \vx_1 - \vx_2$ to $P_2$ and $P_3$. Since $\vx_1$ and $\vx_2$ are joint
randomness produced by $P_1$ and $P_2$ then any two parties (no less) can reconstruct the
secret.


\begin{Boxfig}{Replicated sharing protocol.}{fig:replicated-share}
  {Protocol $\mathsf{ReplicatedShare}_{[P_1, P_2, P_3]}(R; \prfkeys, \vx)$}
  
  Let $P_i$ be the player holding $\vx$.
  
  \begin{enumerate}
  \item Let $\left( (k_1^1, k_2^1), (k_2^2, k_3^2), (k_2^3, k_1^3) \right) = \prfkeys$ be PRF keys generated during setup.
  
  \item On $P_i$:
  \begin{enumerate}
    \item $\shape^i \asn \mathsf{Shape}(\vx)$.
    \item ${\seed}^i \asn \mathsf{DeriveSeed}(0; k^i_i)$.
    \item $\vx_i^i \asn \RingSample(R; \shape^i, \seed^i)$.
    \item $\vx_{i+1}^i \asn \vx - \vx_i^i$.
    \item Send $\shape^i$ to $P_{i-1}$ and $P_{i+1}$.
    \item Send $\vx_{i+1}^i$ to $P_{i+1}$.
  \end{enumerate}
  
  \item On $P_{i-1}$:
  \begin{enumerate}
    \item Receive $\shape^{i-1}$ from $P_i$.
    \item ${\seed}^{i-1} \asn \mathsf{DeriveSeed}(0; k^{i-1}_i)$.
    \item $\vx_{i-1}^{i-1} \asn \mathsf{Zeros}(R; \shape^{i-1})$.
    \item $\vx_i^{i-1} \asn \RingSample(R; \shape^{i-1}, \seed^{i-1})$.
  \end{enumerate}
  
  \item On $P_{i+1}$:
  \begin{enumerate}
    \item Receive $\shape^{i+1}$ from $P_i$.
    \item Receive $\vx_{i+1}^i$ from $P_i$.
    \item $\vx_{i+1}^{i+1} \asn \vx_{i+1}^i$.
    \item $\vx_{i+2}^{i+1} \asn \mathsf{Zeros}(R; \shape^{i+1})$.
  \end{enumerate}
  
  \item Return $\share{\vx}_R = \left( (\vx_1^1, \vx_2^1), (\vx_2^2, \vx_3^2), (\vx_3^3, \vx_1^3) \right)$.
  \end{enumerate}
\end{Boxfig}



\subsection{Revealing secrets}

We sometimes use interchangeably the terms \textit{open} and \textit{reveal} a secret to refer to taking a replicated tensors $\share{\vx}$ and making it $\vx$ known to a single party  (Figure~\ref{fig:replicated-open}) or to all parties (Figure~\ref{fig:additive-open}).
Note that Figure~\ref{fig:replicated-open} corresponds to a replicated opening whereas Figure~\ref{fig:additive-open} to an two out of two additive sharing scheme which is used in some more
advanced protocols below for truncation or share conversion. Correctness is guaranteed as long as
the secret was correctly shared. This is also secure as we only send the required
shares to the set of parties that want the secret revealed.

\begin{Boxfig}{Replicated opening protocol to a specific party $P_i$.}{fig:replicated-open}
  {$\Open[P_1, P_2, P_3](R; P_i, \share{\vx}_R)$}
  
  \begin{enumerate}
  \item Let $\left( (\vx_1^1, \vx_2^1), (\vx_2^2, \vx_3^2), (\vx_3^3, \vx_1^3) \right) = \share{\vx}_R$.
  \item On Host($P_{i-1}$):
  \begin{enumerate}
    \item Send $\vx_{i-1}^{i-1}$ to $P_{i-1}$.
  \end{enumerate}
  \item On Host($P_i$):
  \begin{enumerate}
   \item Receive $\vx_{i-1}^{i-1}$ from $P_{i-1}$.
   \item Store $\vx = \vx_{i-1}^{i-1} + \vx_i^i + \vx_i^{i+1}$.
  \end{enumerate}
  \item Return $\mathsf{Unit}$.
\end{enumerate}
\end{Boxfig}



\subsection{Secret dot-products using MPC}

The goal is to compute $\sum_{k=1}^m \share{x_k} \cdot \share{y_k}$.

\subsubsection{3-party semi-honest (HM)}
There are different ways to achieve this, depending on the secret sharing
type. Throughout this document we will stick with replicated secret sharing
although this works for Shamir Secret sharing. Both replicated and Shamir
work for the field and ring case, however for Shamir secret sharing it's a
bit more complicated to do MPC over rings - it is doable but with some
overhead \cite{EPRINT:CraRamXin19, TCC:ACDEY19}. To see how this works for
replicated secret sharing each party $P_i$ holds a sharing of $x$ as $(x_i,
x_{(i+1) \bmod 3})$ and a sharing of $y$ as $(y_i, y_{ (i+1) \bmod 3})$.
Since players offsets start with $1$:

\begin{itemize}
    \item $P_1$ has $(x_1, x_2)$ and $(y_1, y_2)$.
    \item $P_2$ has $(x_2, x_3)$ and $(y_2, y_3)$.
    \item $P_3$ has $(x_3, x_1)$ and $(y_3, y_1)$.
\end{itemize}

\noindent
To compute a sharing $\share{x \cdot y}$ using the shares of $x$ and $y$ parties do the following:

\begin{itemize}
    \item $P_1$ sets $(x \cdot y)_{1} \asn x_1 \cdot y_1 + x_1 \cdot y_2 + x_2 \cdot
    y_1 + \alpha$.
    \item $P_2$ sets $(x \cdot y)_2 \asn x_2 \cdot y_2 + x_2 \cdot y_3 + x_3
    \cdot y_2 + \beta$.
    \item $P_3$ sets $(x \cdot y)_3 \asn x_3 \cdot y_3 + x_3 \cdot y_1 + x_1 \cdot y_3 +
    \gamma$.
\end{itemize}

\noindent In the next phase parties send around the shares
\begin{itemize}
    \item $P_1$ sends privately $(x \cdot y)_1$ to $P_2$.
    \item $P_2$ sends privately $(x \cdot y)_2$ to $P_3$.
    \item $P_3$ sends privately $(x \cdot y)_3$ to $P_1$.
\end{itemize}

\noindent In the final step parties set their shares of $x \cdot y$ in the following way:
\begin{itemize}
    \item $P_1$ sets $\share{x \cdot y}$ as $(x\cdot y)_3, (x \cdot y)_1$.
    \item $P_2$ sets $\share{x \cdot y}$ as $(x\cdot y)_1, (x \cdot y)_2$.
    \item $P_3$ sets $\share{x \cdot y}$ as $(x\cdot y)_2, (x \cdot y)_3$.
\end{itemize}

Although this was written with $x$ and $y$ being a single ring element,
i.e. $x,y \in \Z_{2^k}$ the same protocols works when $x, y \in \Z^m_{2^k}$
and the ring multiplication operator `$\cdot$` is replaced with the dot
product. In this manner $x \cdot y = \sum_{i=1}^m x_i y_i \in \Z_{2^k}$ while
$x, y \in \Z^m_{2^k}$. This can be extended for arbitrary length tensors as long as
$\alpha, \beta, \gamma$ have the correct dimensions to mask randomize $x \cdot y$.

\subsubsection{3-party malicious (HM)}
Note that the first progress
for this problem was done for the field case by Chida et al. \cite{C:CGHIKL18}.
For the ring case see \cite{EPRINT:ADEN19,cryptoeprint:2020:1330}.

\subsubsection{Dishonest majority}

Here parties fetch a triple for each $\share{x_k} \cdot \share{y_k}$ multiplication
ending with a cost of $m$ preprocessed triples. This is valid for secret sharing (SPDZ-type
\cite{EC:KelPasRot18,C:DPSZ12,C:CDESX18})
but also for garbled circuits (\cite{AC:HazSchSor17,CCS:WanRanKat17b}) type protocols.


\subsection{Truncation}
\label{subsec:truncation}

Given a replicated tensor $\share{\va}$ the goal is to compute $\share{\va \bmod 2^m}$ where $m$ is a public constant. The ways of doing this can roughly be classified into two lines of thought: the more efficient but with a rounding error (probabilistic truncation) while the other slightly less efficient but with no rounding errors (deterministic
truncation).

Current state of the art techniques use dabits/edabits as described by Escudero et al. in \cite{C:EGKRS20} together with various share conversions. In~\cite{C:EGKRS20} there are several techniques to achieve both probabilistic and deterministic truncation for field and ring case. For our special case of semi-honest three-parties model, 
we implement the probabilistic truncation from Dalskov et al. \cite{PoPETS:DalEscKel20} 
which avoids additional preprocessing such as edabits and have slightly better performance.
For more details on edabits versus Dalskov et al. the reader can check Table 10 in \cite{C:EGKRS20}.
We describe Dalskov et al. probabilistic protocol $\mathsf{TruncPR}$ for replicated secret
sharing below in Fig~\ref{fig:truncpr-rep-ss}. 

At a high-level the idea is to convert the replicated sharing to a $(2,2)$ additive sharing
and then execute the truncation protocol between two parties Figure~\ref{fig:truncpr-generic}. 
The preprocessing for the 2PC computation is 
generated by $P_3$ which is later additively shared to $P_1$ and $P_2$ using Figure~\ref{fig:get-additive}. After the two-party truncation is done, the output share is then converted
back to a replicated sharing using Figure~\ref{fig:two-to-three}. Correctness and security follow from work of Dalskov et al \cite{PoPETS:DalEscKel20}.


\begin{Boxfig}{Ring truncation protocol for semi-honest
RSS.}{fig:truncpr-rep-ss}{$\mathsf{TruncPr}_{[P_1, P_2, P_3]} \left(\Z_{2^k}; \prfkeys, \share{\vx}, m \right)$}
\begin{enumerate}
  \item Let $\left( (k_1^1, k_2^1), (k_2^2, k_3^2), (k_3^3, k_1^3) \right) = \prfkeys$.
  
  \item Let $\left( (\vx_1^1, \vx_2^1), (\vx_2^2, \vx_3^2), (\vx_3^3, \vx_1^3) \right) = \share{\vx}$.
  
  \item On $P_3$:
  \begin{enumerate}
    \item $\shape \asn \mathsf{Shape}(\vx^3_3)$.
    \item $\mathsf{key} \asn \mathsf{GenPrfKey}()$.
    \item $\seed_\ell \asn \derives(\ell; k)$ for $\ell \in [n]$.
    \item $\mathbf{b}_\ell \asn \mathsf{SampleBits}(\shape, \seed_\ell)$ for $\ell \in [n]$.
    \item $\mathbf{r}_{\mathsf{msb}} \asn \vb_{k-1}$
    \item $\mathbf{r}_{\mathsf{bot}} \asn \sum_{\ell=0}^{k-1}2^\ell \cdot \vb_\ell$
    \item $\mathbf{r}_{\mathsf{top}} \asn \sum_{\ell=m}^{k-2} 2^{\ell-m} \cdot \vb_\ell$.
  \end{enumerate}

  \item $\share{\vr_{\mathsf{msb}}} \asn \mathsf{AdditiveShare}_{[D=P_3, P_1, P_2]}(\vr_{\mathsf{msb}})$

  \item $\share{\vr_{\mathsf{bot}}} \asn \mathsf{AdditiveShare}_{[D=P_3, P_1, P_2]}(\vr_{\mathsf{bot}})$
  
  \item $\share{\vr_{\mathsf{top}}} \asn \mathsf{AdditiveShare}_{[D=P_3, P_1, P_2]}(\vr_{\mathsf{top}})$
 
  \item $\share{\vx'} \asn \mathsf{ReplicatedToAdditive}_{[P_1, P_2]}(\share{\vx})$.

  \item $\share{\vy'} \asn \mathsf{2PCTruncPr}_{[P_1, P_2]}(\share{\vx'}, m)$.

  \item Return $\share{\vy} \asn \mathsf{AdditiveToReplicated}_{[P_1, P_2, P_3]}(\share{\vy'})$.

\end{enumerate}

%  \begin{enumerate}
%    \item Each $P_i$ for $i \in [1, 2]$ sets the output of $\mathsf{TruncPr}$
%    as $y'_i$ and $\hat{y}_i$ as the sharing received from $P_3$. Send $y'_i -
%    \hat{y}_i$ to $P_{3-i}$. The received values are denoted by $\tilde{y}_i$.
%    \item $P_1$ sets the output share as $(y_1, y'_1 - \hat{y}_1 + \tilde{y}_1)$
%    \item $P_2$ sets the output share as $(y'_2 - \hat{y}_2 + \widehat{y}_2, y_3)$.
%  \end{enumerate}
\end{Boxfig}

\begin{Boxfig}{Generic truncation protocol.}{fig:truncpr-generic}
  {$\mathsf{2PCTruncPr}_{[P_1, P_2]} \left( \share{\vx}, m \right)$}
  Assume $\vx \in \Z_{2^k}^{\shape}$ has $\mathsf{MSB}(\vx) = 0$. If $\vx$ is negative then
  we make it positive by adding $2^{k-1}$ to it and assume that $\mathsf{abs}(\vx) < 2^{k-1}$. \commentM{Dragos could you add a few more details here?}
  \begin{enumerate}
  
  \item $\share{\vc} \asn \share{\vx} + \share{\vr_{\mathsf{bot}}}$.
  
  \item $\vc^1, \vc^2 \asn \Open_{[P_1, P_2]}(\share{\vc})$.
  
  \item On $P_i$ for $i \in [2]$:
    \begin{enumerate}
    \item $\vc_t^{i} \asn (\vc^i / 2^m) \bmod 2^{k-m-1}$. This can be done using one $\mathsf{ring\_shl}$ followed by one $\mathsf{ring\_shr}$ operation.
    \end{enumerate}
    
  \item $\share{\vb} \asn \share{\vr_{\mathsf{msb}}} \oplus (\vc / 2^{k-1})$. Note here that $\oplus$ in arithmetic circuits boils down to $a + b - 2a\cdot b$. Here one of the operands is public so the result can be computed using only local operations.
    
  \item Return $\share{\vy} \asn \vc_t - \share{\vr_{\mathsf{top}}} + \share{\vb} \cdot 2^{k-m-1}$.
\end{enumerate}

\end{Boxfig}


\begin{Boxfig}{Additive sharing.}{fig:get-additive}
  {$\mathsf{AdditiveShare}_{[D, P_1, P_2]}(R; \vx)$}
  
  Let $D$ be the party holding $\vx$.
  
  \begin{enumerate}
    \item On $D$:
  \begin{enumerate}
    \item $\shape \asn \mathsf{Shape}(\vx)$.
    \item $k \asn \mathsf{GenPrfKey}()$.
    \item $\seed \asn \derives(0; k)$.
    \item $\vx^1 \asn \RingSample(R; \shape, \seed)$.
    \item $\vx^2 \asn \vx - \vx^1$.
    %\item Send $\shape$ and $\seed$ to $P_1$.
    \item Send $\vx^1$ to $P_1$.
    \item Send $\vx^2$ to $P_2$.
  \end{enumerate}
  \item On $P_1$:
  \begin{enumerate}
      \item Receive $\vx^1_1$ from Dealer.
      %\item $\vx^1 \asn \RingSample(R; \mathsf{Shape}(\vx), \seed)$.
  \end{enumerate}
  \item On $P_2$:
  \begin{enumerate}
      \item Receive $\vx^2_2$ from Dealer.
  \end{enumerate}
  \item Return $\share{\vx} = (\vx^1_1, \vx^2_2)$.
\end{enumerate}
\end{Boxfig}

\begin{Boxfig}{Conversion from replicated shares to additive shares.}{fig:replicated-to-additive}
  {$\mathsf{ReplicatedToAdditive}_{[P_1, P_2]}\left( \share{\vx} \right)$}
  
  \begin{enumerate}
  
  \item Let $\left( (\vx_1^1, \vx_2^1), (\vx_2^2, \vx_3^2), (\vx_3^3, \vx_1^3) \right) = \share{\vx}$.
  
  \item On $P_1$:
    \begin{enumerate}
    \item $\vx'_1 \asn \vx^1_1 + \vx^1_2$.
    \end{enumerate}
    
  \item On $P_2$:
    \begin{enumerate}
    \item $\vx'_2 \asn \vx^2_3$.
    \end{enumerate}
  
  \item Return $\share{\vx'} = (\vx'_1, \vx'_2)$.
\end{enumerate}
\end{Boxfig}



% \begin{Boxfig}{Opening protocol}{fig:open2-3}
%   {Protocol $\Proto{\mathsf{OpenTo}}^{(2,3)}(\share{\vx})$}
%   On $P_i$:
%   \begin{enumerate}
%     \item Send $\vx_1$ to $P_2$.
%   \end{enumerate}
%    On $P_2$:
%   \begin{enumerate}
%     \item Send $\vx_2$ to $P_1$.
%   \end{enumerate}
%   On $P_1$
%   \begin{enumerate}
%     \item Receive $\vx_2$ from $P_2$. Output
%     $\vx \asn \vx_1 + \vx_2$.
%   \end{enumerate}
%   On $P_2$
%   \begin{enumerate}
%     \item Receive $\vx_1$ from $P_1$. Output $\vx \asn \vx_1 + \vx_2$.
%   \end{enumerate}
% \end{Boxfig}







\subsection{Comparisons}
Given $\share{x}$ where $x \in \Z_{2^k}$ our goal is to compute a ring sharing of $\share{x > 0}$.
This translates into how to compute a sharing of the most significant bit $\share{\mathsf{msb}(x)}$.
Since we are in the 3PC semi-honest model there are few approaches to this:
\begin{enumerate}

   \item SecureNN: there's some extra preprocessing which needs to be done
   although the round complexity seems to be a bit lower than ABY3 - need to
   check this though. The downside is that it uses arithmetic modulo weird
   fields. IMO we should stick with ABY3 because it either operates on rings
   or boolean type (so a bit more standard than SecureNN).
   \item ABY3:
   There's an implementation of this in MP-SPDZ as well, using
   the \textit{split} option when compiling a program. The main idea is for
   each party to locally bit-decompose their shares over $\Z_{2^k}$ and then
   reconstruct the secret modulo $\Z_{2^k}$ using shares from $\Z^k_2$ and a
   binary adder. Once we get a sharing of the MSB then in order to use it in
   other operations we need to convert the sharing back to one in $\Z_{2^k}$.
   In ABY3 this was done using a three-party OT protocol. In MP-SPDZ this was
   achieved using a daBit.
  \item The newest one is using edaBits \cite{C:EGKRS20}. The downside of
   this method is that we need to implement edaBit generation and a boolean
   circuit doing binary addition. The upside of this is that for 3PC
   semi-honest the edaBit generation doesn't seem to be that hard ie
   we can avoid the cut and choose since everyone will execute the protocol
   semi-honestly.
\end{enumerate}

\noindent In Moose we use ABY3 protocol with a Kogge-Stone binary adder with
some minor optimizations for tensor operations. We avoid the three party OT
done by ABY3 by using a slightly more efficient share conversion described
below.

\subsubsection{Improved ABY3 boolean to ring sharing protocol}
When using tensors we can perform the share conversion $\share{\cdot}_2
\mapsto \share{\cdot}_{\Z_{2^k}}$ much easier by making use that all parties
follow the protocol specifications.
Acting as a trusted third party, $P_3$ generates a
daBit $(\share{b}_2, \share{b}_{2^k})$ locally and then shares it to $P_1$ and $P_2$.
Since the shared tuple contains tensors which sum up to the bit $b$ then $P_3$ can
send to $P_2$ only two seeds $s_2, s_{2^k}$ which are going to be later expanded
locally by $P_2$. Now $P_3$ sends to $P_1$ the difference $b_2 \oplus
\mathsf{Expand}(s_2, b_2.\mathsf{shape})$ and $b_{2^k} -
\mathsf{Expand}(s_{2^k}, b_{2^k}.\mathsf{shape})$.

Now $P_1$ and $P_2$ run a two-party protocol to convert $\share{x}_2$ to
$\share{x}_{2^k}$ by computing $c \asn \mathsf{Open}(\share{x}_2 \oplus
\share{b}_2)$ and then locally XOR-ing in the arithmetic domain
$\share{x}_{2^k} = c + \share{b}_{2^k} - 2 \cdot c \cdot \share{b}_{2^k}$.
Finally they convert the $(2,2)$ sharing of $\share{x}_{2^k}$ to a $(2,3)$
sharing using $\Proto{(2,2) \rightarrow (2,3)}$ from
Figure~\ref{fig:two-to-three}.


\begin{Boxfig}{$(2,2)$ to $(2,3)$ share conversion protocol for semi-honest
RSS.}{fig:two-to-three}{Protocol $\Proto{(2,2) \rightarrow (2,3)}$}
Input is $\share{x}$ which is additively shared amongst $P_1$ and $P_2$. Party $P_1$ holds $x_1$ and $P_2$ holds $x_2$ 
such that $x_1 + x_2 = x$. \\
$P_3$ does the following:
  \begin{enumerate}
    \item Sample seeds $s_1$ and $s_2$ and seeds $s_i$ to $P_i$.
    \item Compute $y_1 \sample \PRG(s_1)$ and $y_3 \sample \PRG(s_2)$.
    \item Set the output share as $(y_3, y_1)$.
 \end{enumerate}

$P_1$ and $P_2$ do the following:
\begin{enumerate}
   \item Parties extend the seeds received by $P_1$ computing $\hat{y}_1 \sample \PRG(s_1)$ and $P_2$ computing
   $\hat{y}_3 \sample \PRG(s_2)$.
   \item Compute $x_i - y_i$ and send it to $P_{3-i}$. 
   The received values are denoted by $\tilde{y}_i$.
   \item $P_1$ sets the output share as $(\hat{y}_1, x_1 - \hat{y}_1  + \tilde{y}_1)$
   \item $P_2$ sets the output share as $(x_2 - \hat{y}_2 + \tilde{y}_2, \hat{y}_3)$.
 \end{enumerate}
\end{Boxfig}




