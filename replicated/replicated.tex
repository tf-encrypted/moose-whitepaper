\section{Replicated Protocols}

In this section we describe our protocols for performing encrypted computations using the three-party replicated secret sharing scheme. Much of this work follows the lines of~\cite{CCS:AFLNO16} with some optimizations derived from the fact that we focus on tensorized computations. All protocols presented here operate in the honest-but-curious security model, meaning players are assumed to follow the protocols but may try to learn additional information from the messages they receive. We currently only present protocols for the operations needed to support linear regressions and associated metrics (Section~\ref{sec:linreg}).



One reason for choosing~\cite{CCS:AFLNO16} as our foundation is that it is currently the fastest protocol for achieving semi-honest three-party multiplications in the preprocessing model using ring arithmetic.
%An year after it was shown by \cite{SP:ABFLLN17} that the boolean version of~\cite{CCS:AFLNO16} can be converted into a protocol with active security that achieves $1$ billion AND triples per second with a cost of $7$ bits per AND gate.
%
Moreover, its ring variant can be efficiently upgraded to malicious security by (roughly) just repeating all procedures twice and use a "zero check" as described in~\cite{cryptoeprint:2020:1330}.

Finally, we also chose this line of work due to its efficiency at computing dot products by communicating a number of ring elements independent in the size of input tensors, and other optimizations made possible by being in the three-party setting.
%Note that the work of efficient dot products in MPC was inspired mainly from~\cite{CCS:LinNof17,C:CGHIKL18,EPRINT:ADEN19,cryptoeprint:2020:1330}.

% Later it was shown by Keller et al. \cite{SCN:KRSW18} how to extend the
% work of Araki et al. \cite{CCS:AFLNO16} to multiple parties assuming there is
% honest majority using some well-known techniques by Maurer
% \cite{SCN:Maurer02}. The caveat of Keller et al. \cite{SCN:KRSW18} is that
% there can be an exponential blowup in terms of $n \choose t$ where $t<n/2$ is
% the corruption threshold and $n$ the number of parties.
% To get a better picture of the available protocols with their underlying
% arithmetic we list the current state of the art protocols for obtaining
% honest-majority MPC. For readability purposes we abbreviate honest majority
% with HM:

% \begin{enumerate}
%   \item HM over large fields $\Fp$ \cite{CCS:LinNof17,C:CGHIKL18}. Although
%   the two constructions listed above are specialized for the malicious
%   setting (at most 1 party arbitrarily deviating from the protocol) these are
%   built assuming $\Frand, \Fmult$ hybrid model where $\Fmult$ needs to be a
%   multiplication protocol secure up to additive attacks. To get such an
%   $\Fmult$ we can simply take
%   a semi-honest protocol for HM for doing multiplications such as \cite{CCS:AFLNO16}.
%   \item HM for binary circuits $\F_2$. Note that Araki et al. \cite{CCS:AFLNO16}
%   works for binary circuits as well for the semi-honest case.
%   To get malicious security Chida et al. \cite{C:CGHIKL18} present a protocol
%   for small field multiplications using cut-and-choose techniques The cut and
%   choose framework works by generating large batches of triples
%   optimistically and then open a fraction to detect whether some cheating
%   occurred.
%   \item HM over small fields (eg. $\F_{2^8}$ which is suitable for AES). To
%   get malicious security, depending on the extension field size, we can
%   either repeat the computation $\sec/|\F|$ times or generate triples using cut and choose
%   from \cite{C:CGHIKL18}.
%   \item HM over rings $\mathbb{Z}_{2^k}$. Abspoel et al. \cite{EPRINT:ADEN19}
%   are the first ones who introduce vector dot-products of secret shares at a cost of
%   communicating a constant number of ring elements independently of the vector sizes
%   for the ring case - however they avoid giving any hint on how to realise
%   reactive computations, this is solved later in \cite{cryptoeprint:2020:1330}.
%   \item HM by mixing circuits: \cite{CCS:MohRin18,cryptoeprint:2020:1330}.
%   Escudero et al. \cite{C:EGKRS20} work in the more generic setting but they
%   have
%   improvements for the honest majority as well. They improve the mixed framework
%   specifically for $\Z_{2^k}$ and $\F_2$ very recently in \cite{cryptoeprint:2020:1330}.

% \end{enumerate}

% Note that almost every protocol can be built in the malicious setting using
% building blocks from the semi-honest protocols with the exception of binary
% triples $c = a \cdot b$ where $a,b \in \F_2$. For the binary triple case we
% need more complex techniques such as cut and choose. In our implementation we
% stick with Araki et al. protocol \cite{CCS:AFLNO16} while borrowing some
% improvements from \cite{cryptoeprint:2020:1330} such as probabilistic
% truncation.

% We
% should keep in mind that although triple generation for large fields is
% usually faster than rings there are benefits when working with ring sharings
% - such as faster methods for secret truncation which in turn makes RELU's
% faster.



\subsection{Notation}

The protocols perform computations in rings $\ring$ of form $\mathbb{Z}_{2^{k}}$, concretely $\mathbb{Z}_{2^{64}}$ and $\mathbb{Z}_{2^{128}}$, which allows us to emulate fixed-point arithmetic as needed for the linear regression use case. Throughout this section we use 
$$
\share{\vx}^{\mathsf{rep}}_\ring = \left( (\vx_1^1, \vx_2^1), (\vx_2^2, \vx_3^2), (\vx_3^3, \vx_1^3) \right)
$$
to denote a replicated value consisting of shares $\vx_1 = \vx_1^1 = \vx_1^3$, $\vx_2 = \vx_2^2 = \vx_2^1$, and $\vx_3 = \vx_3^3 = \vx_3^2$ over ring $\ring$ such that $\vx = \sum_{i=1}^3 \vx_i \in \Z_{2^k}$. It will typically be the case that $P_i$ holds $\vx_i^i$ and $\vx_{i+1}^i$. To ease the notation we occasionally write this simply as $\share{\vx}$, and we let indices wrap around such that for instance $P_{3+1} = P_1$.
%Note that $\share{x}$ hence also states that the $x_{i+1}$ held by $P_i$ matches the $x_{i+1}$ held by $P_{i+1}$. We furthermore use this to express statements about results of computing on secret shared values such as for instance $\share{\vz} = \share{\vx + \vy} = \share{\vx} + \share{\vy}$.

\subsection{Setup}

Some of the replicated protocols rely on per-session PRF keys produced during an initial setup phase as described by the $\mathsf{Setup}$ protocol in Figure~\ref{fig:replicated-setup}. Note that keys are distributed the same way as replicated shares so that $P_i$ ends up knowing both $k_i$ and $k_{i+1}$. Generation of individual keys by $\mathsf{GenPrfKey}$ is implemented by sampling 128 random bits using~\cite{libsodium}.

\begin{Boxfig}{Replicated setup.}{fig:replicated-setup}
  {Protocol $\mathsf{Setup}_{[P_1, P_2, P_3]}$}
  
  \begin{enumerate}
  \item On $P_i$ for $i \in [3]$:
  \begin{enumerate}
    \item $k^{i}_{i} \asn \mathsf{GenPrfKey}()$.
    \item Send $k^{i}_{i}$ to $P_{i-1}$.
    \item Receive $k^{i}_{i+1}$ from $P_{i+1}$.
  \end{enumerate}
  
  \item Return $\prfkeys = \left( (k^1_1, k^1_2), (k^2_2, k^2_3), (k^3_3, k^3_1) \right)$.
  \end{enumerate}
\end{Boxfig}


\subsection{Addition and Subtraction}

To compute the addition of two replicated tensors $\share{\vz} = \share{\vx + \vy}$
the parties simply add their local shares of $\share{\vx}$ and $\share{\vy}$
as described in protocol~$\mathsf{RepAdd}$ in Figure~\ref{fig:replicated-add}. Subtraction as shown in Figure~\ref{fig:replicated-sub} is almost identical, with the parties simply subtracting their shares locally instead of adding. Correctness and security follow from~\cite{CCS:ABFLNO16}.


\msubsubsection
{$\mathsf{RepAdd}_{[P_1, P_2, P_3]}\left(\share{\vx}, \share{\vy} \right)$}
\label{fig:replicated-add}
  Replicated addition protocol.

  \begin{enumerate}
  \item Let $\left( (\vx_1^1, \vx_2^1), (\vx_2^2, \vx_3^2), (\vx_3^3, \vx_1^3) \right) = \share{\vx}$.

  \item Let $\left( (\vy_1^1, \vy_2^1), (\vy_2^2, \vy_3^2), (\vy_3^3, \vy_1^3) \right) = \share{\vy}$.

  \item On $P_i$ for $i \in [3]$:
  \begin{enumerate}
    \item $\vz_i^i \asn \vx_i^i + \vy_i^i$
    \item $\vz_{i+1}^i \asn \vx_{i+1}^i + \vy_{i+1}^i$
  \end{enumerate}

  \item Return $\share{\vz} = \left( (\vz_1^1, \vz_2^1), (\vz_2^2, \vz_3^2), (\vz_3^3, \vz_1^3) \right)$.
  \end{enumerate}

\subsubsection
{$\mathsf{RepSub}_{[P_1, P_2, P_3]}\left(\share{\vx}, \share{\vy} \right)$}
\label{fig:replicated-sub}
  Replicated subtraction protocol.

  \begin{enumerate}
  \item Let $\left( (\vx_1^1, \vx_2^1), (\vx_2^2, \vx_3^2), (\vx_3^3, \vx_1^3) \right) = \share{\vx}$.

  \item Let $\left( (\vy_1^1, \vy_2^1), (\vy_2^2, \vy_3^2), (\vy_3^3, \vy_1^3) \right) = \share{\vy}$.

  \item On $P_i$ for $i \in [3]$:
  \begin{enumerate}
    \item $\vz_i^i \asn \vx_i^i - \vy_i^i$
    \item $\vz_{i+1}^i \asn \vx_{i+1}^i - \vy_{i+1}^i$
  \end{enumerate}

  \item Return $\share{\vz} = \left( (\vz_1^1, \vz_2^1), (\vz_2^2, \vz_3^2), (\vz_3^3, \vz_1^3) \right)$.
  \end{enumerate}

\subsection{Multiplication}

To see how this works for replicated secret sharing each party $P_i$ holds a
sharing of $x$ as $(x_i, x_{ \inc{i} })$ and a sharing of $y$ as $(y_i, y_{
\inc{i} })$. To compute a sharing $\share{x \cdot y}$ using the shares of $x$
and $y$ parties do the following:

\commentM{TODO: turn into proper protocol; describe how $\alpha + \beta + \gamma$ is computed}
\commentD{Done}


\begin{Boxfig}{Replicated multiplication}{fig:replicated-mul}
  {Protocol $\mathsf{RepMul}_{[P_1, P_2, P_3]}\left( R; \prfkeys, \share{\vx}_R, \share{\vy}_R \right)$}
  \begin{enumerate}
  \item Let $\left( (\vx_1^1, \vx_2^1), (\vx_2^2, \vx_3^2), (\vx_3^3, \vx_1^3) \right) = \share{\vx}_R$.
  \item Let $\left( (\vy_1^1, \vy_2^1), (\vy_2^2, \vy_3^2), (\vy_3^3, \vy_1^3) \right) = \share{\vy}_R$.
  
  \item With $P_i$ for $i \in [3]$:
  \begin{enumerate}
    \item $\vv_i^i \asn \vx_i^i \cdot \vy_i^i + \vx_i^i \cdot \vy_{i+1}^i + \vx_{i+1}^i \cdot \vy_i^i$.
    \item $\shape_i \asn \mathsf{Shape}(\vv_i^i)$.
  \end{enumerate}

  \item $\valpha_1, \valpha_2, \valpha_3 \asn \mathsf{SampleZero}_{[P_1, P_2, P_3]}\big( R; \prfkeys, \shape_1, \shape_2, \shape_3 \big)$

  \item With $P_i$ for $i \in [3]$:
  \begin{enumerate}
    \item $\vz_i^i \asn \vv_i^i + \alpha_i$.
    \item Send $\vz_i^i$ to $P_{i-1}$.
    \item Receive $\vz_{i+1}^i$ from $P_{i+1}$.
  \end{enumerate}

  \item Return $\share{\vz}_R = \left( (\vz_1^1, \vz_2^1), (\vz_2^2, \vz_3^2), (\vz_3^3, \vz_1^3) \right)$.
  \end{enumerate}
\end{Boxfig}


\begin{Boxfig}{Sample Zero protocol.}{fig:replicated-sample-zero}
  {Protocol $\mathsf{SampleZero}_{[P_1, P_2, P_3]}(R; \sid, \prfkeys, \shape_1, \shape_2, \shape_3)$}
  
  \begin{enumerate}
  \item Let $\left( (k_1^1, k_2^1), (k_2^2, k_3^2), (k_2^3, k_1^3) \right) = \prfkeys$ be PRF keys generated during setup.
  
  \item With $P_i$ for $i \in [3]$:
  \begin{enumerate}
    \item ${\seed}^i_i \asn \mathsf{PRF}(k^i_i, \sid \| i)$.
    \item ${\seed}^i_{i+1} \asn \mathsf{PRF}(k^i_{i+1}, \sid \| i + 1)$.
    \item $\alpha_i \asn \RingSample(R; \seed^i_i, \shape_i) - \RingSample(R; \seed^i_{i+1}, \shape_i)$.
  \end{enumerate}
  
  \item Return $\alpha = \left( \alpha_1, \alpha_2, \alpha_3 \right)$.
  \end{enumerate}
\end{Boxfig}
\subsection{Sum and Mean}

Similarly to addition, we build a summation protocol (Figure~\ref{fig:replicated-sum}) to compute the sum along an axis of a replicated tensor by simply computing the sum along the axis of the individual shares. This can be extended to a protocol for computing the mean by subsequently performing a public multiplication with the fixed-point inverse of the size of the axis $n$.


\begin{Boxfig}{Replicated summation.}{fig:replicated-sum}
  {Protocol $\mathsf{RepSum}_{[P_1, P_2, P_3]}\left(\share{\vx}, \mathsf{axis} \right)$}
  
  \begin{enumerate}
  \item Let $\left( (\vx_1^1, \vx_2^1), (\vx_2^2, \vx_3^2), (\vx_3^3, \vx_1^3) \right) = \share{\vx}$.
  
  \item On $P_i$ for $i \in [3]$:
  \begin{enumerate}
    \item $\vz_i^i \asn \mathsf{sum}(\vx_i^i, \mathsf{axis})$
    \item $\vz_{i+1}^i \asn \mathsf{sum}(\vx_{i+1}^i, \mathsf{axis})$
  \end{enumerate}
  
  \item Return $\share{\vz} = \left( (\vz_1^1, \vz_2^1), (\vz_2^2, \vz_3^2), (\vz_3^3, \vz_1^3) \right)$.
  \end{enumerate}
\end{Boxfig}

\subsection{Secret dot-products using MPC}

The goal is to compute $\sum_{k=1}^m \share{x_k} \cdot \share{y_k}$.

\subsubsection{3-party semi-honest (HM)}

Although this was written with $x$ and $y$ being a single ring element,
i.e. $x,y \in \Z_{2^k}$ the same protocols works when $x, y \in \Z^m_{2^k}$
and the ring multiplication operator `$\cdot$` is replaced with the dot
product. In this manner $x \cdot y = \sum_{i=1}^m x_i y_i \in \Z_{2^k}$ while
$x, y \in \Z^m_{2^k}$. This can be extended for arbitrary length tensors as long as
$\alpha, \beta, \gamma$ have the correct dimensions to mask randomize $x \cdot y$.

\subsubsection{3-party malicious (HM)}
Note that the first progress
for this problem was done for the field case by Chida et al. \cite{C:CGHIKL18}.
For the ring case see \cite{EPRINT:ADEN19,cryptoeprint:2020:1330}.

\subsubsection{Dishonest majority}

Here parties fetch a triple for each $\share{x_k} \cdot \share{y_k}$ multiplication
ending with a cost of $m$ preprocessed triples. This is valid for secret sharing (SPDZ-type
\cite{EC:KelPasRot18,C:DPSZ12,C:CDESX18})
but also for garbled circuits (\cite{AC:HazSchSor17,CCS:WanRanKat17b}) type protocols.


\subsection{Sharing data}

In Figure~\ref{fig:replicated-share} we show how a party $D$ can translate a
private tensor $\vx$ (known only to $D$) into a replicated tensor.
Here we distinguish two cases: a) when $D$ is amongst the parties on the replicated placement
i.e. $D \in \{P_1, P_2, P_3\}$ and b) when $D$ is an external party, not in the replicated set.

In the first case the parties use a similar mechanism as in the
$\mathsf{ZeroShare}$ to produce a set of replicated shared seeds. Wlog., consider $D=P_i$,
then $P_i$ sends to $P_{i-1}$ the shape of $\vx$ in order for $P_{i-1}$ to derive a correct
sized random tensor using the shared seed. Then the inputting party
$P_i$ masks the private tensor with a random value $\vx - \vx_i^i$ and sends it
to $P_{i+1}$.

Correctness can be seen from the fact that at the end of the protocol all
$\vx_1^j + \vx_2^j + \vx_3^{j+1} = \vx$ for all $j \in [1,3]$. For example, when
$j = 1$ we have the following: $$\vx_1^1 + \vx_2^1 + \vx_3^2 = \mathbf{0} +
\vx_2^1 + (\vx - \vx_2^2) = \vx $$ due to $\vx_2^1 = \vx_2^2$ since they were
sampled from identical seeds. Security follows from~\cite{CCS:ABFLNO16}
or~\cite{CCS:MohRin18}.


In the second case $D$ is an external party sharing their input. Two of the
parties $(P_1, P_2)$ produce the randomness used to mask $\vx$ in $\vx_3 \asn
\vx - \vx_1 - \vx_2$ such that any two parties can reconstruct the secret
$\share{\vx}$ afterwards. Correctness can be seen as in the first case since all
$\vx_1^j + \vx_2^j + \vx_3^{j+1} = \vx$ for all $j \in [1,3]$. For example, when
$j = 1$ we have: $$ \vx_1^1 + \vx_2^1 + \vx_3^2 = \vx_1 + \vx_2 + (\vx - \vx_1 -
\vx_2) = \vx.$$ The security argument differs slightly then the first case since
$D$ broadcasts $\vx - \vx_1 - \vx_2$. Since $\vx_1$ and $\vx_2$ are joint
randomness produced by $P_1$ and $P_2$ then any two parties (no less) can reconstruct the
secret.

\subsection{Sharing an input}


\begin{Boxfig}{Replicated share}{fig:replicated-share}
  {Protocol $\mathsf{RepShare}[P_1, P_2, P_3](R; P_i, \vx)$}
  
  \begin{enumerate}
  \item Suppose each party $P_j$ has a shared seeds $(\seed_j^j, \seed_{j+1}^j)$.
  \item Divide the set of parties into two sets: $I = \{P_i\}, O = \{P_1, P_2, P_3\} - \{P_i\}$.
  \item On Host($P_{i-1}$):
  \begin{enumerate}
    \item $\vx_{i-1}^{i-1} = \mathsf{zero}(\shape(\vx))$.
    \item Compute $\vx_i^{i-1} = \RingSample(R; \shape(\vx), \seed_i^{i-1})$.
  \end{enumerate}
  \item On Host($P_i$):
  \begin{enumerate}
    \item $\vx_i^i = \RingSample(R; \shape(\vx), \seed_{i}^i)$.
    \item $\vx_{i+1}^i = \vx - \vx_i^i$.
    \item Send $\vx_{i+1}^i$ to $P_{i+1}$.
  \end{enumerate}
  \item On Host($P_{i+1}$):
  \begin{enumerate}
    \item Receive $\vx_{i+1}^i$ from $P_i$. Set $\vx_{i+1}^{i+1} = \vx_{i+1}^i$.
    \item $\vx_{i+2}^{i+1} = \mathsf{zero}(\shape(\vx))$.
  \end{enumerate}
   \item Return $\share{\vx}_R = \left( (\vx_1^1, \vx_2^1), (\vx_2^2, \vx_3^2), (\vx_3^3, \vx_1^3) \right)$.
  \end{enumerate}
\end{Boxfig}



\subsection{Revealing secrets}

We sometimes use interchangeably the terms \textit{open} and \textit{reveal} a secret to refer to taking a replicated tensors $\share{\vx}$ and making it $\vx$ known to a single party  (Figure~\ref{fig:replicated-open}) or to all parties (Figure~\ref{fig:additive-open}).
Note that Figure~\ref{fig:replicated-open} corresponds to a replicated opening whereas Figure~\ref{fig:additive-open} to an two out of two additive sharing scheme which is used in some more
advanced protocols below for truncation or share conversion. Correctness is guaranteed as long as
the secret was correctly shared. This is also secure as we only send the required
shares to the set of parties that want the secret revealed.

\begin{Boxfig}{Replicated opening protocol to a specific party $P_i$.}{fig:replicated-open}
  {$\Open_{[P_1, P_2, P_3]} \left( \share{\vx}^{\mathsf{rep}} \right)$}

  Let $P_i$ be the player to which $\vx$ is to be revealed.

  \begin{enumerate}
  \item Let $\left( (\vx_1^1, \vx_2^1), (\vx_2^2, \vx_3^2), (\vx_3^3, \vx_1^3) \right) = \share{\vx}^{\mathsf{rep}}$.
  \item On $P_{i+1}$:
  \begin{enumerate}
    \item Send $\vx_{i+2}^{i+1}$ to $P_{i}$.
  \end{enumerate}
  \item On $P_i$:
  \begin{enumerate}
   \item Receive $\vx_{i+2}^{i+1}$ from $P_{i}$.
   \item $\vx \asn \vx_{i+2}^{i+1} + \vx_i^i + \vx_{i+1}^{i}$.
  \end{enumerate}
  \item Return $\vx$.
\end{enumerate}
\end{Boxfig}


\begin{Boxfig}{Additive opening protocol.}{fig:additive-open}
  {$\Open_{[P_1, P_2]} \left( \share{\vx}^{\mathsf{adt}} \right)$}
  \begin{enumerate}
  \item Let $(\vx_1, \vx_2) = \share{\vx}^{\mathsf{adt}}$.
  \item On $P_i$ for $i \in [2]$:
  \begin{enumerate}
    \item Send $\vx_i$ to $P_{i+1}$.
    \item Receive $\vx_{i+1}$ from $P_{i+1}$.
    \item $\vx^i \asn \vx_1 + \vx_2$.
  \end{enumerate}
  \item Return $\vx^1$ and $\vx^2$.
\end{enumerate}
\end{Boxfig}

\subsection{Truncation}
Given secret shared $\share{a}$ the goal is to compute $\share{a \bmod 2^m}$
where $m$ is a public constant. This is also denoted as
$\mathsf{trunc}(\share{a}, m)$. There are many ways to truncate a secret that
can be roughly classified into two lines of thought: a) probabilistic
truncation where there is a rounding error with probability $p$ and b) deterministic
truncation where the output is always exact.

Current state of the art techniques in truncating a secret use
dabits/edabits as described by Escudero et al. in \cite{C:EGKRS20} and
various share conversions. In \cite{C:EGKRS20} there are several techniques
to achieve both probabilistic and deterministic truncation for field and ring
case. To avoid edabit generation required for generating random bounded integers
by the truncation protocols in \cite{C:EGKRS20} we use the protocols from
\cite{PoPETS:DalEscKel20}. We describe their probabilistic protocols for replicated secret
sharing below in Fig~\ref{fig:truncpr-rep-ss}.
Note that although the protocol in Fig~\ref{fig:truncpr-rep-ss} looks
very efficient, the edabit variant from \cite{C:EGKRS20} requires
about $\approx 35$ times less communication. \commentD{Need to re-check this, I think it's not that bad}



\begin{Boxfig}{Truncation protocol for semi-honest
RSS.}{fig:truncpr-rep-ss}{Protocol $\Proto{\mathsf{TruncPr}}$}
On input
$\mathsf{TruncPrSp}(\share{\vx}, m)$ from all parties where $\vx$ is a sharing
of a tensor over $\Z_{2^k}^{\mathsf{s}}$: \\
$P_3$ does the following:
  \begin{enumerate}
    \item Sample key $k_3 \sample \mathsf{SampleKey}()$.
    \item For $i \in [k]$ call $\seed(i) \asn \derives(k_3, i)$
    \item For $i \in [k]$ sample locally random bits $\mathbf{b}_i \in \Z_{2^k}^{\mathsf{s}}$.
    This is done by calling
    generate $\mathbf{b}_i \sample \mathsf{Sample}(\seed(i), \mathsf{s}, [0,1])$.
    \item Compute
    $\mathbf{r}_{\mathsf{bot}} = \sum_{i=0}^{k-1}2^i \cdot \vb(i)$,
    $\mathbf{r}_{\mathsf{msb}} \asn \vb(k-1)$ and
    $\mathbf{r}_{\mathsf{top}} = \sum_{i=m}^{k-2} \vb(i) \cdot 2^{i-m}$. Send
    the corresponding shares to $P_1$ and $P_2$.
    \item For each of $\vr_{\mathsf{bot}}, \vr_{\mathsf{msb}} \vr_{\mathsf{top}}$
    generate 2-out-of-2 additive shares by calling $\Proto{\mathsf{AdditiveShare}}(\cdot)$
    on each entry and get $
    \vr\ui_{\mathsf{bot}}, \vr\ui_{\mathsf{msb}}, \vr\ui_{\mathsf{top}}$ for $i \in [0,1]$. Send the corresponding $\vr\ui$ to party $P_i$.
 \end{enumerate}
$P_1$ together with $P_2$ do the following:

\begin{enumerate}
   \item Convert $\share{\vx}$ into a $2$-out-of-$2$ sharing.
   \begin{enumerate}
      \item On $P_1$ set $\vx'_1 \asn \vx_1 + \vx_2$.
      \item On $P_2$ set $\vx'_2 \asn \vx_2$.
   \end{enumerate}

   \item $P_1$ and $P_2$ then execute the protocol $\Proto{\mathsf{TruncPR}}(\share{\vx'}, m)$
   using the $2$-out-of-$2$ sharing of $\vx$. Store the output as $\vy$.
\end{enumerate}

 Now parties convert the $2$-out-of-$2$ sharing $\vy$ to a $2$-out-of-$3$
 sharing. All parties output $\Proto{(2,2)\rightarrow(2,3)}(\vy)$.

%  \begin{enumerate}
%    \item Each $P_i$ for $i \in [1, 2]$ sets the output of $\mathsf{TruncPr}$
%    as $y'_i$ and $\hat{y}_i$ as the sharing received from $P_3$. Send $y'_i -
%    \hat{y}_i$ to $P_{3-i}$. The received values are denoted by $\tilde{y}_i$.
%    \item $P_1$ sets the output share as $(y_1, y'_1 - \hat{y}_1 + \tilde{y}_1)$
%    \item $P_2$ sets the output share as $(y'_2 - \hat{y}_2 + \widehat{y}_2, y_3)$.
%  \end{enumerate}
\end{Boxfig}

\begin{Boxfig}{Generic truncation protocol.}{fig:truncpr-generic}
  {Protocol $\Proto{\mathsf{2PCTruncPr}}(\share{\vx}, m)$}
  Note this is a two party protocol, meaning only $P_1$ and $P_2$ are involved.
  Assume $\vx \in \Z_{2^k}^{\shape}$ has $\mathsf{MSB}(\vx) = 0$. If $\vx$ is negative then
  we make it positive by adding $2^{k-1}$ to it and assume that $\mathsf{abs}(\vx) < 2^{k-1}$.
  \\
  On $P_1$ and $P_2$:
  \begin{enumerate}
    \item Compute $\vc = \Open(\share{\vx} + \share{\vr_{\mathsf{bot}}})$.
    \item $\vc' \asn (\vc/2^m) \bmod 2^{k-m-1}$. This can be done using one $\mathsf{ring\_shl}$
    followed by one $\mathsf{ring\_shr}$ operation.
    \item Compute $\share{\vb} \asn \share{\vr_{\mathsf{msb}}} \oplus (\vc / 2^{k-1})$. Note here that $\oplus$
    in arithmetic circuits boils down to $a + b - 2a\cdot b$. Here one of the operands is public so
    the result can be computed using only local operations.
    \item Output $\vc' - \share{\vr_{\mathsf{top}}} + \share{\vb} \cdot 2^{k-m-1}$.
 \end{enumerate}

\end{Boxfig}


\begin{Boxfig}{Local conversion to $(2,2)$ additive shares}{fig:get-additive}
  {Protocol $\Proto{\mathsf{AdditiveShare}}(\vx)$}
  Roles: D, $P_1$, $P_2$ where $\vx.\mathsf{placement}$ is Host(D) \newline
  With Host(D):
  \begin{enumerate}
    \item $k \asn \mathsf{SampleKey}()$.
    \item $\seed \asn \derives(k, 0)$.
    \item $\vx^1 \asn \mathsf{SampleUniform}(\vx.\mathsf{shape}, \seed)$.
    \item $\vx^2 \asn \vx - \vx^1$.
    \item Send $\vx^i$ to $P_i$.
  \end{enumerate}
  With Host($P_1$):
  \begin{enumerate}
      \item Receive $\vx^1$ from Dealer.
  \end{enumerate}
  With Host($P_2$):
  \begin{enumerate}
      \item Receive $\vx^2$ from Dealer.
  \end{enumerate}
  Let $\vx = Additive(\vx^1, \vx^2)$.
\end{Boxfig}

\begin{Boxfig}{Opening protocol}{fig:open2-2}
  {Protocol $\Proto{\Open}^{(2,2)}(\share{\vx})$}
  On $P_1$:
  \begin{enumerate}
    \item Send $\vx_1$ to $P_2$.
  \end{enumerate}
   On $P_2$:
  \begin{enumerate}
    \item Send $\vx_2$ to $P_1$.
  \end{enumerate}
  On $P_1$
  \begin{enumerate}
    \item Receive $\vx_2$ from $P_2$. Output
    $\vx \asn \vx_1 + \vx_2$.
  \end{enumerate}
  On $P_2$
  \begin{enumerate}
    \item Receive $\vx_1$ from $P_1$. Output $\vx \asn \vx_1 + \vx_2$.
  \end{enumerate}
\end{Boxfig}


% \begin{Boxfig}{Opening protocol}{fig:open2-3}
%   {Protocol $\Proto{\mathsf{OpenTo}}^{(2,3)}(\share{\vx})$}
%   On $P_i$:
%   \begin{enumerate}
%     \item Send $\vx_1$ to $P_2$.
%   \end{enumerate}
%    On $P_2$:
%   \begin{enumerate}
%     \item Send $\vx_2$ to $P_1$.
%   \end{enumerate}
%   On $P_1$
%   \begin{enumerate}
%     \item Receive $\vx_2$ from $P_2$. Output
%     $\vx \asn \vx_1 + \vx_2$.
%   \end{enumerate}
%   On $P_2$
%   \begin{enumerate}
%     \item Receive $\vx_1$ from $P_1$. Output $\vx \asn \vx_1 + \vx_2$.
%   \end{enumerate}
% \end{Boxfig}






%\subsection{Comparison}

%In order to compute the MAPE (mean absolute percentage error) metric which boils down to compute the absolute value of a secret $\share{|\vx|}$. Since our primitives work with ring arithmetic one of its most efficient evaluation of the absolute value is $\share{|\vx|} = \share{\vx} \cdot \share{\vx > 0}$.

The problem of computing secure comparisons translates directly into computing a sharing of the most significant bit $\share{\mathsf{msb}(\vx)}$.
In the 3PC semi-honest model there are few approaches to this:
\begin{enumerate}
   \item ABY3 \cite{CCS:MohRin18} The main idea is for
   each party to locally bit-decompose their shares over $\Z_{2^k}$ and then
   reconstruct the secret modulo $\Z_{2^k}$ using shares from $\Z^k_2$ and a
   binary adder. Once a boolean sharing of the MSB is computed using the binary adder
   this is converted to a ring sharing in $\Z_{2^k}$.
   In ABY3 this was done using a three-party OT protocol. In MP-SPDZ \cite{CCS:Keller20} 
   the boolean to ring sharing conversion was achieved using a daBit.
   \item SecureNN \cite{PoPETS:WagGupCha19} 
   similar to ABY3 with the downside that it uses arithmetic modulo
   some small fields but avoids ring-to-boolean conversion.
  \item Comparisons using edaBits \cite{C:EGKRS20}. The 3PC case has roughly the same cost as ABY3.
\end{enumerate}

We use the MSB extraction protocol from ABY3 with a Kogge-Stone binary adder and minor optimizations for tensor operations. We avoid the special three-party OT that ABY3 had along with edaBits preprocessing by introducing our custom protocol \textsf{B2A} for binary to ring share conversion in Figure~\ref{fig:b2a-protocol}.
The MSB protocol can be found in Figure~\ref{fig:msb-protocol} where $\overline{\cdot}$ is used to denote a vector with $k$ elements indexed using $(i)$ for $i \in [k]$.




\begin{Boxfig}{Ring bit decomposition to binary shares}{fig:bitdecraw}
  {$\mathsf{BitDecRaw}_{[P_1, P_2, P_3]} \left( \share{\vx}^\mathsf{rep}_{2^k} \right) \rightarrow [\share{\cdot}^\mathsf{rep}_2; k]$}
  \begin{enumerate}
  \item Let $\left( (\vx_1^1, \vx_2^1), (\vx_2^2, \vx_3^2), (\vx_3^3, \vx_1^3) \right) = \share{\vx}^{\mathsf{rep}}_{2^k}$.

  \item On $P_1$:
  \begin{enumerate}
    \item $\overline{\va} \asn \textsf{LocalBitDec}(\vx^1_1 + \vx^1_2)$.
    \item $\overline{\share{\va}_2}(i) \asn \mathsf{ReplicatedShare}(\Z_2; \overline{\va}(i))$ for $i \in [k]$.
  \end{enumerate}

  \item On $P_2$:
  \begin{enumerate}
      \item $\overline{\vb^2} \asn \textsf{LocalBitDec}(\vx_3^2)$.
  \end{enumerate}

  \item On $P_3$:
  \begin{enumerate}
      \item $\overline{\vb^3} \asn \textsf{LocalBitDec}(\vx_3^3)$.
  \end{enumerate}
  \item Let $\overline{\share{\vb}_2}(i) = \left( (0, 0), (0, \overline{\vb^2}(i)), (\overline{\vb^3}(i), 0) \right)$ for $i \in [k]$.
  \item Return $\textsf{BinaryAdder}(\overline{\share{\va}_2}, \overline{\share{\vb}_2})$.
\end{enumerate}

\end{Boxfig}


\begin{Boxfig}{MSB computation from a replicated ring share}{fig:msb-protocol}
  {$\mathsf{MSB}_{[P_1, P_2, P_3]} \left( \share{\vx}^\mathsf{rep}_{2^k} \right)$}
  \begin{enumerate}
  \item $\overline{\share{\vb}} \asn \mathsf{BitDecRaw}(\share{\vx})$.
  \item Return $\textsf{B2A}(\overline{\share{\vb}_2}(k-1))$.
  \end{enumerate}

\end{Boxfig}



\subsubsection{Improved ABY3 boolean to ring sharing protocol}
When the inputs are tensors we can perform the share conversion $\share{\cdot}_2
\mapsto \share{\cdot}_{2^k}$ ($\mathsf{B2A}$ function)
more efficient by making use of the fact that all parties
follow the protocol specifications. The starting point is using a similar idea
from daBit/edaBit \cite{INDOCRYPT:RotWoo19,C:EGKRS20}
line of work with the twist that $P_3$ generates the preprocessing
material. We fully describe this share conversion protocol in Figure~\ref{fig:b2a-protocol}.

Acting as a trusted third party, $P_3$ generates a random
daBit $(\share{\vb}_2, \share{\vb}_{2^k})$ locally and shares it to $P_1$ and $P_2$. Then $P_1$ and $P_2$ run a two-party protocol to convert $\share{\vx}_2$ to
$\share{\vx}_{2^k}$ by computing $\vc \asn \vx_2 \oplus \vb_2$ and then locally XOR-ing in the arithmetic domain $\vx_{2^k} = \vc + \vb_{2^k} - 2 \cdot \vc \cdot \vb_{2^k}$.
Finally they convert back to a replicated sharing using $\mathsf{AdditiveToReplicated}$ from Figure~\ref{fig:two-to-three}.

\begin{Boxfig}{Binary to arithmetic share conversion computation.}{fig:b2a-protocol}
  {$\mathsf{B2A}_{[P_1, P_2, P_3]} \left( \shareB{\vx}^\mathsf{rep} \right)$}
  \begin{enumerate}
  \item Let $\left( (\vx_1^1, \vx_2^1), (\vx_2^2, \vx_3^2), (\vx_3^3, \vx_1^3) \right) = \shareB{\vx}^\mathsf{rep}$.
  \item On $P_1$:
  \begin{enumerate}
  \item $\vx^1_{12} \asn \vx_1^1 \oplus \vx_2^1$.
  \item $\shape \asn \mathsf{Shape}(\vx^1_1)$.
  \end{enumerate}
  \item Let $\shareB{\vx}^\mathsf{add} = (\vx^1_{12}, \vx_3^2)$.
  \item $\shareB{\vb}^\mathsf{add}, \shareR{\vb}^\mathsf{add} \asn \textsf{DaBit}_{[D=P_3, P_1, P_2]}(\shape)$.
  \item $\shareB{\vc}^{\mathsf{add}} \asn \shareB{\vb}^\mathsf{add} \oplus \shareB{\vx}^{\mathsf{add}}$.
  \item $\vc \asn \Open_{[P_1, P_2]}(\shareB{\vc}^{\mathsf{add}})$.
  \item $\shareR{\vx}^\mathsf{add} \asn \vc + \shareR{\vb}^{\mathsf{add}} - 2 \cdot \vc \cdot \shareR{\vb}^{\mathsf{add}}$.
  \item Return $\textsf{AdditiveToReplicated}_{[P_1,P_2,P_3]}(\shareR{\vx}^\mathsf{add})$.
\end{enumerate}

\end{Boxfig}


\begin{Boxfig}{DaBit protocol generation}{fig:dabit-protocol}
  {Protocol $\Proto{DaBit[D, P_1, P_2]}(\shape)$}
  Roles: $P_1$, $P_2, P_3$ where $P_3$ acts as a dealer to distribute
  an additive share of a bit of shape $\shape$ in the boolean and arithmetic domain.
  \begin{enumerate}
  \item With Host(D):
  \begin{enumerate}
    \item $\seed \asn \derives(\dkey_D, \GenNonce)$.
    \item $\vb \asn \RingSample(\seed, \shape, \{0, 1\})$.
    \item $\seed(\textsf{bin}) \asn \derives(\dkey_D, \GenNonce), \seed(\textsf{ring}) \asn \derives(\dkey_D, \GenNonce)$.
    \item Send $\seed(\textsf{bin})$ and $\seed(\textsf{ring})$ to $P_1$.

  \end{enumerate}
  \item  With Host($P_1$):
 \begin{enumerate}
     \item Receive $\seed(\textsf{bin})$ and $\seed(\textsf{ring})$ from Dealer.
     \item Locally compute $\vb^1_{2^k} \asn \RingSample(\seed(\textsf{ring}), \shape, \Z_{2^k})$.
     \item Locally compute $\vb^1_{2} \asn \RingSample(\seed(\textsf{bin}), \shape, \{0, 1\})$.
 \end{enumerate}
\item With Host(D):
\begin{enumerate}
    \item Sample $\vb^1_{2^k} \asn \RingSample(\seed(\textsf{ring}), \shape, \Z_{2^k})$. Set $\vb^2_{2^k} = \vb - \vb^1_{2^k}$.
    \item Sample $\vb^1_{2} \asn \RingSample(\seed(\textsf{bin}), \shape, \{0, 1\})$. Set $\vb^2_{2} = \vb - \vb^1_{2}$. 
    \item Send $\vb^2_{2^k}$ and $\vb^2_2$ to $P_2$.
\end{enumerate}
\item With Host($P_2$):
\begin{enumerate}
    \item Receive $\vb^2_{2^k}$ and $\vb^2_2$ from Dealer.
\end{enumerate}
\item Let $\shareR{\vb} = \text{Additive}(\vb^1_{2^k}, \vb^2_{2^k})$ and $\shareB{\vb} = \text{Additive}(\vb^1_2, \vb^2_2)$. Output $(\shareR{\vb}, \shareB{\vb})$.
\end{enumerate}
\end{Boxfig}


% 
\begin{Boxfig}{$(2,2)$ to $(2,3)$ share conversion protocol for semi-honest
RSS.}{fig:two-to-three}{Protocol $\Proto{(2,2) \rightarrow (2,3)}$}
Input is $\share{x}$ which is additively shared amongst $P_1$ and $P_2$. Party $P_1$ holds $x_1$ and $P_2$ holds $x_2$ 
such that $x_1 + x_2 = x$. \\
$P_3$ does the following:
  \begin{enumerate}
    \item Sample seeds $s_1$ and $s_2$ and seeds $s_i$ to $P_i$.
    \item Compute $y_1 \sample \PRG(s_1)$ and $y_3 \sample \PRG(s_2)$.
    \item Set the output share as $(y_3, y_1)$.
 \end{enumerate}

$P_1$ and $P_2$ do the following:
\begin{enumerate}
   \item Parties extend the seeds received by $P_1$ computing $\hat{y}_1 \sample \PRG(s_1)$ and $P_2$ computing
   $\hat{y}_3 \sample \PRG(s_2)$.
   \item Compute $x_i - y_i$ and send it to $P_{3-i}$. 
   The received values are denoted by $\tilde{y}_i$.
   \item $P_1$ sets the output share as $(\hat{y}_1, x_1 - \hat{y}_1  + \tilde{y}_1)$
   \item $P_2$ sets the output share as $(x_2 - \hat{y}_2 + \tilde{y}_2, \hat{y}_3)$.
 \end{enumerate}
\end{Boxfig}


\begin{Boxfig}{Absolute value computation from a replicated ring share.}{fig:abs-protocol}
  {Protocol $\Proto{abs[P_1, P_2, P_3]}(\share{\vx})$}
  Roles: Replicated, $P_1$, $P_2, P_3$ where $\vx \in \Z_{2^k}^\shape$ and
  $\vx.\mathsf{placement}$ is on Replicated($P_1, P_2, P_3$). \newline
  With Replicated($P_1, P_2, P_3$):
  \begin{enumerate}
    \item Compute $\share{\vb} = \Proto{\mathsf{msb}}(\vx)$.
    \item Let $\share{\vs} \asn 1 - 2 \cdot \share{\vb}$.
    \item Output $\share{\vs} \cdot \share{\vb}$.
  \end{enumerate}

\end{Boxfig}


