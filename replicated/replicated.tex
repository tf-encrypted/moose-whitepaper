\section{Replicated Placements}

In the last couple of years there was a boom in MPC for honest majority,
especially in 3PC one. I think it all started with Araki et al.
\cite{CCS:AFLNO16} which at the moment provides the best performance for
semi-honest security in the preprocessing model. A year after they converted
the protocol to active security which achieves $1$ billion AND triples per
second \cite{SP:ABFLLN17} which had a cost of $7$ bits per AND gate. A couple
of folks showed later how to take the Araki et al. \cite{CCS:AFLNO16}
blueprint and extend it to multiple parties assuming there is honest majority
\cite{SCN:KRSW18} using some known techniques by Maurer \cite{SCN:Maurer02}
with the caveat that there can be an exponential blowup in terms of $n
\choose t$ where $t<n/2$ is the threshold and $n$ the number of parties. In
this way we had some intuition on why the $(3,1)$ case was so efficient.


\subsection{Rationale}
\section{Choosing MPC depending on circuit type} For the linear regression use-case
we think that $128$ bit field size is enough using a fixed point precision by
$16$ bits while keeping the numbers magnitude to $40$ bits (thus making the
integral part of size $24$ bits). We should keep in mind that although triple
generation for large fields is usually faster than rings there are benefits
when working with ring sharings - such as faster methods for secret
truncation which in turn makes RELU's faster.

For now we are only interested to solve the secret matrix-matrix
multiplication. We might want to add some gaussian noise over the output to
avoid one party getting the input matrix of the other party. For this large
field operations should suffice, however we list the current state of the art
protocols for obtaining MPC with different arithmetic structure. For
readability purposes we abbreviate honest majority with HM:

\begin{enumerate}
  \item HM over large fields $\Fp$ \cite{CCS:LinNof17,C:CGHIKL18}. Although the
  two constructions above are for the malicious setting these are constructing
  assuming $\Frand, \Fmult$ hybrid model where $\Fmult$ needs to be a multiplication
  protocol secure up to additive attacks. To get such an $\Fmult$ we can simply take
  a semi-honest protocol for HM for doing multiplications such as \cite{CCS:AFLNO16}.
  \item HM for binary circuits $\F_2$. Note that Araki et al. \cite{CCS:AFLNO16}
  works for binary circuits as well for the semi-honest case.
  To get malicious security Chida et al. \cite{C:CGHIKL18} present a protocol
  for small field multiplications using cut-and-choose techniques The cut and
  choose framework works by generating large batches of triples
  optimistically and then open a fraction to detect whether some cheating
  occurred.
  \item HM over small fields (eg. $\F_{2^8}$ which is suitable for AES). To
  get malicious security, depending on the extension field size, we can
  either repeat the computation $\sec/|\F|$ times or generate triples using cut and choose
  from \cite{C:CGHIKL18}.
  \item HM over rings $\mathbb{Z}_{2^k}$. Abspoel et al. \cite{EPRINT:ADEN19}
  are the first ones who introduce vector dot-products of secret shares at a cost of
  communicating a constant number of ring elements independently of the vector sizes
  for the ring case - however they avoid giving any hint on how to realise
  reactive computations, this is solved later in \cite{cryptoeprint:2020:1330}.
  \item HM by mixing circuits: \cite{CCS:MohRin18,cryptoeprint:2020:1330}.
  Escudero et al. \cite{C:EGKRS20} work in the more generic setting but they
  have
  improvements for the honest majority as well. They improve the mixed framework
  specifically for $\Z_{2^k}$ and $\F_2$ very recently in \cite{cryptoeprint:2020:1330}.

\end{enumerate}

Small note that almost everything can be built in the malicious setting using
building blocks from the semi-honest protocols with the exception of binary triples
$c = a \cdot b$ where $a,b \in \F_2$. For the binary triple case we need more
complex techniques such as cut and choose.


\subsection{Small note.}
If most of the things here seem to you like a word salad of many unknown
abbreviation I strongly recommend reading the first sections
\cite{evans2017pragmatic} and more important Chapter 7 about honest majority MPC.

\section{Reasons why we should choose Dalskov, Escudero and Keller
\cite{cryptoeprint:2020:1330}}

Main advantage is that performs natively on $\Z_{2^k}$ and $\F_p$.
Pros:
\begin{enumerate}
  \item Truncations are much cheaper in a ring and they have developed some
  cool techniques on making those even faster using some share conversions.
  \item Computations are more suitable to classical processor instructions.
  \item Passive security protocol is the same as Araki et al.
  \item Code is already available in MP-SPDZ git.
\end{enumerate}

Cons:
\begin{enumerate}
  \item I haven't checked the security proofs yet...
  \item Need to check how exactly the security parameter $s$ affects the
  register size
\end{enumerate}

\subsection{Secret dot-products using MPC}

The goal is to compute $\sum_{k=1}^m \share{x_k} \cdot \share{y_k}$.

\subsubsection{3-party semi-honest (HM)}

Although this was written with $x$ and $y$ being a single ring element,
i.e. $x,y \in \Z_{2^k}$ the same protocols works when $x, y \in \Z^m_{2^k}$
and the ring multiplication operator `$\cdot$` is replaced with the dot
product. In this manner $x \cdot y = \sum_{i=1}^m x_i y_i \in \Z_{2^k}$ while
$x, y \in \Z^m_{2^k}$. This can be extended for arbitrary length tensors as long as
$\alpha, \beta, \gamma$ have the correct dimensions to mask randomize $x \cdot y$.

\subsubsection{3-party malicious (HM)}
Note that the first progress
for this problem was done for the field case by Chida et al. \cite{C:CGHIKL18}.
For the ring case see \cite{EPRINT:ADEN19,cryptoeprint:2020:1330}.

\subsubsection{Dishonest majority}

Here parties fetch a triple for each $\share{x_k} \cdot \share{y_k}$ multiplication
ending with a cost of $m$ preprocessed triples. This is valid for secret sharing (SPDZ-type
\cite{EC:KelPasRot18,C:DPSZ12,C:CDESX18})
but also for garbled circuits (\cite{AC:HazSchSor17,CCS:WanRanKat17b}) type protocols.



\section{Truncation}

Given $\share{a}$ compute $\share{a \bmod 2^m}$ where $m$ is a public constant.
There are many ways to truncate a secret which can be roughly classified into 2:
\begin{itemize}
  \item probabilistic truncation - correct result with probability $p$.
  \item deterministic truncation - always output the exact truncation.
\end{itemize}

\noindent
The state of the art techniques in truncating a secret are using
dabits/edabits as described by Escudero et al. in \cite{C:EGKRS20} and
various share conversions. In \cite{C:EGKRS20} there are several techniques
to achieve both probabilistic and deterministic truncation for field and ring
case.

If we want to avoid edabit generation required for generating random bounded integers
by the truncation protocols in \cite{C:EGKRS20} then we can use the protocols from
\cite{PoPETS:DalEscKel20}. We describe their probabilistic protocols for replicated secret
sharing below in Fig~\ref{fig:truncpr-rep-ss}.

\begin{Boxfig}{Truncation protocol for semi-honest
RSS.}{fig:truncpr-rep-ss}{Protocol $\Proto{\mathsf{TruncPrSp}}$}
On input
$\mathsf{TruncPrSp}(\share{x}, m)$ from all parties where $x$ is a sharing
over $\Z_{2^k}$: \\
$P_3$ does the following:
  \begin{enumerate}
    \item Sample locally random bits $r_i$ for $i \in [0, k-1]$.
    \item Generate additive shares ($2$-out-of-$2$) for $r = \sum_{i=0}^{k-1}r_i$,
    $r_{k-1}$ and $\sum_{i=m}^{k-2} r_i \cdot 2^{i-m}$. Send the corresponding shares
    to $P_1$ and $P_2$.
    \item Generate random numbers $y_1, y_3 \in \Z_{2^k}$. Send $y_1$ to
    $P_1$ and $y_3$ to $P_2$.
    \item Set own share of the truncation result as $(y_3, y_1)$.
 \end{enumerate}
$P_1$ together with $P_2$ do the following:

\begin{enumerate}
   \item Convert $\share{x}$ into a $2$-out-of-$2$ sharing. This is
   accomplished by $P_1$ setting its local share to $x_1 + x_2$ while $P_2$
   setting its local share as $x_3$.
   \item $P_1$ and $P_2$ then execute the protocol $\Proto{\mathsf{TruncPR}}$
   using the $2$-out-of-$2$ sharings of $x$ and the shares provided by $P_3$.
\end{enumerate}
 Now parties convert the $2$-out-of-$2$ sharing output of
 $\mathsf{TruncPR}$ to a $2$-out-of-$3$
 sharing using the randomness provided by $P_3$.
 \begin{enumerate}
   \item Each $P_i$ for $i \in [1, 2]$ sets the output of $\mathsf{TruncPr}$
   as $y'_i$ and $\hat{y}_i$ as the sharing received from $P_3$. Send $y'_i -
   \hat{y}_i$ to $P_{3-i}$. The received values are denoted by $\tilde{y}_i$.
   \item $P_1$ sets the output share as $(y_1, y'_1 - \hat{y}_1 + \tilde{y}_1)$
   \item $P_2$ sets the output share as $(y'_2 - \hat{y}_2 + \widehat{y}_2, y_3)$.
 \end{enumerate}
\end{Boxfig}

\begin{Boxfig}{Generic truncation protocol.}{fig:truncpr-generic}
  {Protocol $\Proto{\mathsf{TruncPr}}(\share{x}, m)$}
  Assume $x \in \Z_{2^k}$ has $\mathsf{MSB}(x) = 0$. If $x$ is negative than
  we make it positive by adding $2^{k-1}$ to it and assume that $\mathsf{abs}(x) < 2^{k-1}$.
  \\
  On input $\mathsf{TruncPr}(\share{x}, m)$ from all parties:
  \begin{enumerate}
    \item Sample $k$ random bits $\share{r_i}$ and compute $r = \sum_{i=1}^k
    r_i$. \item Open $c \asn \share{x} + \share{r}$, truncate $c$ publicly
    and
    assign it to $c' \asn (c/2^m) \bmod 2^{k-m-1}$.
    \item Compute $\share{b} \asn \share{r_{k-1}} \oplus (c / 2^{k-1})$.
    \item Output $c' - \sum_{i=m}^{k-2} \share{r_i} \cdot 2^{i-m} + \share{b}
    \cdot 2^{k-m-1}$.
  \end{enumerate}

\end{Boxfig}


Note that although the protocol in Fig~\ref{fig:truncpr-rep-ss} looks
very efficient, the edabit variant from \cite{C:EGKRS20} requires
about $\approx 35$ times less communication.

\section{Fixed point division}
TBD. SecureNN has some leakage, check MP-SPDZ - there was no description in
ABY3 on fixed point division.

\subsection{Comparison}

%In order to compute the MAPE (mean absolute percentage error) metric which boils down to compute the absolute value of a secret $\share{|\vx|}$. Since our primitives work with ring arithmetic one of its most efficient evaluation of the absolute value is $\share{|\vx|} = \share{\vx} \cdot \share{\vx > 0}$.

The problem of computing secure comparisons translates directly into computing a sharing of the most significant bit $\share{\mathsf{msb}(\vx)}$.
In the 3PC semi-honest model there are few approaches to this:
\begin{enumerate}
   \item ABY3 \cite{CCS:MohRin18} The main idea is for
   each party to locally bit-decompose their shares over $\Z_{2^k}$ and then
   reconstruct the secret modulo $\Z_{2^k}$ using shares from $\Z^k_2$ and a
   binary adder. Once a boolean sharing of the MSB is computed using the binary adder
   this is converted to a ring sharing in $\Z_{2^k}$.
   In ABY3 this was done using a three-party OT protocol. In MP-SPDZ \cite{CCS:Keller20} 
   the boolean to ring sharing conversion was achieved using a daBit.
   \item SecureNN \cite{PoPETS:WagGupCha19} 
   similar to ABY3 with the downside that it uses arithmetic modulo
   some small fields but avoids ring-to-boolean conversion.
  \item Comparisons using edaBits \cite{C:EGKRS20}. The 3PC case has roughly the same cost as ABY3.
\end{enumerate}

We use the MSB extraction protocol from ABY3 with a Kogge-Stone binary adder and minor optimizations for tensor operations. We avoid the special three-party OT that ABY3 had along with edaBits preprocessing by introducing our custom protocol \textsf{B2A} for binary to ring share conversion in Figure~\ref{fig:b2a-protocol}.
The MSB protocol can be found in Figure~\ref{fig:msb-protocol} where $\overline{\cdot}$ is used to denote a vector with $k$ elements indexed using $(i)$ for $i \in [k]$.




\begin{Boxfig}{Ring bit decomposition to binary shares}{fig:bitdecraw}
  {$\mathsf{BitDecRaw}_{[P_1, P_2, P_3]} \left( \share{\vx}^\mathsf{rep}_{2^k} \right) \rightarrow [\share{\cdot}^\mathsf{rep}_2; k]$}
  \begin{enumerate}
  \item Let $\left( (\vx_1^1, \vx_2^1), (\vx_2^2, \vx_3^2), (\vx_3^3, \vx_1^3) \right) = \share{\vx}^{\mathsf{rep}}_{2^k}$.

  \item On $P_1$:
  \begin{enumerate}
    \item $\overline{\va} \asn \textsf{LocalBitDec}(\vx^1_1 + \vx^1_2)$.
    \item $\overline{\share{\va}_2}(i) \asn \mathsf{ReplicatedShare}(\Z_2; \overline{\va}(i))$ for $i \in [k]$.
  \end{enumerate}

  \item On $P_2$:
  \begin{enumerate}
      \item $\overline{\vb^2} \asn \textsf{LocalBitDec}(\vx_3^2)$.
  \end{enumerate}

  \item On $P_3$:
  \begin{enumerate}
      \item $\overline{\vb^3} \asn \textsf{LocalBitDec}(\vx_3^3)$.
  \end{enumerate}
  \item Let $\overline{\share{\vb}_2}(i) = \left( (0, 0), (0, \overline{\vb^2}(i)), (\overline{\vb^3}(i), 0) \right)$ for $i \in [k]$.
  \item Return $\textsf{BinaryAdder}(\overline{\share{\va}_2}, \overline{\share{\vb}_2})$.
\end{enumerate}

\end{Boxfig}


\begin{Boxfig}{MSB computation from a replicated ring share}{fig:msb-protocol}
  {$\mathsf{MSB}_{[P_1, P_2, P_3]} \left( \share{\vx}^\mathsf{rep}_{2^k} \right)$}
  \begin{enumerate}
  \item $\overline{\share{\vb}} \asn \mathsf{BitDecRaw}(\share{\vx})$.
  \item Return $\textsf{B2A}(\overline{\share{\vb}_2}(k-1))$.
  \end{enumerate}

\end{Boxfig}



\subsubsection{Improved ABY3 boolean to ring sharing protocol}
When the inputs are tensors we can perform the share conversion $\share{\cdot}_2
\mapsto \share{\cdot}_{2^k}$ ($\mathsf{B2A}$ function)
more efficient by making use of the fact that all parties
follow the protocol specifications. The starting point is using a similar idea
from daBit/edaBit \cite{INDOCRYPT:RotWoo19,C:EGKRS20}
line of work with the twist that $P_3$ generates the preprocessing
material. We fully describe this share conversion protocol in Figure~\ref{fig:b2a-protocol}.

Acting as a trusted third party, $P_3$ generates a random
daBit $(\share{\vb}_2, \share{\vb}_{2^k})$ locally and shares it to $P_1$ and $P_2$. Then $P_1$ and $P_2$ run a two-party protocol to convert $\share{\vx}_2$ to
$\share{\vx}_{2^k}$ by computing $\vc \asn \vx_2 \oplus \vb_2$ and then locally XOR-ing in the arithmetic domain $\vx_{2^k} = \vc + \vb_{2^k} - 2 \cdot \vc \cdot \vb_{2^k}$.
Finally they convert back to a replicated sharing using $\mathsf{AdditiveToReplicated}$ from Figure~\ref{fig:two-to-three}.

\begin{Boxfig}{Binary to arithmetic share conversion computation.}{fig:b2a-protocol}
  {$\mathsf{B2A}_{[P_1, P_2, P_3]} \left( \shareB{\vx}^\mathsf{rep} \right)$}
  \begin{enumerate}
  \item Let $\left( (\vx_1^1, \vx_2^1), (\vx_2^2, \vx_3^2), (\vx_3^3, \vx_1^3) \right) = \shareB{\vx}^\mathsf{rep}$.
  \item On $P_1$:
  \begin{enumerate}
  \item $\vx^1_{12} \asn \vx_1^1 \oplus \vx_2^1$.
  \item $\shape \asn \mathsf{Shape}(\vx^1_1)$.
  \end{enumerate}
  \item Let $\shareB{\vx}^\mathsf{add} = (\vx^1_{12}, \vx_3^2)$.
  \item $\shareB{\vb}^\mathsf{add}, \shareR{\vb}^\mathsf{add} \asn \textsf{DaBit}_{[D=P_3, P_1, P_2]}(\shape)$.
  \item $\shareB{\vc}^{\mathsf{add}} \asn \shareB{\vb}^\mathsf{add} \oplus \shareB{\vx}^{\mathsf{add}}$.
  \item $\vc \asn \Open_{[P_1, P_2]}(\shareB{\vc}^{\mathsf{add}})$.
  \item $\shareR{\vx}^\mathsf{add} \asn \vc + \shareR{\vb}^{\mathsf{add}} - 2 \cdot \vc \cdot \shareR{\vb}^{\mathsf{add}}$.
  \item Return $\textsf{AdditiveToReplicated}_{[P_1,P_2,P_3]}(\shareR{\vx}^\mathsf{add})$.
\end{enumerate}

\end{Boxfig}


\begin{Boxfig}{DaBit protocol generation}{fig:dabit-protocol}
  {Protocol $\Proto{DaBit[D, P_1, P_2]}(\shape)$}
  Roles: $P_1$, $P_2, P_3$ where $P_3$ acts as a dealer to distribute
  an additive share of a bit of shape $\shape$ in the boolean and arithmetic domain.
  \begin{enumerate}
  \item With Host(D):
  \begin{enumerate}
    \item $\seed \asn \derives(\dkey_D, \GenNonce)$.
    \item $\vb \asn \RingSample(\seed, \shape, \{0, 1\})$.
    \item $\seed(\textsf{bin}) \asn \derives(\dkey_D, \GenNonce), \seed(\textsf{ring}) \asn \derives(\dkey_D, \GenNonce)$.
    \item Send $\seed(\textsf{bin})$ and $\seed(\textsf{ring})$ to $P_1$.

  \end{enumerate}
  \item  With Host($P_1$):
 \begin{enumerate}
     \item Receive $\seed(\textsf{bin})$ and $\seed(\textsf{ring})$ from Dealer.
     \item Locally compute $\vb^1_{2^k} \asn \RingSample(\seed(\textsf{ring}), \shape, \Z_{2^k})$.
     \item Locally compute $\vb^1_{2} \asn \RingSample(\seed(\textsf{bin}), \shape, \{0, 1\})$.
 \end{enumerate}
\item With Host(D):
\begin{enumerate}
    \item Sample $\vb^1_{2^k} \asn \RingSample(\seed(\textsf{ring}), \shape, \Z_{2^k})$. Set $\vb^2_{2^k} = \vb - \vb^1_{2^k}$.
    \item Sample $\vb^1_{2} \asn \RingSample(\seed(\textsf{bin}), \shape, \{0, 1\})$. Set $\vb^2_{2} = \vb - \vb^1_{2}$. 
    \item Send $\vb^2_{2^k}$ and $\vb^2_2$ to $P_2$.
\end{enumerate}
\item With Host($P_2$):
\begin{enumerate}
    \item Receive $\vb^2_{2^k}$ and $\vb^2_2$ from Dealer.
\end{enumerate}
\item Let $\shareR{\vb} = \text{Additive}(\vb^1_{2^k}, \vb^2_{2^k})$ and $\shareB{\vb} = \text{Additive}(\vb^1_2, \vb^2_2)$. Output $(\shareR{\vb}, \shareB{\vb})$.
\end{enumerate}
\end{Boxfig}


% 
\begin{Boxfig}{$(2,2)$ to $(2,3)$ share conversion protocol for semi-honest
RSS.}{fig:two-to-three}{Protocol $\Proto{(2,2) \rightarrow (2,3)}$}
Input is $\share{x}$ which is additively shared amongst $P_1$ and $P_2$. Party $P_1$ holds $x_1$ and $P_2$ holds $x_2$ 
such that $x_1 + x_2 = x$. \\
$P_3$ does the following:
  \begin{enumerate}
    \item Sample seeds $s_1$ and $s_2$ and seeds $s_i$ to $P_i$.
    \item Compute $y_1 \sample \PRG(s_1)$ and $y_3 \sample \PRG(s_2)$.
    \item Set the output share as $(y_3, y_1)$.
 \end{enumerate}

$P_1$ and $P_2$ do the following:
\begin{enumerate}
   \item Parties extend the seeds received by $P_1$ computing $\hat{y}_1 \sample \PRG(s_1)$ and $P_2$ computing
   $\hat{y}_3 \sample \PRG(s_2)$.
   \item Compute $x_i - y_i$ and send it to $P_{3-i}$. 
   The received values are denoted by $\tilde{y}_i$.
   \item $P_1$ sets the output share as $(\hat{y}_1, x_1 - \hat{y}_1  + \tilde{y}_1)$
   \item $P_2$ sets the output share as $(x_2 - \hat{y}_2 + \tilde{y}_2, \hat{y}_3)$.
 \end{enumerate}
\end{Boxfig}


\begin{Boxfig}{Absolute value computation from a replicated ring share.}{fig:abs-protocol}
  {Protocol $\Proto{abs[P_1, P_2, P_3]}(\share{\vx})$}
  Roles: Replicated, $P_1$, $P_2, P_3$ where $\vx \in \Z_{2^k}^\shape$ and
  $\vx.\mathsf{placement}$ is on Replicated($P_1, P_2, P_3$). \newline
  With Replicated($P_1, P_2, P_3$):
  \begin{enumerate}
    \item Compute $\share{\vb} = \Proto{\mathsf{msb}}(\vx)$.
    \item Let $\share{\vs} \asn 1 - 2 \cdot \share{\vb}$.
    \item Output $\share{\vs} \cdot \share{\vb}$.
  \end{enumerate}

\end{Boxfig}


\import{key-distribution}{infra.tex}

