\subsection{Fixed point arithmetic}
\label{subsec:fixed-point}

% \commentM{TODO move this up, perhaps till just after the overview}

We define $\verb|fixed|(k, f)$ as the set of rational numbers $\{x \in \Q : x
= \bar{x} \cdot 2^{-f}, \bar{x} \in \Z_{\share{k}}\}$. Here $\bar{x} \in
Z_{\share{k}}$ denotes that $\bar{x}$ is at most a $k$ bit integer. A fixed
point number is represented in memory as $x \cdot 2^{f} = \bar{x} \in
\Z_{\share{k}}$. All operations done on fixed point numbers boil down to ring
arithmetic which is described below. The lowering from fixed point operations
to ring arithmetic is described in more detail in Appendix~\ref{app:fixed-point}
along with the truncation protocol in Section~\ref{subsec:truncation}.


