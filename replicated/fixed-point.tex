\subsection{Fixed-point arithmetic}
\label{subsec:fixed-point}

% \commentM{TODO move this up, perhaps till just after the overview}

We define $\verb|fixed|(k, f)$ as a subset of the real numbers $\{x \in \mathbb{R} : x = \bar{x} \cdot 2^{-f}, \bar{x} \in \Z, -2^k < \bar{x} < 2^{k-1}\}$ where $\bar{x}$ is at most a $k$ bit signed integer. A fixed point number is represented as $x \cdot 2^{f} = \bar{x} \in \Z_{2^k}$. All operations done on fixed point numbers boil down to ring arithmetic which is described below. The lowering from fixed point operations to ring arithmetic is described in more detail in Appendix~\ref{app:fixed-point} along with the truncation protocol in Section~\ref{subsec:truncation}.
