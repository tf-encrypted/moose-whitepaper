\section{Implementation}

\subsection{AES based PRG}
\label{sec:aes}

To sample efficiently a large batch of random bits we use an AES based PRG.
To produce $n \cdot 128$ random bits the main idea is to select a random
$128$ bit key $k$ and call $\mathsf{AES}_k(0), \dots, \mathsf{AES}_k(n-1)$.
Since AES works on $128$ bit blocks the integer counters $0, \cdots, (n-1)$
are encoded using Little Endian format.  In case one wants to retrieve a
number of $\ell$ random bits which is non-divisible by $128$ then the last
$128 - \ell$ bits are discarded.

This method of generating
randomness is used in a couple of MPC libraries such as MP-SPDZ
\cite{CCS:Keller20}, SCALE-MAMBA \cite{SCALE} or Swanky \cite{url-swanky}.
Security proofs for using fixed key AES in various MPC protocols were given
in \cite{SP:GKWY20}.

The PRNG wrapper we have built can be found in \textsf{rust/src/prng.rs} and
it is based on \cite{url-cargo-aes}.  We set the \textsf{AES} crate to
compile with special SSE3 CPU instructions whenever it is possible.
The PRNG key $k$ is generated using a call to \textsf{randombytes\_into}
from the Sodium Oxide rust crate \cite{url-sodiumoxide}.

