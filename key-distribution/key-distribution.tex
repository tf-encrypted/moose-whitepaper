% \tableofcontents
\section{Simple libsodium description}

The current infrastructure we have is for all parties to generate
a pair of public-secret keys \verb|pk,sk| using libsodium
\begin{verbatim}
crypto_box_keypair(pk, sk)
\end{verbatim}

\noindent For simplicity let's assume we have two parties: Alice and Bob who
both have called the \verb|crypto_box_keypair| to own a pair of public-secret keys.
In order for Bob to send a message to Alice he encrypts a message with the following function:
\begin{verbatim}
crypto_box_easy(ciphertext, MESSAGE, MESSAGE_LEN, nonce,
                    bob_publickey, alice_secretkey)
\end{verbatim}

In the \verb|crypto_box_easy| call there are few things that happening, according to \cite{bernstein2009cryptography}
and by also looking at the code \cite{libsodium}.
First, there is an internal call to  \verb|crypto_box_curve25519xsalsa20poly1305| which
says that there is a combination of protocols such as:
\begin{enumerate}
  \item Curve25519 elliptic-curve-Diffie–Hellman key exchange to establish a shared secret.
  \item After the shared secret $k$ is established than the message \verb|MESSAGE| is encrypted using
  Salsa20 stream cipher keyed with $k$ and nonce \verb|nonce| to produce a ciphertext $c$
  \item In order to keep integrity of the ciphertext then Poly1305 MAC scheme is computed over $c$.
  The MAC scheme and the ciphertext are now concatenated and sent around as \verb|ciphertext| from the output
  of \verb|crypto_box_easy|
\end{enumerate}

\section{Key setup infrastructure}

We have three organizations into play: \verb|A|, \verb|B| and \verb|Cape|.
Each org will own a set of workers to perform computations on encrypted data
- think of workers as parties in an MPC protocol. On both sides of \verb|A|
and \verb|B| there are two different types of parties, besides the workers,
an operator and a data scientist denoted as \verb|OpA|, \verb|DS-A|,
\verb|OpB|, \verb|DS-B|.

The operators and data scientists talk to each other across organizations through a
communication service ran by \verb|Cape|.
We now describe the key steps for workers to establish authenticated secret channels
in order for the computations to be done securely:
\begin{enumerate}
  \item \verb|OpA| generates a pair of keys $\VK_A, \SK_A$
  using libsodium \verb|crypto_sign_keypair|.
  \item The secret key $\SK_A$ is broadcasted internally to the workers
  inside the org to be able to sign their ephemeral keys across sessions.
  Finally, a hash of the public verification key $\VK_A$
  denoted as $h_A^\OP = \HASH(\VK_A)$
  is then sent to \verb|B-DS| by the operator through Cape.
  \item Optionally, \verb|B-DS| can check via an authenticated channel (say \verb|mail.google.com|
  or some QR code) that the hash $h_A^\OP$ received from Cape corresponds to the one sent by \verb|OpA|.

  \item On the \verb|B| side the operator \verb|OpB| generates a symmetric key $K_B$
  using \verb|crypto_auth_keygen| in libsodium.
  Then $K_B$ is broadcasted to all parties on the \verb|B| side (workers +
  DS). The DS-B fetches $h_A^\OP$ from Cape and sends $\tau_{h_A}^\DSB \asn
  \MAC(K_B, h_A^\OP)$ along with $h_A^\DSB \asn h_A^\OP$ to all workers on the B side
  through Cape service. This MAC is used by workers on the B side to check
  that the public key setup messages between workers across organizations are
  forwarded correctly by Cape.

  \item On the \verb|A| side all workers $W_A^i$ generate $\EK_A^i, \DK_A^i$
  using \verb|crypto_box_keypair|.

  \item For the key-setup between workers, on the \verb|A| side each worker
  sends $\EK_A^i, \VK_A, \verb|datetime|, \sigma_i \asn \SIGN(\SK_A, \EK_A^i
  || \verb|datetime|)$ to Cape which then forwards this to every worker on
  the B-side. On the B-side the workers receive the following: $M_\VK,
  \EK_A^i, \VK_A, \verb|datetime|, \sigma^A_i$.
  They continue iff the following equations hold:

  \begin{enumerate}
    \item Checks if $\HASH(\VK_A) = h_A^\DSB$.
    \item $\VER(K_B, \tau_{h_A^\DSB}) = 1$. This ensures that the workers
    received the correct $\HASH(\VK_A)$ sent by \verb|B-DS| through Cape.

    \item Finally they check the signature $\VER(\VK_A, \sigma^i_A) = 1$.
    with the $\VK_A$ received from the worker through Cape.
    \item If all checks pass then keep $\EK_A^i$ as the public key for worker $W_A^i$.
    Note that \verb|datetime| is used to prevent replay attacks, this way the workers
    register the other PKs iff \verb|datetime| is within some specific time frame.
  \end{enumerate}

  \item Repeat this for parties on the $B$ side (see Figure~\ref{fig:b-flow}).

  \item Now that after all workers established the public key infrastructure
  using the description above they continue by calling \verb|crypto_box_easy|
  which launches the Curve25519 key exchange and communicate further using a
  symmetric key as described in the first section. For more details on the libsodium API
  the reader can check Figure~\ref{fig:worker-flow}. Note that communication
  between workers happens through an entity called \verb|Broker| who is
  simply forwarding messages between workers.

% Expiration date for signatures
\end{enumerate}


\begin{sidewaysfigure}
\begin{table}[H]
\centering
\begin{adjustbox}{width=\textwidth}
\begin{tabular}{c c c c c c c c c c c c c}
\text{\textsf{DS-A}} & & \text{\textsf{OpA}} & & \text{\textsf{WorkerA}} & & \text{\textsf{Cape}} & &
\text{\textsf{WorkerB}} & & \text{\textsf{OpB}} & & \text{\textsf{DS-B}} \\

& & $\VK_A, \SK_A \asn$ & & & & &&&& $K_B \asn$ & \\
& & \verb|crypto_sign_keypair| & & & &&&& & \verb|crypto_auth_keygen| & \\
& & $h_A^{\OP} \asn \HASH(\VK_A)$ &  &&&&&&  &   & \\

&$\xleftarrow{h_A^{\OP}}$ & & $\xrightarrow{ \SK_A, \VK_A }$ &&&&&&
$\xleftarrow{K_B}$ & & $\xrightarrow{K_B}$ \\

& & \tikzmark{a}{0} & & && \tikzmark{b}{0} \link{a}{b}{$h_A^{\OP}$}
\tikzmark{a1}{0} & & & & & & \tikzmark{b1}{0} \link{a1}{b1}{$h_A^{\OP}$}\\

set $h_A^{DS} \asn h_A^{\OP}$ & \\
\tikzmark{a3}{0} &&&&&&&&&&&& \tikzmark{b3}{0} \link{a3}{b3}{(optional): via QR code $h_A^{DS}$} \\

&&&&&&&&&&&& $h_A^{DS} \stackrel{?}{=} h_A^{\OP}$ \\
&&&&&&&&&&&& abort o/wise \\



&&&&&&&&&&&& Set $h_A^\DSB\asn h_A^\OP$\\
&&&&&&&&&&&& Set $\tau_{h_A^\DSB} \asn$ \\
&&&&&&&&&&&& $\MAC(K_B, h_A^\OP)$ \\

&&&&&&&\tikzmark{a2}{0} & &&&& \tikzmark{b2}{0} \link{b2}{a2}{$h_A^\DSB, \tau_{h_A^\DSB}$} \\

% Worker side
&&&& $\EK_A^i, \DK_A^i \asn$ & \\
&&&& \verb|crypto_box_keypair| & \\

&&&& $\sigma^i_A = \SIGN(\SK_A, \EK_A^i || \verb|datetime| )$ & \\
& & \\
&&&&&
$\xrightarrow{\EK_A^i, \VK_A, \sigma^i_A, \texttt{datetime}}$ & &
$\xrightarrow{\EK_A^i, \VK_A, \sigma^i_A, \texttt{datetime}, \tau_{h_A^\DSB}}$ &

& & \\

&&&&&&&& $\HASH(\VK_A) \stackrel{?}{=} h_A^\DSB$ \\
&&&&&&&& $\VER(K_B, \tau_{h_A^\DSB}) \stackrel{?}{=} 1$ \\
&&&&&&&& $\VER(\VK_A, \sigma^i_A) \stackrel{?}{=} 1$ \\
&&&&&&&& Set $\EK_A^i$ as pk for $W_A^i$ if all checks pass \\
&&&&&&&& Reject o/wise \\

% & \xleftarrow{\hspace{1em}c\hspace{1em}} & \\
\end{tabular}
\end{adjustbox}
\end{table}
\caption{PK setup for Organization A}
\end{sidewaysfigure}


\begin{sidewaysfigure}
\begin{table}[H]
\centering
\begin{adjustbox}{width=\textwidth}
\begin{tabular}{c c c c c c c c c c c c c}
\text{\textsf{DS-A}} & & \text{\textsf{OpA}} & & \text{\textsf{WorkerA}} & & \text{\textsf{Cape}} & &
\text{\textsf{WorkerB}} & & \text{\textsf{OpB}} & & \text{\textsf{DS-B}} \\

&&$K_A \asn$ &&&&&&&& $\VK_B, \SK_B \asn$ \\
&&\verb|crypto_auth_keygen| &&&&&&&& \verb|crypto_sign_keypair| \\
&&           &&&&&&&& $h_B^{\OP} \asn \HASH(\VK_B)$ \\

&$\xleftarrow{K_A}$ && $\xrightarrow{K_A}$ & &&&&& $\xleftarrow{\SK_B,
\VK_B}$ & & $\xrightarrow{ h_B^\OP }$ &\\

\tikzmark{a}{0} &&&&& \tikzmark{b}{0} & \link{b}{a}{$h_B^\OP$} &
\tikzmark{a1}{0} & & & \tikzmark{b1}{0} \link{b1}{a1}{$h_B^\OP$}\\

&&&&&&&&&&&& set $h_B^{DS} \asn h_B^{\OP}$ \\
\tikzmark{a3}{0} &&&&&&&&&&& \tikzmark{b3}{0} \link{b3}{a3}{(optional): via QR code $h_B^{DS}$}
 \\

$h_B^{DS} \stackrel{?}{=} \HASH(\VK_B)$ \\
abort o/wise \\
& \\

Set $h_B^\DSA \asn h_B^\OP$\\
Set $\tau_{h_B^\DSA} \asn$ \\
$\MAC(K_A, h_B^\OP)$ \\

\tikzmark{a2}{0} &&&&&& \tikzmark{b2}{0} \link{a2}{b2}{$h_B^\DSA, \tau_{h_B^\DSA}$} \\

% Worker side
&&&&&&&& $\EK_B^i, \DK_B^i \asn$ \\
&&&&&&&& \verb|crypto_box_keypair| \\

&&&&&&&& $\sigma^i_B = \SIGN(\SK_B, \EK_B^i || \texttt{datetime})$ & \\
&&&&& $\xleftarrow{\EK_B^i, \VK_B, \sigma^i_B, \texttt{datetime}}$ & &
 $\xleftarrow{\EK_B^i, \VK_A, \sigma^i_B, \texttt{datetime}, \tau_{h_B^\DSA}}$ & & \\


&&&& $\HASH(\VK_B) \stackrel{?}{=} h_B^\DSA$ \\
&&&& $\VER(K_A, \tau_{h_B^\DSA}) \stackrel{?}{=} 1$ \\
&&&& $\VER(\VK_B, \sigma^i_B) \stackrel{?}{=} 1$ \\
&&&& Set $\EK_B^i$ as pk for $W_B^i$ if all checks pass \\
&&&& Reject o/wise \\


% & \xleftarrow{\hspace{1em}c\hspace{1em}} & \\
\end{tabular}
\end{adjustbox}
\end{table}
\caption{PK setup for Organization B}
\label{fig:b-flow}
\end{sidewaysfigure}


\clearpage



