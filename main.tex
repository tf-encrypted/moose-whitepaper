\documentclass{article}

\usepackage{fullpage}
\usepackage{amsmath}
\usepackage{amssymb}
\usepackage{fancyhdr}

\usepackage[colorlinks,linkcolor=dblue,filecolor=black,citecolor=dblue,urlcolor=dblue]{hyperref}
\usepackage{etex}
\bibliographystyle{alpha}
\usepackage[english]{babel}
\usepackage{amsfonts,amssymb,amsmath,sectsty,url}
\usepackage{mathrsfs}

\usepackage{tikz}
\usepackage{color}
%\usepackage{mathtools}
\usepackage[all]{xy}
\usepackage{ctable}
% \usetikzlibrary{calc}
\usepackage{tabularx,multirow}
\usepackage{graphicx}
\usepackage{ifpdf}
\usepackage{float}
\usepackage{multirow}
\usepackage{multicol,color}
\usepackage{tablefootnote}
\usepackage{pgfplots}

\usepackage{stmaryrd}
\usepackage{booktabs}
\usepackage{multirow}
\usepackage{import}
\usepackage{rotating}

% For tables
\usepackage{adjustbox}
\usepackage{booktabs}
\usepackage{multirow}
\usepackage{tabularx}

% For better precision math computations and formatting in tables
\usepackage{tikz,fp}
\usepackage[nottoc]{tocbibind}

\usepackage{listings}% http://ctan.org/pkg/listings
\lstset{
  basicstyle=\ttfamily,
  mathescape
}

% COMMANDS

\renewcommand{\emptyset}{\varnothing}

% Comments
\newcommand{\commentD}[1]{\textcolor[rgb]{0,1,0}{{\sf Drago\c{s}:} {\sl{#1}}}}
\newcommand{\commentM}[1]{\textcolor[rgb]{0,0.75,1}{{\sf Morten:} {\sl{#1}}}}
\newcommand{\commentJ}[1]{\textcolor[rgb]{1,0,0}{{\sf Jason:} {\sl{#1}}}}
% Colors

\definecolor{dblue}{rgb}{0.00, 0.50, 0.90}
\definecolor{lblue}{rgb}{0.70, 0.80, 1.00}
\definecolor{lpink}{rgb}{0.90, 0.70, 1.00}
\definecolor{lgreen}{rgb}{0.80, 0.95, 0.75}
\definecolor{lred}{rgb}{0.99, 0.50, 0.55}
\definecolor{lyellow}{rgb}{1.00, 0.95, 0.75}
\definecolor{llgrey}{rgb}{0.95, 0.95, 0.95}
\definecolor{salmon}{rgb}{0.99, 0.90, 0.90}

% Dataflow stuff
\newcommand{\shape}{\mathsf{shape}}

% Linear regression stuff
\newcommand{}[]{}

% UC security things

\newcommand{\Abort}{\mathsf{Abort}}
\newcommand{\Checkk}{\mathsf{Check}}
\newcommand{\Commit}{\mathsf{Commit}}
\newcommand{\Fcomm}{\mathcal F_\Commit}
\newcommand{\Fail}{\mathsf{Fail}}
\newcommand{\Keys}{\mathsf{Keys}}
\newcommand{\MACCheck}{\mathsf{MACCheck}}
\newcommand{\Open}{\mathsf{Open}}
\newcommand{\OK}{\mathsf{OK}}
\newcommand{\Pcomm}{\Pi_\Commit}
\newcommand{\Scomm}{\mathcal S_\Commit}
\newcommand{\Triple}{\mathsf{Triple}}
\newcommand{\Val}{\mathsf{Val}}

\newcommand{\VK}{\mathsf{vk}}
\newcommand{\DSB}{\mathsf{DSB}}
\newcommand{\DSA}{\mathsf{DSA}}
\newcommand{\OP}{\mathsf{op}}
\newcommand{\EK}{\mathsf{ek}}
\newcommand{\DK}{\mathsf{dk}}
\newcommand{\SK}{\mathsf{sk}}
\newcommand{\HASH}{\mathsf{hash}}
\newcommand{\SIGN}{\mathsf{Sign}}
\newcommand{\VER}{\mathsf{Ver}}
\newcommand{\TAG}{\mathsf{Tag}}

% fixed point arithmetic symbols
\newcommand{\Decode}{\mathsf{Decode}}


\newcommand{\inc}[1]{\mathsf{inc}({#1})}

% MPC things
\newcommand{\Pa}{\mathcal{P}}
\newcommand{\Engine}{\mathsf{Eng}}
\newcommand{\Ext}{\mathsf{Ext}}
\newcommand{\authshare}[1]{\llbracket {#1} \rrbracket} % a finite field
\newcommand{\auth}[1]{\authshare{#1}} % a finite field

% ZK things
\newcommand{\prov}{\mathcal{P}}
\newcommand{\ver}{\mathcal{V}}

% CS things
\newcommand{\Oh}{ \mathcal{O} }
\newcommand{\QuasiOh}{ \widetilde{ \mathcal{O} } }

\newcommand{\B}         {\mathrm{B}}
\newcommand{\KB}        {\mathrm{KB}}
\newcommand{\MB}        {\mathrm{MB}}
\newcommand{\GB}        {\mathrm{GB}}

\newcommand{\ms}        {\mathrm{ms}}
\newcommand{\mus}       {\mathrm{{\mu}s}}



% Sets, fields, rings, groups, etc...
\newcommand{\ring}{R}
\newcommand{\C}{\mathbb{C}} % complex nr
\newcommand{\Z}{\mathbb{Z}} % the integers
\newcommand{\Q}{\mathbb{Q}} % the rationals
\newcommand{\N}{\mathbb{N}} % the non-negative integers
\newcommand{\F}{\mathbb{F}} % a finite field
\newcommand{\Fp}{\mathbb{F}_p}
\newcommand{\Fn}{\mathbb{F}^n_{2}}
\newcommand{\Fq}{\mathbb{F}_q}
\newcommand{\range}[1]{[1\dots {#1} ]} % a finite field

\newcommand{\sample}{\stackrel{\small{\$}}{\leftarrow}}

\newcommand{\oid}{\mathsf{oid}}
\newcommand{\sid}{\mathsf{sid}}
\newcommand{\nonce}{\mathsf{nonce}}
\newcommand{\prfkeys}{\mathbf{k}}
\newcommand{\prfkey}{\mathsf{key}}
\newcommand{\RingSample}{\mathsf{SampleUniform}}
\newcommand{\SampleKey}{\mathsf{SampleKey}}
\newcommand{\dkey}{\textsf{dk}}
\newcommand{\GenNonce}{\textsf{GenNonce}()}

\newcommand{\M}{\mathcal{M}} % message space
\newcommand{\T}{\mathcal{T}} % tag space
\newcommand{\R}{\mathcal{R}} % randomness space

\newcommand{\tensor}{\otimes}
\newcommand{\unit}{\mathbf{u}}
\newcommand{\transpose}[1]{#1^{\intercal}}
\newcommand{\identitymatrix}{\mathbb{I}}



% Probability and DP
\newcommand{\Avg}{\textup{E}} % Average of random variables

\newcommand{\seed}{\mathsf{seed}} % Average of random variables
\newcommand{\derives}{\mathsf{DeriveSeed}} % Average of random variables


\newcommand{\D}{\mathcal{D}} % a distribution
\newcommand{\U}{\mathcal{U}} % the uniform distribution

\newcommand{\uniform}[1]            {\mathcal{U}(#1)}
\newcommand{\binomial}[1]           {\mathcal{B}(#1)}
\newcommand{\gaussian}[1]           {\mathcal{N}(#1)}
\newcommand{\dgaussian}[1]     {\mathcal{DG}(#1)}
\newcommand{\hwt}[1]     {\mathcal{HWT}(#1)}
\newcommand{\zo}[1]     {\mathcal{ZO}(#1)}
\newcommand{\laplacian}[1]          {\mathrm{Lap}(#1)}
\newcommand{\distribution}          {\mathcal{D}}
\newcommand{\leftovernoise}     {\mathcal{E}}

\newcommand{\pr}[1]                 {\Pr \left[ #1 \right]}
\newcommand{\floor}[1]                  {\lfloor#1\rfloor}
\newcommand{\ceil}[1]                   {\left\lceil#1\right\rceil}



% Labels

\newcommand{\pk}{\mathsf{pk}}
\newcommand{\sk}{\mathsf{sk}}
\newcommand{\sfT}{\mathsf{T}}

\newcommand{\Sim}{\mathcal{S}}
\newcommand{\Simu}[1]{\mathcal{S}_{\mathsf{#1}}}



% Operators

\newcommand{\asn}{\gets} % to be read as ``computed as'' or ``taken according to the distribution...'' or ``taken uniformly from...''

\newcommand{\ParamGen}{\mathsf{ParamGen}}
\newcommand{\KeyGen}{\mathsf{KeyGen}}
\newcommand{\Enc}{\mathsf{Enc}}
\newcommand{\Dec}{\mathsf{Dec}}
\newcommand{\Add}{\mathsf{Add}}
\newcommand{\ek}{ek}
\newcommand{\dk}{dk}
\newcommand{\plaintext}{m}
\newcommand{\ciphertext}{e}
\newcommand{\plaintextspace}{\mathcal{P}}
\newcommand{\ciphertextspace}{\mathcal{E}}
\newcommand{\INDCPA}{\textrm{IND-CPA}}

\newcommand{\Share}{\mathsf{Share}}
\newcommand{\Shamir}{\mathsf{Shamir}}
\newcommand{\RS}{\mathsf{RS}}
\newcommand{\Rec}{\mathsf{Recon}}

\newcommand{\share}[1]{\langle {#1} \rangle} % a finite field
\newcommand{\shareR}[1]{{\langle {#1} \rangle}_{2^k}} % a ring
\newcommand{\shareB}[1]{{\langle {#1} \rangle}_2} % a binary field
\newcommand{\secret}{m}

\newcommand{\MAC}{\mathsf{MAC}}
\newcommand{\Encode}{\mathsf{Encode}}

\newcommand{\PRG}{\mathsf{PRG}}

\newcommand{\Adv}{\mathsf{Adv}}
\newcommand{\Gen}{\mathsf{Gen}}



\newcommand{\ineff}{\mathsf{ineff}}
\newcommand{\superpoly}{\mathsf{superpoly}}
\newcommand{\poly}{\mathsf{poly}}
\newcommand{\negl}{\mathsf{negl}}
\newcommand{\const}{\mathsf{const}}
\newcommand{\polylog}{\mathsf{polylog}}

\newcommand{\normtwo}[1]{\|{#1}\|_2}
\newcommand{\normI}[1]{\|{#1}\|_\infty}

\newcommand{\Bclean}{B_\mathsf{clean}}
\newcommand{\Bplain}{B_\mathsf{plain}}
\newcommand{\Brand}{B_\mathsf{rand}}

\newcommand{\secp}{\mathsf{sec}} % statistical security parameter
\newcommand{\ssec}{\secp} % use this for stat sec param
\newcommand{\csec}{k} % comp sec param

\newcommand{\Func}{\mathcal{F}}
\newcommand{\Funct}[1]{\Func_\mathsf{#1}}
\newcommand{\Proto}[1]{\Pi_\mathsf{#1}}
\newcommand{\nPi}{\mathrm{\Pi}}
\newcommand{\nPhi}{\mathrm{\Phi}}
\newcommand{\Bra}[1]{\llbracket #1 \rrbracket}
\newcommand{\Fauth}{\Func_{\Bra{\cdot}}}
\newcommand{\Pauth}{\Proto{\Bra{\cdot}}}
\newcommand{\Sauth}{\Sim_{\Bra{\cdot}}}
\newcommand{\PMACCheck}{\Proto{\MACCheck}}
\newcommand{\Ftriple}{\Funct{\Triple}}
\newcommand{\Ptriple}{\Proto{\Triple}}
\newcommand{\Striple}{\Sim_{\Triple}}
\newcommand{\Fmult}{\Funct{Mult}}
\newcommand{\Pmult}{\Proto{2Mult}}
\newcommand{\Smult}{\Sim_\mathsf{2Mult}}
\newcommand{\Ffmult}{\Funct{2FMult}}
\newcommand{\Pfmult}{\Proto{2FMult}}
\newcommand{\Sfmult}{\Sim_\mathsf{2FMult}}
\newcommand{\Frand}{\Funct{Rand}}
\newcommand{\Fpok}{\Funct{ZKPoK}^S}
\newcommand{\Fgpok}{\Funct{gZKPoK}^S}
\newcommand{\Pgpok}{\Proto{gZKPoK}^S}
\newcommand{\Sgpok}{\Simu{gZKPoK}^S}
\newcommand{\Gcpa}{\mathcal{G}_\mathsf{cpa+}}

\newcommand{\vl}{\mathbf{l}}
\newcommand{\vL}{\mathbf{L}}
\newcommand{\vz}{\mathbf{z}}
\newcommand{\vx}{\mathbf{x}}
\newcommand{\vm}{\mathbf{m}}
\newcommand{\vM}{\mathbf{M}}
\newcommand{\vy}{\mathbf{y}}
\newcommand{\vu}{\mathbf{u}}
\newcommand{\vk}{\mathbf{k}}
\newcommand{\vq}{\mathbf{q}}
\newcommand{\vt}{\mathbf{t}}
\newcommand{\vs}{\mathbf{s}}
\newcommand{\vr}{\mathbf{r}}
\newcommand{\vv}{\mathbf{v}}
\newcommand{\va}{\mathbf{a}}
\newcommand{\vh}{\mathbf{h}}
\newcommand{\vb}{\mathbf{b}}
\newcommand{\vc}{\mathbf{c}}
\newcommand{\vd}{\mathbf{d}}
\newcommand{\vdp}{\mathbf{d^{'(i)}}}
\newcommand{\ve}{\mathbf{e}}
\newcommand{\vw}{\mathbf{w}}
\newcommand{\vX}{\mathbf{X}}
\newcommand{\vC}{\mathbf{C}}
\newcommand{\vn}{\mathbf{n}}
\newcommand{\vf}{\mathbf{f}}
\newcommand{\vg}{\mathbf{g}}

\newcommand{\Input}{\mathsf{Input}}
% \newcommand{\Triple}{\mathsf{Triple}}
\newcommand{\Lincomb}{\mathsf{LinComb}}
\newcommand{\id}{\mathsf{id}}

\newcommand{\vDelta}{\boldsymbol{\Delta}}

\newcommand{\valpha}{\boldsymbol\alpha}
\newcommand{\vbeta}{\boldsymbol\beta}
\newcommand{\vchi}{\boldsymbol\chi}
\newcommand{\vdelta}{\boldsymbol\delta}
\newcommand{\vepsilon}{\boldsymbol\epsilon}
\newcommand{\veta}{\boldsymbol\eta}
\newcommand{\vgamma}{\boldsymbol\gamma}
\newcommand{\vlambda}{\boldsymbol\lambda}
\newcommand{\vmu}{\boldsymbol\mu}
\newcommand{\vnu}{\boldsymbol\nu}
\newcommand{\vphi}{\boldsymbol\phi}
\newcommand{\vpsi}{\boldsymbol\psi}
\newcommand{\vrho}{\boldsymbol\rho}
\newcommand{\vsigma}{\boldsymbol\sigma}
\newcommand{\vtau}{\boldsymbol\tau}
\newcommand{\vtheta}{\boldsymbol\theta}
\newcommand{\vxi}{\boldsymbol\xi}
\newcommand{\vzeta}{\boldsymbol\zeta}

\newcommand{\ui}{^{(i)}}
\newcommand{\upi}{^{'(i)}}
\newcommand{\uti}{^{(i){\intercal}}}

\newcommand{\uj}{^{(j)}}
\newcommand{\uij}{^{(ij)}}
\newcommand{\uji}{^{(ji)}}

\newcommand{\mask}[1]{\ifnum\ANON=0#1\fi}
\newcommand{\anon}[1]{\ifnum\ANON=1#1\fi}
\newcommand{\maskanon}[2]{\ifnum\ANON=1#1\else#2\fi}




% Environments
\newenvironment{boxfig}[2]{% {#1}{#2} = {Caption}{label}
\begin{figure}
  \newcommand{\FigCaption}{#1}
  \newcommand{\FigLabel}{#2}
  \vspace{-\medskipamount}
  \begin{center}
    \begin{small}
      \begin{tabular}{@{}|@{~~}l@{~~}|@{}}
        \hline
        \rule[-1.5ex]{0pt}{1ex}
        \begin{minipage}[b]{.96\linewidth}
          \vspace{1ex}
          \smallskip
          }{%
        \end{minipage}\\
        \hline
      \end{tabular}
    \end{small}
    \vspace{-1\bigskipamount}
    \caption{\small \FigCaption}
    \label{\FigLabel}
%  \vspace{-0.3cm}
  \end{center}
\end{figure}
}
\newenvironment{Boxfig}[3]{% {Caption}{Label}{Title}
  \begin{boxfig}{#1}{#2}
    \begin{center}
      \textbf{#3}
    \end{center}
  }{%
    \end{boxfig}
  }



% LATEX

% this crossreferences description items
\makeatletter
\let\orgdescriptionlabel\descriptionlabel
\renewcommand*{\descriptionlabel}[1]{%
  \let\orglabel\label
  \let\label\@gobble
  \phantomsection
  \edef\@currentlabel{#1}%
  %\edef\@currentlabelname{#1}%
  \let\label\orglabel
  \orgdescriptionlabel{#1}%
}
\makeatother




% a temporary counter
\newcounter{tempcount}

%%% Local Variables:
%%% mode: latex
%%% TeX-master: "main"
%%% End:

%\usepackage[color=oxygenorange]{todonotes}
%\newcommand{\crec}[1]{\todo[inline]{\textbf{crec:} #1}\xspace}

% \usepackage{draftwatermark}
% \SetWatermarkLightness{0.9}
% \SetWatermarkText{DRAFT \\ \today}
% \SetWatermarkScale{2}


\usetikzlibrary{matrix,chains,positioning,decorations.pathreplacing,arrows,calc}

\newcommand\tikzmark[2]{%
\tikz[remember picture,baseline] \node[right, outer sep=0pt, inner sep=0pt] (#1){\phantom{#2}};%
}

\newcommand\link[3]{%
\begin{tikzpicture}[remember picture, overlay, >=stealth, shift={(0,0)}]
  \draw[->] (#1) -- node[auto,] {#3} (#2);
\end{tikzpicture}%
}

% \pagestyle{fancy}
% \fancyhf{}
% \setlength{\headheight}{22pt}
% \setlength{\headsep}{0.2in}
% \renewcommand{\headrulewidth}{0.4pt}
% \renewcommand{\footrulewidth}{0.4pt}% default is 0pt
% \cfoot{\thepage}
% \lhead{\includegraphics[width=3cm]{cape_logo.png}}
% \rhead{Proprietary and Confidential}

\newcommand\todo[1]{\textcolor{red}{#1}}

%
\begin{document}
%
\newcommand{\mainsection}[1]{\newpage \section{#1}}
\newcommand{\msubsection}[1]{\newpage \subsection{#1}}
\newcommand{\msubsubsection}[1]{\subsubsection{#1}}

\title{$\mathsf{Moose}$ documentation}
% \author{Morten Dahl, Yann Dupis, Jason Mancuso, Dragos Rotaru}
% \institute{Cape Privacy}


\maketitle
\tableofcontents

\thispagestyle{fancy}


\section{Introduction}

This paper details the cryptographic techniques and protocols used by Cape Privacy's encrypted learning solution. We expect the reader to have some familiarity with secure multi-party computation, linear algebra, and ring arithmetic; as a good starting point we recommend \cite{evans2017pragmatic}. We also omit detailed explanations of machine learning and the models we support, instead referring the interested reader to~\cite{??}.

Note that this paper is currently scoped to the linear regression task outlined in section~\ref{??}. Future versions of the paper will expand upon this with additional models and protocols.


\subsection{Linear Regression}

We here focus on performing linear regression between a data provider and a model owner (or data subscriber). From a cryptographic perspective, this boils down to performing a matrix dot product on secret shared data to train the weights of the model. In addition, we also compute several metrics on secret shared data to measure the impact of the regression. Concretely, we compute the MSE (mean squared error)

In order to compute the MAPE (mean absolute percentage error) metric which boils down to compute
the absolute value of a secret $\share{|\vx|}$.

% \definecolor{codegreen}{rgb}{0,0.6,0}
% \definecolor{codegray}{rgb}{0.5,0.5,0.5}
% \definecolor{codepurple}{rgb}{0.58,0,0.82}
% \definecolor{backcolour}{rgb}{0.95,0.95,0.92}

% \lstdefinestyle{mystyle}{
%     backgroundcolor=\color{backcolour},
%     commentstyle=\color{codegreen},
%     keywordstyle=\color{magenta},
%     numberstyle=\tiny\color{codegray},
%     stringstyle=\color{codepurple},
%     basicstyle=\ttfamily\footnotesize,
%     breakatwhitespace=false,
%     breaklines=true,
%     captionpos=b,
%     keepspaces=true,
%     numbers=left,
%     numbersep=5pt,
%     showspaces=false,
%     tabsize=2
% }

% \lstset{style=mystyle}

% \begin{lstlisting}[language=Python]
% @edsl.computation
% def lin_reg():

%     with x_owner:
%         X = edsl.load("X")
%         bias = edsl.ones(edsl.slice(edsl.shape(X), begin=0, end=1))
%         reshaped_bias = edsl.expand_dims(bias, 1)
%         X_b = edsl.concatenate([reshaped_bias, X], axis=1)
%         A = edsl.inverse(edsl.dot(edsl.transpose(X_b), X_b))
%         B = edsl.dot(A, edsl.transpose(X_b))

%     with y_owner:
%         y_true = edsl.load("y")
%         totals_ss = ss_tot(y_true)

%     with replicated:
%         w = edsl.dot(B, y_true)
%         y_pred = edsl.dot(X_b, w)
%         mse_result = mse(y_pred, y_true)
%         residuals_ss = ss_res(y_pred, y_true)

%     with x_owner:
%         rsquared_result = r_squared(residuals_ss, totals_ss)
%         w = edsl.identity(w)
%         mse = edsl.identity(mse)
%         residuals_ss = edsl.identity(residuals_ss)
% \end{lstlisting}

\commentM{TODO: briefly describe the usecase and its high-level computation. }

\commentM{TODO: insert eDSL snippet}

For the linear regression use-case we believe that $128$ bit field size is
enough using a fixed point precision by $16$ bits while keeping the numbers
magnitude to $40$ bits (thus making the integral part of size $24$ bits).

\subsection{Fixed point arithmetic}
\label{subsec:fixed-point}

% \commentM{TODO move this up, perhaps till just after the overview}

We define $\verb|fixed|(k, f)$ as the set of rational numbers $\{x \in \Q : x
= \bar{x} \cdot 2^{-f}, \bar{x} \in \Z_{\share{k}}\}$. Here $\bar{x} \in
Z_{\share{k}}$ denotes that $\bar{x}$ is at most a $k$ bit integer. A fixed
point number is represented in memory as $x \cdot 2^{f} = \bar{x} \in
\Z_{\share{k}}$. All operations done on fixed point numbers boil down to ring
arithmetic which is described below. The lowering from fixed point operations
to ring arithmetic is described in more detail in Appendix~\ref{app:fixed-point}
along with the truncation protocol in Section~\ref{subsec:truncation}.




\section{Fixed point arithmetic}
\label{app:fixed-point}

\subsection{Encoding} To encode a number $x \in \mathbb{R}$ into its fixed
point representation define $\mathsf{Encode}(x, f) = \floor{x \cdot 2^f}$.
The rounding here is arbitrary, one can also use \verb|ceil| function as long
as the same rounding method is used in the decoding.

\subsection{Decoding} To decode a number from its fixed point representation
$x \in \verb|fixed|(k, f)$ output $\mathsf{Decode}(x, f) = x \cdot 2^{-f} \in \mathbb{R}$.

\subsection{Addition} To add two secret shared fixed point numbers
$\share{x}, \share{y}$ where $x,y \in \verb|fixed(k, f)|$ output
$\share{x}+\share{y}$. Note that $\mathsf{Decode}(x+y)$ will be successful
iff $(x+y)\cdot 2^f$ does not overflow $\Z_{2^k}$.

\subsection{Multiplication}

To multiply two fixed point numbers $x, y \in \verb|fixed|(k, f)$ we can
compute $z = x \cdot y \in \verb|fixed|(2 \cdot k, 2\cdot f)$. On the ring
level the operation works as $\bar{z} = x \cdot y \cdot 2 ^ {2f}$, hence the
correct decoding function becomes $\Decode(z, 2f) = x \cdot y$. Although
multiplications are pretty fast this way we can see that the numbers grow by
$2^f$ after each multiplication. To circumvent this we introduce the
\verb|Trunc| operation which takes $\bar{z} = x \cdot y \cdot 2^{2f}$ and outputs $\bar{z} / 2^{f}$
which makes $\verb|trunc|(\bar{z}, f) \in \verb|fixed|(2 \cdot k, f)$.
In the main body the \verb|trunc| operation is replaced by the probabilistic truncation $\mathsf{TruncPR}$ function.
Next we describe how these operations are done using secret shared inputs.

\noindent{\textbf{Fixed point multiplication between two secrets}.} Note that
multiplication between two fixed point numbers $\share{x}, \share{y}$ where
$x,y \in \verb|fixed(k, f)|$ gets us $z = x \cdot y \cdot 2^{2f}$. One can
now simply output $\Decode(z, 2f)$ if this is the last multiplication.
To avoid $z$ overflow in $\Z_{2^k}$ and allow
subsequent computations on $z$ call $z \asn \mathsf{trunc}(2 \cdot k, f)(z)$.

\noindent{\textbf{Multiplication by (clear) integer scalars}}.
To multiply a public scalar
$c$ with $\share{x}$ where $x \in \verb|fixed(k, f)|$ we output
$c \cdot \share{x}$. In this way when we decode the product
this becomes (over the ring)
$\Decode(c \cdot x \cdot 2^f) = c \cdot x$. Again, this operation is going to
be successful iff. the product $c \cdot x$ fits in $\Z_{2^k}$.

\noindent{\textbf{Multiplication by (clear) floating scalars}}. To multiply a
public floating point scalar $c$ with $\share{x}$ we compute $\hat{c} \asn
\Encode(c, f)$ and then set the output $\share{c \cdot x} =
\mathsf{trunc}(\hat{c} \cdot \share{x}, f)$. Note that $\hat{c} \cdot \share{x}$
can be done with local computations.

% \subsubsection{Division}
% TBD, SecureNN has some leakage, check MP-SPDZ - there was no description in
% ABY3 on fixed point division.

\subsection{Secret fixed point division by a public value}

Suppose here that $x = \bar{x} \cdot 2^f$ is a fixed point encoding of $\bar{x}$. 
Given two inputs $\share{x}$ and a public value $c$ which is known to all parties the goal is to output $\share{x /c}$.
This can be done naively by first encoding the public constant into its fixed point representation and compute the division between two fixed point numbers accordingly \cite{FC:CatSax10,SCN:CatDeH10}.

A slightly more efficient way is for all parties to compute an encoding of $1/c$
and then do a local multiplication i.e. $\share{t} \asn \mathsf{ceil}(1/c) \cdot 2^f \cdot \share{x}$.
Since $t$ evaluates to $1/c \cdot 2^f \cdot \bar{x} \cdot 2^f$ then truncating \verb|trunc|($\share{t}$) computes a fixed point encoding of $\share{x/c}$ which is want we wanted in the first place.



\section{Computational Model}

In this section we describe our computational model for protocols, which is loosely based on the data-flow paradigm~\cite{abadi16} and a simplified UC model~\cite{canetti-cohen-lindell15}. Concretely, protocols are expressed as graphs where nodes represent operations to be performed by a specific party, and edges present values ``flowing'' between operations.

The main motivation for using the data-flow paradigm is that it leads to a very natural concurrent execution model, where in the extreme we can see each node as being executed by a separate task (e.g. green thread or actor). We take full advantage of this in the runtime which is based on the asynchronous execution paradigm. The reason for basing our computational model on the UC model is that it is a well-known paradigm for ensuring security under concurrent composition.


\subsection{Sessions}

Every execution of a graph is performed under a unique session id $\sid$ used to identify values and ensure isolation when running protocols concurrently. As we shall see in more detail later, session ids are for instance used to non-interactively derive nonces and sample correlated randomness by the secret sharing schemes.

Session ids must be unique and of fixed length for security reasons, but can otherwise safely be chosen by an untrusted coordinator, for instance by sampling a random string. To satisfy the security requirements, each party maintains a list of previous session ids in which it has engaged, and refuses to re-run any computation using those ids; this prevents for instance replay and selective failure attacks. It additionally checks that session ids have the correct length.


\subsection{Sub-protocols}

We allow graphs to call sub-graphs similar to calling sub-routines. When doing so, the sub-graph is executed under a sub-session id $\sid'$ derived from $\sid$ and an activation key $\mathsf{ac\_key}$ statically related to the call site:
$$
\sid' = \mathsf{h}( \sid \| \mathsf{ac\_key} )
$$
with $\mathsf{h}$ being a secure hash function and $\|$ denoting string concatenation. For security we require $\sid$ to have fixed length $\ell$. Calling a sub-graph is done through $\mathsf{Enter}$ and $\mathsf{Exit}$ operations that are implicitly linked to $\mathsf{Input}$ and $\mathsf{Output}$ operations.


\subsection{Communication}

Transmission of values between parties is done using $\mathsf{Send}$ and
$\mathsf{Receive}$ nodes where each pair is linked together by a static
$\mathsf{rdv\_key}$ attribute. Together with the session id, this allows us to
uniquely identify all values by tagging them with $(\sid, \mathsf{rdv\_key})$
during transmission.

% In some configurations we also use this pair to derive unique nonces for
% encrypting messages (see Section~\ref{sec:infrastructure}). This is done using a
% secure hash function as $\mathsf{nonce} = \mathsf{h}(\sid \| \mathsf{rdv\_key})$
% and security again depends on $\sid$ having fixed length $\ell$.


\subsection{Inlining}

In the presentation given in this paper we make heavy use of calling sub-protocols, yet for performance reasons it may be interesting to inline sub-graphs. To do so securely, special attention must be paid to certain node attributes that control uniqueness.

As an example, the $\mathsf{rdv\_key}$ attribute of $\mathsf{Send}$ and $\mathsf{Receive}$ nodes in the graph being inlined must be updated to $\mathsf{h}(\mathsf{ac\_key} \| \mathsf{rdv\_key})$ where $\mathsf{ac\_key}$ is the activation key of the $\mathsf{Enter}$ and $\mathsf{Exit}$ nodes being replaced, and $\mathsf{rdv\_key}$ in the graph being inlined into must be updated to $\mathsf{h}(\mathsf{rdv\_key})$. Other examples are the $\mathsf{sync\_key}$ attribute of $\mathsf{DeriveSeed}$ operations and the $\mathsf{ac\_key}$ attribute of $\mathsf{Enter}$ and $\mathsf{Exit}$ operations. For this to be secure we require all $\mathsf{ac\_key}$, $\mathsf{rdv\_key}$, and $\mathsf{sync\_key}$ to be of fixed length $\ell$.

\import{replicated}{replicated.tex}
\section{Implementation}

\subsection{AES based PRG}
\label{sec:aes}

To sample efficiently a large batch of random bits we use an AES based PRG. 
To produce $n \cdot 128$ random bits the main idea is to select a random $128$ bit
key $k$ and call $\mathsf{AES}_k(0), \dots, \mathsf{AES}_k(n-1)$. Since AES works
on $128$ bit blocks the integer counters $0, \cdots, (n-1)$ are encoded using Little Endian format.
In case one wants to retrieve a number of $\ell$ random bits which is non-divisible by $128$ then
the last $128 - \ell$ bits are discarded.

This method of generating
randomness is used in a couple of MPC libraries such as MP-SPDZ, SCALE-MAMBA or Swanky \cite{url-swanky}.
Security proofs for using fixed key AES in various MPC protocols were given in \cite{xiao-paper}.

This is implemented in \textsf{rust/src/prng.rs} and 
In order to sample efficiently a large batch of ring elements we instantiate
the $\RingSample$ command with an AES based PRG. 
In our $\RingSample$ after a random key $k$ is generated using the sodiuomxide entropy pool
% then  $\mathsf{AES}_{k}(generate a large chunk of random bits }$

% \subsection{$128$ bitarithmetic}
% Currently we only implement 64-bit ring arithmetic ($\Z_{2^{64}}$)



\newpage
% ---- Bibliography ----
%
\bibliography{mybib,cryptobib/abbrev3,cryptobib/crypto}


\clearpage
\appendix

\section{Fixed point arithmetic}
\label{app:fixed-point}

\subsection{Encoding} To encode a number $x \in \mathbb{R}$ into its fixed
point representation define $\mathsf{Encode}(x, f) = \floor{x \cdot 2^f}$.
The rounding here is arbitrary, one can also use \verb|ceil| function as long
as the same rounding method is used in the decoding.

\subsection{Decoding} To decode a number from its fixed point representation
$x \in \verb|fixed|(k, f)$ output $\mathsf{Decode}(x, f) = x \cdot 2^{-f} \in \mathbb{R}$.

\subsection{Addition} To add two secret shared fixed point numbers
$\share{x}, \share{y}$ where $x,y \in \verb|fixed(k, f)|$ output
$\share{x}+\share{y}$. Note that $\mathsf{Decode}(x+y)$ will be successful
iff $(x+y)\cdot 2^f$ does not overflow $\Z_{2^k}$.

\subsection{Multiplication}

To multiply two fixed point numbers $x, y \in \verb|fixed|(k, f)$ we can
compute $z = x \cdot y \in \verb|fixed|(2 \cdot k, 2\cdot f)$. On the ring
level the operation works as $\bar{z} = x \cdot y \cdot 2 ^ {2f}$, hence the
correct decoding function becomes $\Decode(z, 2f) = x \cdot y$. Although
multiplications are pretty fast this way we can see that the numbers grow by
$2^f$ after each multiplication. To circumvent this we introduce the
\verb|Trunc| operation which takes $\bar{z} = x \cdot y \cdot 2^{2f}$ and outputs $\bar{z} / 2^{f}$
which makes $\verb|trunc|(\bar{z}, f) \in \verb|fixed|(2 \cdot k, f)$.
In the main body the \verb|trunc| operation is replaced by the probabilistic truncation $\mathsf{TruncPR}$ function.
Next we describe how these operations are done using secret shared inputs.

\noindent{\textbf{Fixed point multiplication between two secrets}.} Note that
multiplication between two fixed point numbers $\share{x}, \share{y}$ where
$x,y \in \verb|fixed(k, f)|$ gets us $z = x \cdot y \cdot 2^{2f}$. One can
now simply output $\Decode(z, 2f)$ if this is the last multiplication.
To avoid $z$ overflow in $\Z_{2^k}$ and allow
subsequent computations on $z$ call $z \asn \mathsf{trunc}(2 \cdot k, f)(z)$.

\noindent{\textbf{Multiplication by (clear) integer scalars}}.
To multiply a public scalar
$c$ with $\share{x}$ where $x \in \verb|fixed(k, f)|$ we output
$c \cdot \share{x}$. In this way when we decode the product
this becomes (over the ring)
$\Decode(c \cdot x \cdot 2^f) = c \cdot x$. Again, this operation is going to
be successful iff. the product $c \cdot x$ fits in $\Z_{2^k}$.

\noindent{\textbf{Multiplication by (clear) floating scalars}}. To multiply a
public floating point scalar $c$ with $\share{x}$ we compute $\hat{c} \asn
\Encode(c, f)$ and then set the output $\share{c \cdot x} =
\mathsf{trunc}(\hat{c} \cdot \share{x}, f)$. Note that $\hat{c} \cdot \share{x}$
can be done with local computations.

% \subsubsection{Division}
% TBD, SecureNN has some leakage, check MP-SPDZ - there was no description in
% ABY3 on fixed point division.

\subsection{Secret fixed point division by a public value}

Suppose here that $x = \bar{x} \cdot 2^f$ is a fixed point encoding of $\bar{x}$. 
Given two inputs $\share{x}$ and a public value $c$ which is known to all parties the goal is to output $\share{x /c}$.
This can be done naively by first encoding the public constant into its fixed point representation and compute the division between two fixed point numbers accordingly \cite{FC:CatSax10,SCN:CatDeH10}.

A slightly more efficient way is for all parties to compute an encoding of $1/c$
and then do a local multiplication i.e. $\share{t} \asn \mathsf{ceil}(1/c) \cdot 2^f \cdot \share{x}$.
Since $t$ evaluates to $1/c \cdot 2^f \cdot \bar{x} \cdot 2^f$ then truncating \verb|trunc|($\share{t}$) computes a fixed point encoding of $\share{x/c}$ which is want we wanted in the first place.



% \begin{Boxfig}{Sampling functionality}{fig:Sample}{Functionality
$\Func_{\mathsf{Sample}}$}
\textbf{Additive}: On input $(\mathsf{SampleAdd}, d, S, P_i)$ from party
$P_i$ the functionality samples a random tensor $\vx \sample S^d$. It also
generates two keys $k_1, k_2$ associated to $\vx$ such that when expanded using PRGs
they sum up to $\vx$, i.e. $\PRG(k_1) + \PRG(k_2) = \vx$.
\end{Boxfig}

% \import{key-distribution}{protocol-flow}
\import{replicated}{extended-protos}
\end{document}



