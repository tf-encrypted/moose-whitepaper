\section{Fixed-point Arithmetic}
\label{sec:fixed-point}


Our cryptographic protocols natively operate on ring elements, and we must use ring arithmetic to emulate other data types and computations. For this reason we use fixed-point arithmetic instead of floating point arithmetic, which results in more efficient computations. In this section we briefly outline fixed-point arithmetic and how to emulate it with ring arithmetic in $\mathbb{Z}_{2^k}$, which in turn can be emulated with our cryptographic protocols and implemented using integer instructions. For efficiency reasons, we only consider $k = 64$ and $k = 128$, since these give us modulus reductions for free when using wrapping instructions.

\subsection{Fixed-point Numbers}
We let $\mathtt{fixed}(k, f)$ be the subset of the real numbers
$$
\{x \in \mathbb{R} : x = \bar{x} \cdot 2^{-f}, \bar{x} \in \Z_{2^k} \}
$$
that have a fixed-point representation in $\Z_{2^k}$ using $f$ bits fractional precision. Here, $\Z_{2^k}$ is considered as the $k$ bit signed integers $\{ \bar{x} : -2^{k-1} \leq \bar{x} < 2^{k-1} \}$.

\subsection{Encoding and Decoding}
To encode a number $x \in \mathbb{R}$ into its fixed-point representation $\bar{x} \in \Z_{2^k}$, we compute $\mathsf{Encode}(x, f) = \floor{x \cdot 2^f}$ where the result of flooring is cast as an integer. To decode a number $\bar{x}$, we compute $\mathsf{Decode}(\bar{x}, f) = x \cdot 2^{-f}$ where $\bar{x}$ is first cast as a floating point.

\subsection{Addition}
To add two fixed-point numbers $\bar{x}$ and $\bar{y}$ where $x,y \in \mathtt{fixed}(k, f)$, we can simply compute $\bar{z} = \bar{x} + \bar{y}$ using $\Z_{2^k}$ arithmetic. Note that $\mathsf{Decode}(\bar{z}) = x + y$ if and only if $x + y \in \mathtt{fixed}(k, f)$.

\subsection{Multiplication}
To multiply two fixed-point numbers $\bar{x}$ and $\bar{y}$ where $x, y \in
\mathtt{fixed}(k, f)$, we can first compute $\bar{z} = \bar{x} \cdot \bar{y}$
using $\Z_{2^k}$ arithmetic. However, in general $z \in \mathtt{fixed}(k,
2f)$ and we see that the fractional precision of the numbers grow by $f$ bits
with each multiplication. To circumvent this we introduce the
$\mathsf{Trunc}(\bar{z})$ operation which computes $\bar{z} / 2^{f}$ over the
integers, which brings us back to $z \in \mathtt{fixed}(k, f)$. In the main
body the $\mathsf{Trunc}$ operation is replaced by a probabilistic variant
$\mathsf{TruncPR}$ that allows for an error in the least significant bit in
exchange for better performance.

% \noindent{\textbf{Fixed point multiplication between two secrets}.} Note that
% multiplication between two fixed point numbers $\share{x}, \share{y}$ where
% $x,y \in \verb|fixed(k, f)|$ gets us $z = x \cdot y \cdot 2^{2f}$. One can
% now simply output $\Decode(z, 2f)$ if this is the last multiplication.
% To avoid $z$ overflow in $\Z_{2^k}$ and allow
% subsequent computations on $z$ call $z \asn \mathsf{trunc}(2 \cdot k, f)(z)$.

% \noindent{\textbf{Multiplication by (clear) integer scalars}}.
% To multiply a public scalar
% $c$ with $\share{x}$ where $x \in \verb|fixed(k, f)|$ we output
% $c \cdot \share{x}$. In this way when we decode the product
% this becomes (over the ring)
% $\Decode(c \cdot x \cdot 2^f) = c \cdot x$. Again, this operation is going to
% be successful iff. the product $c \cdot x$ fits in $\Z_{2^k}$.

% \noindent{\textbf{Multiplication by (clear) floating scalars}}. To multiply a
% public floating point scalar $c$ with $\share{x}$ we compute $\hat{c} \asn
% \Encode(c, f)$ and then set the output $\share{c \cdot x} =
% \mathsf{trunc}(\hat{c} \cdot \share{x}, f)$. Note that $\hat{c} \cdot \share{x}$
% can be done with local computations.

% \subsubsection{Division}
% TBD, SecureNN has some leakage, check MP-SPDZ - there was no description in
% ABY3 on fixed point division.


\subsection{Inverse and Division}

Given a fixed-point $\bar{x}$ we can compute its inverse $1/x$ over the reals by
computing the integer division of $\bar{1} \cdot 2^f = 1 \cdot 2^{2f}$ and $\bar{x}$,
so that the result $\bar{y} = \bar{1} \cdot 2^f / \bar{x} = y \cdot 2^f$ for some
$y \in \mathtt{fixed}(k, f)$. Likewise, the division of $\bar{x}$ by $\bar{y}$ can
be computed as the integer division of $\bar{x} \cdot 2^f$ by $\bar{y}$. Note that
no truncation is needed.

In order to compute $N/D$ using fixed-point arithmetic only relying on additions
and multiplications there are two main lines of thought: Newton-Raphson method
or Goldschmidt method.

They both start with finding a good approximation of the reciprocal $D' = 1/D$
where $D' = [0.5, 1]$ in order to find a function $f(X) = 1/X - D = 0$. The
Newton-Raphson iteration is given by $X_{i+1} = X_i - \frac{f(X_i)}{f'(X_i)} =
X_i(2 - DX_i)$. The most straightforward choice would be to use Newton-Raphson
method for finding $a/b$ described below.

The Goldschmidt method starts with the same reciprocal approximation like
Newton-Raphson. However, instead of applying the result of each iteration on the
denomitor, one iteration has an effect on the nominator as well.  Once an
approximation $D_{-1} \approx 1/D$ is found then one Goldschmidt iteration is the
following:
\begin{eqnarray*}
X_i &=& 2 - D_{i-1} \\
D_i &=& X_i \cdot D_{i-1} \\
N_i &=& X_i \cdot N_{i-1}
\end{eqnarray*}
where $N_0 = N$ and $D_0 = D$.

\msubsubsection{$\mathsf{NewtonDivFloat}(N,D)$:}
% We first give division for when $a, b \in \Qk{f}$ and $b$ is in the clear.
\begin{enumerate}
  \item $D' \asn D$.
  \item $p = 1$.
  \item While $D' > 1$:
  \begin{enumerate}
    \item $D' \asn D' / 2$.
    \item $p \asn p \cdot 2$.
  \end{enumerate}
  \item $X \asn 48/17 - 32/17 \cdot D'$. (initial approximation)
  \item For $i \in [1, \theta]$:
  \begin{enumerate}
      \item $X \asn X \cdot (2 - D'X)$.
  \end{enumerate}
  \item Return $N / p \cdot x$.
\end{enumerate}


% TODO(Dragos) uncomment this for square root
% \subsection{Square root}

% A protocol for accurately computing the square root function $f(x) = \sqrt{x}$ on a
% secret was first described by Liedel in \cite{Liedel2012SecureDC}, and subsequently
% generalized in \cite{ACNS:AlySma19}. In both cases, we start with an initial
% approximation (of $\sqrt{x}$ or $\frac{1}{\sqrt{x}}$) that is then refined by
% Newton-Raphson or Goldschmidt iterations. Since Liedel's algorithm does not
% necessarily work for all fixedpoint configurations, we only implement the algorithm
% from \cite{ACNS:AlySma19} for computing square root on fixedpoints. Note this
% function is referred to as SimplifiedFxSqrt in the SCALE-MAMBA package.
